\documentclass[11pt,openany,a4paper]{scrartcl}

\usepackage{indentfirst}
\usepackage{amsmath,amsthm,amssymb,amsfonts,amsopn}
\usepackage{mathtext}
\usepackage{enumitem}
\usepackage[T1,T2A]{fontenc}
\usepackage[utf8]{inputenc}
\usepackage[english,russian]{babel}
\usepackage[intlimits]{mathtools}
\usepackage[makeroom]{cancel}
\usepackage{titletoc}
\renewcommand{\bfdefault}{sbc}
\usepackage{ccfonts,eulervm,microtype}
\usepackage{enumitem}
\usepackage{tikz}
\usetikzlibrary{arrows}
\usetikzlibrary{calc}

\tikzset{
    pil/.style={
           ->,
           thick,
           shorten <=2pt,
           shorten >=2pt},
    axis/.style={very thick, ->, >=stealth'},
}

\usepackage[portrait,a4paper,margin=2.5cm,headsep=5mm]{geometry}

\author{Ф. Л. Бахарев \thanks{Конспект подготовлен студентом Яскевичем С. В.}}
\title{Функциональный анализ}

\theoremstyle{plain}
\newtheorem{theorem}{Теорема}[section]
\newtheorem{corollary}[theorem]{Следствие}
\newtheorem{proposition}[theorem]{Предложение}
\newtheorem{lemma}[theorem]{Лемма}
\newtheorem{exercise}[theorem]{Упражнение}

\theoremstyle{definition}
\newtheorem{definition}[theorem]{Определение}
\newtheorem{remark}[theorem]{Замечание}
\newtheorem{example}[theorem]{Пример}
\newtheorem{examples}[theorem]{Примеры}
\newtheorem{num}[theorem]{}

\newcommand\mb{\mathbb}
\newcommand\real{\mb R}
\newcommand{\complex}{\mb C}
\newcommand\eqdef{\mathrel{\stackrel{\makebox[0pt]{\mbox{\normalfont\tiny def}}}{=}}}
\newcommand\lparagraph[1]{\paragraph{#1}\mbox{}\\}
\newcommand{\pd}[2]{\frac{\partial #1}{\partial #2}}
\newcommand{\uto}{\rightrightarrows}
\newcommand{\underto}[1]{\xrightarrow[#1]{}}
\newcommand{\overto}[1]{\xrightarrow{#1}}
\newcommand{\dif}{\, \mathrm d}
\newcommand{\bigslant}[2]{{\raisebox{.2em}{$#1$}\left/\raisebox{-.2em}{$#2$}\right.}}
\DeclareMathOperator{\Ree}{Re}
\DeclareMathOperator{\Img}{Im}
\DeclareMathOperator{\Arg}{Arg}
\DeclareMathOperator{\dist}{dist}
\DeclareMathOperator{\const}{const}
\DeclareMathOperator{\Ln}{Ln}
\DeclareMathOperator{\card}{card}
\DeclareMathOperator{\Ker}{Ker}

\begin{document}

\maketitle

\tableofcontents

\pagebreak

\section{Линейное нормированное пространство}

\begin{definition}
    Линейное множество $L$ над полем скаляров $\real$ ($\complex$) — множество с
    операциями сложения и умножения на скаляр, удовлетворяющее свойствам:
    \begin{enumerate}
        \item $(x + y) + z = x + (y + z)$ $\forall x,y,z \in L$
        \item $x + y = y + x$ $\forall x,y,z \in L$
        \item Существует элемент $0$ такой, что $x + 0 = x$ $\forall x \in L$
        \item Для любого $x \in L$ существует обратный элемент по сложению $-x$ такой, что
        $-x + x = 0$
        \item $\lambda(\mu x) = (\lambda \mu) x$ $\forall \lambda, \mu
        \in \real(\complex)$, $x \in L$
        \item $\lambda(x + y) = \lambda x + \lambda y$ $\forall \lambda \in 
        \real(\complex)$, $x,y \in L$
        \item $(\lambda + \mu)x = \lambda x + \mu y$ $\forall \lambda,
        \mu \in \real(\complex)$, $x,y \in L$
    \end{enumerate}
\end{definition}

\begin{definition}
    $\varphi: L \to \real$ называется нормой, если:
    \begin{enumerate}
        \item $\varphi(x + y) \leqslant \varphi(x) + \varphi(y)$
        $\forall x, y \in L$
        \item $\varphi(\lambda x) = |\lambda|\varphi(x)$
        $\forall x \in L$, $\lambda \in \real(\complex)$
        \item $\varphi(x) \geqslant 0$ $\forall x \in L$
        \item $\varphi (x) = 0 \iff x = 0$
    \end{enumerate}

    Если выполнены только первых три свойства, то $\varphi$ называется полунормой.
\end{definition}
\begin{remark}
\mbox{}
    \begin{enumerate}
        \item $\rho (x, y) = \varphi(x - y)$ — метрика.
        \item Если на пространстве задана норма $\|\cdot\|$, то $X = (L, \varphi)$ —
        нормированное пространство.
    \end{enumerate}
\end{remark}

\begin{definition}
    $x_n \to x$ в $X$, если $\|x_n - x\| \to 0$ при $n \to \infty$, то есть $\forall
    \varepsilon > 0$ $\exists N$:~$\forall n > N$ $\|x_n - x\| < \varepsilon$
\end{definition}

\begin{definition}
    $\{x_n\} \subset X$ — фундаментальная последовательность (сходящаяся в себе,
    последовательность Коши), если $\|x_n - x_m\| \underto{m,n \to \infty} 0$, то есть
    $\forall \varepsilon > 0$ $\exists N$: $\forall m,n > N$ $\|x_m - x_m\| < \varepsilon$
\end{definition}

\begin{remark}
    $x_n \to x \implies \{x_n\}$ — фундаментальная. Обратное, вообще говоря, неверно.
\end{remark}
\begin{definition}
    Нормированное пространство $X$ называется полным, если из фундаментальности
    последовательности следует существование предела.
\end{definition}

\begin{definition}
    Пусть $x_n \in X$. $\sum\limits_{j = 1}^\infty x_j$ сходится, если
    $S_n = \sum\limits_{j = 1}^n x_j$ имеет предел $\lim S_n = S$. $S$ называется
    суммой ряда.
\end{definition}

\begin{definition}
    Ряд $\sum\limits_{j = 1}^\infty x_j$ называется \emph{сходящимся абсолютно},
    если $\sum\limits_{j = 1}^\infty \|x_j\|$ сходится.
\end{definition}
\begin{remark}
    Из абсолютной сходимости не следует обычная сходимость.
\end{remark}

$S_n$ сходится $\iff |S_n - S_m| \to 0$. Пусть $C_n = \sum\limits_{j = 1}^n \|x\|$.
$C_n$ сходится $\iff |C_n - C_m|~\to~0$.
Если мы хотим, чтобы сходимость $S_n$ была равносильна
$\|S_n - S_m\| \to 0$, то нам нужна полнота пространства.

\begin{definition}
    Полное линейное нормированное пространство называется банаховым пространством (в честь
    польского математика Стефана Банаха).
\end{definition}
\begin{examples}
\mbox{}
    \begin{itemize}
        \item Евклидово пространство: $\real^n$ с нормой
        $\|x\| = |x| = \sqrt[n]{|x_1|^2 + \ldots + |x_n|^2}$ — то же, что
        $\ell_n^2$ с нормой $\|\cdot \|_2$;
        \item $\ell_n^1 = (\real^n, \|\cdot\|_1)$, где
        $\|x\|_1 = |x_1| + \ldots + |x_n|$;
        \item $\ell_n^\infty = (\real^n, \|\cdot\|_\infty)$,
        где $\|x\|_\infty = \max\limits_{1 \leqslant j \leqslant n} |x_j|$;
        \item $\ell_n^p = (\real^n, \|\cdot\|_p$,
        $\|x\|_p = \bigg(\sum\limits_{j=1}^n |x_j|^p\bigg)^{\frac{1}{p}}$,
        $p \geqslant 1$;
        \item $C(\overline \Omega)$ с нормой
        $\|x\| = \max\limits_{t \in \overline \Omega} |x(t)|$, где
        $\Omega$ — область в $\real^m$. $\overline\Omega$ — замыкание
        $\Omega$. Ясно, что $\overline \Omega$ — компакт в
        $\real^m$.
    \end{itemize}
\end{examples}
\begin{exercise}
    Верно ли, что $\|x\|_p \underto{p \to \infty} \|x\|_\infty$?
\end{exercise}

\begin{theorem}
    Пространство $C(\overline \Omega)$ полно.
\end{theorem}
\begin{proof}
    Рассмотрим фундаментальную последовательность $x_n \in C(\overline \Omega)$.
    $$
    \forall \varepsilon > 0 \exists N: \forall k, n > N \|x_k - x_n\| =
    \max_{t \in \overline \Omega} |x_n(t) - x_k(t)| < \varepsilon
    $$

    Возьмём $t \in \overline \Omega$. $\{x_n(t)\}$ — числовая последовательность.
    Тогда получаем $|x_n(t) - x_k(t)| < \varepsilon$, отсюда $\{x_n(t)\}$ —
    фундаментальна, значит существует $\lim\limits_{n \to \infty} x_n(t) = x(t)$.

    Проверим, что $\max\limits_{t \in \overline \Omega} |x_n(t) - x(t)|
    \underto{n \to \infty} 0$, т. е. $x_n \uto x$ на $\overline \Omega$.
    Заметим, что $\forall k, n > N$ $|x_k(t) - x_n(t)| < \varepsilon \implies
    |x(t) - x_n(t)| \leqslant \varepsilon$.

    Почему же $x$ непрерывна? Потому что равномерный предел непрерывных функций
    непрерывен.

\end{proof}

Пусть $[a, b] \subset \real$.
Рассмотрим пространство дифференцируемых функций $C^1[a, b]$.
Какую норму на нём выбрать?
\begin{itemize}
    \item $\varphi_1(x) = \max\limits_{t \in [a, b]} |x(t)|$;
    \item $\varphi_2(x) = \max\limits_{t \in [a, b]} |x'(t)|$;
    \item $\varphi_3(x) = \varphi_1(x) + \varphi_2(x)$;
    \item $\varphi_4(x) = |x(a)| + \max\limits_{t \in [a, b]} |x'(t)|$.
\end{itemize}

Заметим, что $\varphi_2$ нормой вообще не является, а $\varphi_1$ не даёт полноты
пространства.
\pagebreak
\begin{theorem}
    \begin{enumerate}
        \item Пространство $(C^1 [a, b], \varphi_1)$ не полно.
        \item Пространство $(C^1 [a, b], \varphi_3)$ полно.
    \end{enumerate}
\end{theorem}

\begin{proof}
    Докажем первое утверждение.

    \emph{Первый аргумент}. $x$ — производная непрерывная на $[a, b]$, негладкая.
        По теореме Вейерштрасса для любого $\varepsilon > 0$ существует многочлен
        $P$ такой, что $\max\limits_{[a, b]} |P - x| < \varepsilon$

    \emph{Второй аргумент}. Пусть $[a, b] = [-1, 1]$, $x(t) = |t| \notin C^1[a, b]$,
    $x^\varepsilon(t) = |t|^{1 + \varepsilon} \in C^1[a, b]$. $\max |x(t) - x^\varepsilon(t)| 
    \underto{\varepsilon \to 0} 0$.

    Для доказательства второго утверждения возьмём $x_n \in C^1[a, b]$ —
    последовательность, фундаментальную относительно $\varphi_3$.
    $$
    \varphi_3(x_n -x_k) \underto{n, k \to \infty} 0 \implies
    \begin{cases}
        \varphi_1(x_n - x_k) \to 0\\
        \varphi_2(x_n - x_k) \to 0
    \end{cases}
    \implies \exists x \in C[a, b], y \in C[a, b]
    $$

    $$
    \begin{cases}
        \varphi_1(x_n - x) \to 0 \iff x_n \uto x \text{ на } [a, b]\\
        \varphi_1(x'_n - y) \to 0 \iff x'_n \uto y \text{ на } [a, b]
    \end{cases}
    \implies x \in C^1[a, b], x' = y
    $$

    Отсюда $\varphi_3(x_n - x) \to 0$
\end{proof}

\section{Пространства Лебега}

\lparagraph{Неравенство Гёльдера}

Рассмотрим $(T, \mu)$ — пространство с мерой, $x, y$ — измеримые функции, и числа
$p, q > 0$ — сопряжённые показатели, т. е. $\frac{1}{p} + \frac{1}{q} = 1$. 
Тогда верно неравенство:
$$
\int\limits_T |x(t)y(t)|\dif \mu(t) \leqslant
\bigg(\int\limits_T |x(t)|^p\dif \mu(t)\bigg)^{\frac{1}{p}}
\bigg(\int\limits_T |y(t)|^q\dif \mu(t)\bigg)^\frac{1}{q}
$$

\lparagraph{Неравенство Минковского}

Если $(T, \mu)$ — пространство с мерой, $x, y$ — измеримые функции,
$p \geqslant 1$, то верно неравенство:
$$
\bigg(\int\limits_T |x(t)|^p\dif \mu(t)\bigg)^{\frac{1}{p}} +
\bigg(\int\limits_T |y(t)|^q\dif \mu(t)\bigg)^\frac{1}{q} \geqslant
\int\limits_T |x(t)y(t)|\mathrm d\mu(t)
$$

Обозначение: $\|x\|_p = (\int\limits_T |x|^p)^\frac{1}{p}$.

\begin{remark}
    Частный случай — $p = q = 2$. Тогда неравенство Гёльдера оказывается
    неравенством Коши-Буняковского-Шварца:
    $$
    \int\limits_T |x(t)|\cdot|y(t)|\dif \mu(t) \leqslant
    \bigg(\int\limits_T |x(t)|^2\dif \mu(t)\bigg)^{\frac{1}{2}}
    \bigg(\int\limits_T |y(t)|^2\dif\mu(t)\bigg)^\frac{1}{2}
    $$
\end{remark}
\begin{remark}
    Пусть $T = \mb N$, и если $M \subset \mb N$,
    то $\#M = \card M$ — количество элементов $M$ — будет мерой.
    Рассмотрим функцию $x: \mb N \to k$, где $k$ — некоторое поле скаляров.
    Мы помним, что функция из натуральных чисел называется последовательностью.
    Как можно вычислять $\int\limits_{\mb N}x(n)\mathrm d\#(n)$? Ясно,
    что такой интеграл — это ряд $\sum\limits_{n \in \mb N} x(n)$, а суммируемые 
    функции в этом случае будут абсолютно сходящимися рядами.
    Неравенство Гёльдера будет выглядеть так:
    $$
    \sum_{n \in \mb N} |x_n||y_n| \leqslant
    \bigg(\sum_{n \in \mb N} |x_n|^p\bigg)^\frac{1}{p}
    \bigg(\sum_{n \in \mb N} |y_n|^p\bigg)^\frac{1}{p}
    $$
    А неравенство Минковского — так:
    $$
    \bigg(\sum_{n \in \mb N} |x_n|^p\bigg)^\frac{1}{p} +
    \bigg(\sum_{n \in \mb N} |y_n|^p\bigg)^\frac{1}{p} \geqslant
    \bigg(\sum_{n \in \mb N} |x_n||y_n|\bigg)^\frac{1}{p}
    $$
\end{remark}

\begin{definition}
    Пространство Лебега $\mathcal{L}^p(T, \mu)$ — это множество
    $\{x\, \big| \int\limits_T |x|^p\dif\mu <
    \infty\}$.
    Оно линейно: $x, y \in \mathcal{L}^p \implies
    x + y \in \mathcal{L}^p$ и $\lambda y \in \mathcal{L}^p$
\end{definition}

Заметим, что $\|x\|_p = \bigg(\int\limits_T |x|^p\mathrm d\mu\bigg)^\frac{1}{p}$ — полунорма
на $\mathcal{L}^p(T, \mu)$. Если $\|x\|_p = 0$, то $x = 0$ почти везде.

Чтобы получить норму, введём следующее отношение эквивалентности:
$$
x_1 \sim x_2 \text{ если } x_1 - x_2 = 0 \text{ почти везде.}
$$
Тогда
$$
\bigslant{\mathcal{L}^p(T, \mu)}{\sim} = L^p(T, \mu)
$$
— это настоящее пространство Лебега. В дальнейшем мы будем считать функции, 
отличающиеся на множестве меры нуль, одинаковыми.

\begin{remark}
    Пусть $T \subset \real^n$, $\mu = \lambda$ — мера Лебега. Тогда будем 
    обозначать $L^p(T, \mu) = L^p(T)$.
\end{remark}

\begin{theorem}
    Пространство $L^p(T, \mu)$ полно при $p \geqslant 1$.
\end{theorem}

\begin{example}
    Рассмотрим $L^2(0, +\infty)$ и $L^1(0, +\infty)$. Какое из этих пространств 
    является вложением в другое? Возьмём функцию $x(t) = \frac{1}{t + 1}$.
    $$
    \int\limits_0^\infty \frac{1}{t + 1}\mathrm dt = \infty
    $$
    $$
    \int\limits_0^\infty \frac{1}{(t+1)^2} \mathrm dt < \infty
    $$
    Отсюда видно, что $L^2(0, +\infty) \not\subset L^1(0, +\infty)$. Легко 
    придумать и пример, доказывающий отсутствие включения в обратную сторону.
\end{example}

\begin{theorem}[О вложенности пространств $L^p$]
Пусть $1 \leqslant p_1 < p_2 \leqslant \infty$. Тогда:
    \begin{enumerate}
        \item $\ell^{p_1} \subset \ell^{p_2}$.
        \item Если $(T, \mu)$ — пространство с мерой, $\mu(T) < \infty$, то
        $L^{p_1}(T, \mu) \supset L^{p_2}(T, \mu)$
    \end{enumerate}
\end{theorem}
\pagebreak
\begin{proof}
\mbox{}
    \begin{enumerate}
        \item Пусть $x = (x_1, x_2, x_3, \ldots)$. Хотим проверить, что
        $x \in \ell^{p_1} \implies x \in \ell^{p_2}$.
        $$
        \sum\limits_{j = 1}^\infty |x_j|^{p_1} < \infty \implies
        \exists N\quad \forall j > N\quad |x_j| < 1 \implies
        |x_j|^{p_1} < |x_j|^{p_2}
        $$
        $$
        \sum\limits_{j = N + 1}^\infty |x_j|^{p_1} >
        \sum\limits_{j = N + 1}^\infty |x_j|^{p_2} \implies
        \sum\limits_{j = 1}^\infty |x_j|^{p_2} < \infty \implies x \in \ell^{p_2}
        $$
        \item Для доказательства второго пункта достаточно применить неравенство 
        Гёльдера.
    \end{enumerate}
\end{proof}

\section{Непрерывность. Сжимающее отображение}

\begin{definition}
    Возьмём отображение $F: X \to Y$, где $X$ и $Y$ — линейные нормированные
    пространства. $F$ называется непрерывным в точке $x_0$, если:
    $$
    \forall \varepsilon > 0\quad
    \exists \delta > 0:\quad \forall x: \|x - x_0\| < \delta\quad
    \|F(x) -F(x_0)\| < \varepsilon
    $$
    $F$ называется непрерывным, если оно непрерывно во всех точках $X$.
\end{definition}

\begin{example}
        $X = Y = C[0, 1]$, $\|x\|_{C[0, 1]} =
        \max\limits_{t \in [0, 1]} |x(t)|$. Рассмотрим отображение
        $(F(x))(t) = \int\limits_0^t x(s)\dif s$ и докажем, что оно 
        непрерывно.
        $$
        \|F(x_1) - F(x_2)\| = \max_{t \in [0, 1]}
        \bigg|\int\limits_0^t x_1(s)\dif s - \int\limits_0^t x_2(s)\dif s\bigg|
        \leqslant
        $$
        $$
        \leqslant
        \max_{t \in [0, 1]} \int\limits_0^t |x_1(s) - x_2(s)|\dif s \leqslant
        \max_{t \in [0, 1]} t \cdot \|x_1 - x_2\| = \|x_1 - x_2\|
        $$
        Достаточно взять $\delta = \varepsilon$ и всё доказано.
\end{example}

\begin{definition}
    Отображение $F: X \to Y$ называется липшицевым, если существует такое $C$, что
    для всех $x_1, x_2 \in X$ выполнено $\|F(x_1) - F(X_2)\| \leqslant
    C\cdot\|x_1 - x_2\|$
\end{definition}

Заметим, что из липшицевости отображения следует его непрерывность. Достаточно 
взять $\delta = \frac{\varepsilon}{C}$.

\begin{definition}
    Отображение $F: X \to Y$ называется сжимающим, если существует такое
    $\gamma < 1$, что $\forall x_1, x_2 \in X$ выполнено $\|F(x_1) - F(x_2)\|
    \leqslant
    \gamma \|x_1 - x_2\|$.
\end{definition}

\begin{theorem}[Банаха о неподвижной точке]
    Если пространство $X$ — полное, а отображение $F$ — сжимающее, то существует
    единственный элемент $x_\ast \in X$ такой, что $F(x_\ast) = x_\ast$. Этот
    элемент называется неподвижной точкой.
\end{theorem}
\begin{proof}
    Докажем существование. Возьмём \emph{траекторию} точки $x_1$:
    $$
    x_1, \underbrace{F(x_1)}_{x_2}, \underbrace{F(F(x_1))}_{x_3}, \ldots,
    \text{ т. е. } x_{n+1} = F(x_n)
    $$
    $$
    \|x_{n+1} - x_n\| = \|F(x_n) - F(x_{n-1})\| \leqslant \gamma \|x_n - x_{n-1}\|
    \leqslant \gamma^2\|x_{n-1} - x_{n-2}\| \leqslant \ldots \leqslant
    \gamma^{n+1}\underbrace{\|x_2 - x_1\|}_{\alpha}
    $$
    Таким образом, при $m > n$:
    $$
    \|x_m - x_n\| \leqslant \|x_m - x_{m-1}\| + \|x_{m-1} - x_{m-2}\| + \ldots
    + \|x_{n+1} - x_n\| \leqslant \alpha\gamma^{m-2} + \alpha\gamma^{m-3} +\ldots +
    $$
    $$
    + \alpha\gamma^{n-1} \leqslant \sum\limits_{j = n-1}^\infty \alpha\gamma^j =
    \alpha\gamma^{n-1}\frac{1}{1-\gamma} \underto{n \to \infty} 0
    $$
    Отсюда получаем, что $\{x_n\}$ фундаментальна, а значит существует
    $\lim\limits_{n \to \infty} x_n$. Обозначим его за $x_\ast$. Ясно, что это
    и будет неподвижная точка.
    
    Докажем единственность. Пусть $x_\ast$ и $x^\ast$ — две неподвижные точки.
    Тогда:
    $$
    \underbrace{\|F(x_\ast) - F(x^\ast)\|}_{\leqslant \gamma\|x_\ast - x^\ast\|} =
    \|x_\ast - x^\ast\|
    $$
    Отсюда $\|x_\ast - x^\ast\| = 0$, что и требовалось.
\end{proof}

\begin{theorem}
    Пусть пространство $X$ — полное, $F: X \to X$ и существует $n$ такое, что
    $F^n$ — сжимающее. Тогда существует единственная точка $x_\ast$ такая, что
    $F(x_\ast) = x_\ast$.
\end{theorem}
\begin{proof}
    Если $F^n$ сжимающее, то существует (и единственна) неподвижная точка:
    $F^n(x_\ast) = x_\ast$.
    Условие теоремы подразумевает, что если $F$ переводит точку $x_\ast$ в
    некоторую точку $x_1$, которую, в свою очередь, переводит в $x_2$, то через
    $n$ итераций точка $x_{n-1}$ снова переходит в $x_\ast$. Отсюда следует,
    что точки $x_1,\ldots,x_{n-1}$ — тоже неподвижные точки $F^n$. Но по теореме
    Банаха такая точка у $F^n$ только одна, следовательно,
    $x_\ast = x_1 = x_2 = \ldots = x_{n-1}$.
\end{proof}
\begin{example}[Интегральное уравнение Фредгольма I рода]
    Пусть нам даны функции $K(s,t)$ и $a(t)$. Мы хотим найти функцию $x(t)$,
    удовлетворяющую уравнению:
    $$
    x(t) = a(t) + \int\limits_{s_1}^{s_2} K(s,t) x(s) \dif s
    $$
    Будем рассматривать частный случай, в котором $K \in C([0,1]\times[0,1])$,
    $a \in C[0,1]$. Задача — найти $x \in C[0,1]$ такое, что
    $$
    x(t) = a(t) + \int\limits_0^t K(s,t) x(s) \dif s
    $$
\end{example}
\begin{proposition}
    Это уравнение имеет единственное решение.
\end{proposition}
\begin{proof}
    Рассмотрим отображение $F: C[0, 1] \to C[0, 1]$.
    $$
    (F(x))(t) = a(t) + \int\limits_0^t K(s, t)x(s)\dif s
    $$
    Заметим, что оно, вообще говоря, не является сжимающим.
    Рассмотрим также $(F_0(x))(t) = \int\limits_0^t K(s, t)x(s)\dif s$.
    
    Обратим внимание на несколько важных свойств:
    \begin{itemize}
        \item $F_0(x) - F_0(y) = F_0(x - y)$
        \item $F(x) - F(y) = F_0(x) - F_0(y)$
        \item $F^n(x) - F^n(y) = F(F^{n-1}(x) - F^{n-1}(y)) =
        F_0(F^{n-1}(x)) - F_0(F^{n-1}(y)) = F_0(F^{n-1}(x) - F^{n-1}(y)) =
        F_0^n(x-y)$
    \end{itemize}
    $$
    (F_0(x - y))(t) = \int\limits_0^t K(s_1, t)(x(s_1) - y(s_1))\dif s_1
    $$
    $$
    (F_0^2(x-y))(t) = \int\limits_0^t K(s_2, t)
    \int\limits_0^{s_2} K(s_1, s_2)(x(s_1) - y(s_1))\dif s_1\dif s_2
    $$
    $$
    \cdots
    $$
    $$
    (F_0^n(x-y))(t) = \int\limits_0^t K(s_n, t)
    \int\limits_0^{s_n} K(s_{n-1}, s_n)\int\limits_0^{s_{n-1}}\ldots
    \int\limits_0^{s_2} K(s_1, s_2)(x(s_1) - y(s_1))\dif s_1\dif s_2\ldots
    \dif s_n
    $$
    Получаем:
    $$
    \|F_0^n(x-y)\| = \max_{t \in [0, 1]} |(F_0^n(x-y))(t)|
    \leqslant M^n\|x-y\|\max_{t \in [0, 1]} \int\limits_0^t\int\limits_0^{s_n}
    \int\limits_0^{s_{n-1}}\ldots\int\limits_0^{s_3}\int\limits_0^{s_2}
    \dif s_1\dif s_2 \ldots \dif s_n \leqslant \frac{M^n}{n!}\|x-y\|
    $$
    Здесь $M = \max |K|$. Коэффициент $\frac{M^n}{n!}$ стремится к нулю, а
    это значит, что $F_0^n$ — сжимающее, следовательно, существует неподвижная
    точка.
\end{proof}

\begin{example}
    Допустим, что мы хотим решить дифференциальное уравнение
    $y'(t) = a(t)y(t) + b(t)$, $y(0) = y_0$, $a,b \in C[0, 1]$ на промежутке
    $[0, 1]$. Это
    уравнение имеет единственное решение $y \in C^1[0, 1]$. Как это доказать?
    Рассмотрим интегральное уравнение:
    $$
    x(t) = \int\limits_0^t a(s)x(s)\dif s + B(t)
    $$
    По предыдущей теореме существует $x \in C[0,1]$, решающее это
    уравнение. Для этого уравнения также верны утверждения:
    \begin{itemize}
        \item $x'(t) = a(t)x(t) + b(t)$, где $b(t) = B'(t)$;
        \item $x(0) = B(0)$.
    \end{itemize}
    Для решения исходной задачи достаточно выбрать $B$ такое, что $B' = b$ и
    $B(0) = y_0$. Откуда взять непрерывную дифференцируемость $y$?
    $$
    b\in C[0,1] \implies B\in C^1[0,1],
    $$
    $$
    \quad x\in C[0,1],\, a\in C[0,1] \implies
    \int\limits_0^t xa \in C^1[0,1]
    $$
    Таким образом всё доказано.
\end{example}

\section{Линейные операторы}
\begin{definition}
    Пусть $X$, $Y$ — линейные нормированные пространства над одним полем скаляров.
    Отображение $U: X \to Y$ называется линейным, если:
    \begin{enumerate}
        \item $U(x_1+x_2) = U(x_1) + U(x_2)$ $\forall x_1, x_2 \in X$
        \item $U(\lambda x) = \lambda U(x)$, где $\lambda$ — скаляр, $x \in X$
    \end{enumerate}
\end{definition}
\begin{remark}
    Ясно, что выполнение обоих этих свойств равносильно
    $U(\lambda_1x_1 + \lambda_2x_2) = \lambda_1U(x_1) + \lambda_2U(x_2)$.
\end{remark}
\begin{remark}
    В дальнейшем будем обозначать $U(x)$ как $Ux$.
\end{remark}
\begin{proposition}[Свойства линейных отображений]
\mbox{}
    \begin{enumerate}
        \item $U(0) = 0$;
        \item $U(\sum\limits_{j=1}^n \lambda_j x_j) = \sum\limits_{j=1}^n
        \lambda_j Ux_j$;
        \item\label{image_linearity} Если $M \subset X$ — линейное множество,
        то множество
        $U(M)$ линейно в $Y$. Если $M \subset X$ — выпуклое множество,
        то множество $U(M)$ выпукло в $Y$;
        \item\label{preimage_linearity} Если $N \in Y$ — линейное (выпуклое),
        то $U^{-1}(N)$ — линейное (выпуклое). Частный случай: если $N = \{0\}$, то множество
        $U^{-1}(N) = U^{-1}(\{0\}) = \Ker U$ — линейное в $X$;
        \item $\Ker U = \{0\} \iff U$ инъективно;
        \item\label{inverse_linearity} Если $U$ — линейная биекция, то
        $U^{-1}$ — линейное;
        \item Пусть $U_1, U_2: X \to Y$ — линейные. Тогда $U_1 + U_2$,
        $\lambda U_1$ тоже линейны;
        \item Если $X \overto{U} Y \overto{V} Z$, то композиция $V\circ U$
        линейна.
    \end{enumerate}
\end{proposition}
\begin{definition}
    Множество $M$ называется выпуклым, если для любых $x_1,x_2 \in M$ отрезок
    $[x_1, x_2]$ лежит в $M$.
\end{definition}
\begin{proof}[Доказательство предложения]
    Докажем выпуклость в свойстве \ref{image_linearity}.
    $$
    y_1, y_2 \in U(M) \implies \exists x_1, x_2 \in M:\, Ux_1 = y_1,\,
    Ux_2 = y_2
    $$
    $$
    \lambda y_1 + (1-\lambda)y_2 = \lambda Ux_1 + (1 - \lambda)Ux_2 =
    U(\underbrace{\lambda x_1 + (1-\lambda)x_2}_{\in M}) \in U(M)
    $$
    
    В свойстве \ref{preimage_linearity}:
    $$
    x_1, x_2 \in U^{-1}(N) \implies Ux_1, Ux_2 \in N \implies
    \forall \lambda_1, \lambda_2\quad \lambda_1 Ux_1 + \lambda_2 Ux_2 \in N
    \implies
    $$
    $$
    \implies U(\lambda_1 x_1 + \lambda_2 x_2) \in N \implies
    \lambda_1x_1 + \lambda_2x_2 \in U^{-1}(N)
    $$
    
    В свойстве \ref{inverse_linearity} биективность $U$ означает, что
    $\forall y_1, y_2$ $\exists x_1, x_2$ такие, что $Ux_1 = y_1$,
    $Ux_2 = y_2$. Отсюда $U^{-1}(y_1+y_2) = U^{-1}(Ux_1 + Ux_2) =
    U^{-1}(U(x_1 + x_2)) = x_1 + x_2 = U^{-1}(x_1) + U^{-1}(x_2)$.
    
    Доказательства остальных свойств тривиальны.
\end{proof}

\begin{theorem}[Эквивалентные условия непрерывности линейного отображения]
    Пусть $U:~X \to Y$ — линейный оператор. Тогда следующие утверждения
    эквивалентны:
    \begin{enumerate}
        \item $U$ непрерывен;
        \item $U$ непрерывен в нуле;
        \item Образ любого ограниченного множества ограничен;
        \item Существует $C$ такое, что $\forall x\in X$
        выполняется $\|U_x\|_Y = C\|x\|_X$.
    \end{enumerate}
\end{theorem}
\begin{proof}
\mbox{}
    \begin{itemize}
        \item $1 \Rightarrow 2$. Tривиально.
        \item $4 \Rightarrow 1$. $\|Ux_1 - Ux_2\| \leqslant
        C\|x_1 - x_2\|$. Это влечёт липшицевость и, как следствие, непрерывность.
        \item $2 \Rightarrow 3$. Непрерывность в нуле означает, что
        $\forall \varepsilon > 0$ $\exists \delta > 0$ такое, что
        $\|x\|<\delta \implies \|Ux\|<\varepsilon$. Ограниченность множества $M$ в $X$
        означает, что $\exists R:$ $N \subset B_R(0)=\{\|x\|\leqslant R\}$.
        Таким образом, $x\in M \implies \|x\| \leqslant R$.
        $\|\frac{\delta}{2R}x\| \leqslant \frac{\delta}{2} < \delta \implies
        \|U(\frac{\delta}{2R}x)\| < \varepsilon$. Отсюда
        $\|Ux| \leqslant \frac{\varepsilon\cdot 2R}{\delta} \implies Ux
        \in B_{\frac{\varepsilon\cdot 2R}{\delta}}(0)$. То есть, $U(M)$ ограничено.
        \item $3 \Rightarrow 4$. $B_1(0)$ — ограниченное множество. Тогда
        $U(B_1(0))$ — ограничено, т. е. существует такое $C$, что
        $U(B_1(0)) \subset B_C(0)$. Если $\|x\|\leqslant 1$, то $\|Ux\| \leqslant C$.
        Теперь возьмём произвольное $x$. $x' = \frac{x}{\|x\|} \in B_1(0) \implies
        \|Ux'\| \leqslant C$. Но $\|Ux\| = \|U\big(\frac{x}{\|x\|}\big)\| =
        \frac{1}{\|x\|} \cdot \|Ux\|$. Отсюда $\|Ux\|\leqslant C\|x\|$.
    \end{itemize} 
\end{proof}

\begin{definition}
    Пусть $U: X \to Y$ — линейный непрерывный оператор. Тогда нормой оператора $U$
    называется величина $\|U\| = \inf\,\{C\,\big|\,\|Ux\| \leqslant C \|x\|\}$.
\end{definition}
\begin{remark}
    В формулировке определения инфимум и минимум совпадают (это можно доказать, перейдя к 
    пределу в неравенстве $\|Ux\| \leqslant C \|x\|$).
\end{remark}
\begin{remark}
    Выполнено неравенство $\|Ux\|_Y \leqslant \|U\|\cdot \|x\|_X$. В частности,
    $\frac{\|Ux\|_Y}{\|x\|_X} \leqslant \|U\|$ $\forall x \in X$, т. е. можно записать
    $\|U\| = \sup\limits_{x \neq 0}\frac{\|Ux\|}{\|x\|}$. 
\end{remark}
\begin{theorem}[Об эквивалентных способах определения нормы оператора]
     Пусть $U: X \to Y$ — линейный непрерывный оператор. Тогда:
     $$
     \|U\| = \underbrace{\sup\limits_{x \neq 0}\frac{\|Ux\|}{\|x\|}}_A =
     \underbrace{\sup\limits_{\|x\|\leqslant 1} \|U_x\|}_B =
     \underbrace{\sup\limits_{\|x\| < 1} \|U_x\|}_C =
     \underbrace{\sup\limits_{\|x\| = 1} \|U_x\|}_D
     $$
\end{theorem}
\begin{remark}
    Так как замкнутость и ограниченность, вообще говоря, неравносильна компактности
    (за исключением конечномерных пространств),
    в $\sup\limits_{\|x\|\leqslant 1} \|Ux\|$ максимум может и не достигаться.
\end{remark}
\begin{proof}[Доказательство теоремы]
    Очевидно, что $B \geqslant C$ и $B \geqslant D$.
    $$
    B =\sup\limits_{\|x\|\leqslant 1,\,x \neq 0} \|Ux\| \leqslant
    \sup\limits_{\|x\|\leqslant 1,\,x \neq 0} \frac{\|Ux\|}{\|x\|} \leqslant
    \sup\limits_{x \neq 0} \frac{\|Ux\|}{\|x\|} = A
    $$
    Докажем, что $D \geqslant A$. Возьмём $x'=\frac{x}{\|x\|}$, тогда $\|x'\|=1$ и
    $\|Ux\|\leqslant D$. $\|U(\frac{x}{\|x\|})\| = \frac{\|Ux\|}{\|x\|}$.
    Итак, $\frac{\|Ux\|}{\|x\|} \leqslant D$, тогда и
    $\sup\limits_{x \neq 0}\frac{\|Ux\|}{\|x\|}$.
    Осталось проверить, что $C \geqslant A$. Возьмём $x \neq 0$, $\varepsilon > 0$.
    Рассмотрим $x' = \frac{x}{\|x\|(1 + \varepsilon)}$. Тогда $\|x\| < 1$. Отсюда
    следует, что $\|Ux'\| \leqslant C \implies \frac{\|Ux\|}{\|x\|(1+\varepsilon)}
    \leqslant C \implies \frac{\|Ux\|}{\|x\|}\leqslant C(1 + \varepsilon)$. Устремив
    $\varepsilon \to 0$, получим
    $A = \sup\limits_{x \neq 0} \frac{\|Ux\|}{\|x\|} \leqslant C$.
\end{proof}

\section{Пространства линейных непрерывных операторов}
\begin{definition}
    Пусть $X$, $Y$ — линейные нормированные пространства над одним полем скаляров. Возьмём
    $B(X,Y) = \{U: X \to Y,\, U \text{ — линейно, непрерывно}\}$.
    Это \emph{линейное пространство}.
\end{definition}
\begin{theorem}[О свойствах операторной нормы]
    $U,V \in B(X, Y)$.
    \begin{enumerate}
        \item $\|U\| \geqslant 0$, $\|U\| = 0 \iff U = 0$;
        \item $\|\lambda U\| = |\lambda|\|U\|$ ($\lambda$ — скаляр);
        \item $\|U + V\| \leqslant \|U\| + \|V\|$;
        \item $W \in B(Y,Z)$. $WU \in B(X, Z)$, $\|WU\| \leqslant \|W\|\|U\|$.
    \end{enumerate}
\end{theorem}
\begin{proof}
\mbox{}
    \begin{enumerate}
        \item Неотрицательность очевидна. Если $\|U\|=0$, то $\|Ux\|\leqslant
        0\cdot \|x\| \implies \|Ux\| = 0$ $\forall x$;
        \item $\|\lambda U\| = \sup\limits_{\|x\|=1} \|(\lambda U)(x)\| =
         \sup\limits_{\|x\|=1} |\lambda|\|Ux\| = |\lambda| \sup\limits_{\|x\|=1}
         \|U_x\| =
         |\lambda|\|U\|$;
         \item $x \in X$. $\|(U+V)(x) = \|Ux + Vx\| \leqslant \|Ux\| + \|Vx\|
         \leqslant \|U\|\|x\| + \|V\|\|x\| = (\|U\| + \|V\|)\|x\|$
         \item $x \in X$. $\|(WU)(x)\| = \|W(U(x))\| \leqslant \|W\|\cdot \|Ux\|
         \leqslant \|W\|\|U\|\|x\|$.
    \end{enumerate}
\end{proof}
\begin{theorem}[О полноте пространства операторов]
    Если $Y$ полно, то $B(X, Y)$ полно.
\end{theorem}
\begin{proof}
    Возьмём фундаментальную последовательность линейных непрерывных отображений
    $U_n \in B(X, Y)$, то есть $\|U_n - U_m\| \underto{m,n \to \infty} 0$:
    $\forall \varepsilon>0$ $\exists N:$ $\forall m,n > N$ $\|U_n - U_m\| < \varepsilon$.
    Это означает, что $\|(U_n - U_m)(x)\| \leqslant \varepsilon \|x\|$. Следовательно,
    $\{U_nx\}$ фундаментальна в $Y$. Обозначим $Ux = \lim\limits_{n \to \infty}
    U_nx$. Мы хотим проверить, что $U$ непрерывно, линейно и что есть сходимость.
    \begin{enumerate}
        \item (Линейность $U$). $U(\alpha_1x_1 + \alpha_2x_2) = \lim\limits_{n \to \infty}
        U_n(\alpha_1x_1 + \alpha_2x_2) = 
        \alpha_1 \lim U_nx_1 + \alpha_2\lim U_nx_2 = \alpha_1 Ux_1 +
        \alpha_2 Ux_2$
        \item (Нерерывность $U$). Возьмём любое $\varepsilon > 0$, $N$, $\forall m,n>N$,
        $\forall x\in X$. $\|U_nx - U_mx\| \leqslant \varepsilon \|x\| \implies
        \|Ux - U_mx\| \leqslant \varepsilon\|x\|$.
        $\|Ux\| = \|(Ux - U_mx) + U_mx\| \leqslant \|(Ux - U_mx)\| + \|U_mx\| \leqslant
        \varepsilon \|x\| + \|U_m\|\|x\|$. Отсюда $\|U\| \leqslant \varepsilon +
        \|U_m\|$.
        \item (Сходимость $U_n$ к $U$).
        $\forall \varepsilon > 0$ $\exists N$: $\forall m,n>N$ $\forall x\in X$ $
        \|U_nx - U_mx\| \leqslant \varepsilon \|x\|$. Устремив $n$ к бесконечности, 
        получим: $\forall \varepsilon > 0$ $\exists N$: $\forall m > N$
        $\forall x\in X$ $\|Ux - U_mx\| = \|(U - U_m)(x)\| \leqslant \|x\|$
        $\implies \|U - U_m\| \leqslant \varepsilon$. Итак,
        $\forall \varepsilon > 0$ $\exists N$: $\forall m > N$
        $\|U - U_m\| \leqslant \varepsilon$, т. е. $U_n \to U$ в $B(X, Y)$.
    \end{enumerate}
\end{proof}

Следует отметить важный частный случай.
\begin{definition}
    $B (X, \text{поле скаляров }) = X^\ast$ называется
    \emph{сопряжённым пространством к $X$}. $f \in X^\ast$ называется \emph{линейным
    непрерывным функционалом}.
\end{definition}

\begin{example}
    $\|\cdot\|: X \to \real$ — функционал.
\end{example}

Норма функционала определяется как $\|f\| = \inf\,\{C\,\big|\,|f(x)| \leqslant C\|x\|\} =
\sup\limits_{x \neq 0}\frac{|f(x)|}{\|x\|} = \sup\limits_{\|x\|=1} |f(x)|$.

\end{document}
