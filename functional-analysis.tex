\documentclass[11pt,openany,a4paper]{scrartcl}

\usepackage{indentfirst}
\usepackage{amsmath,amsthm,amssymb,amsfonts,amsopn}
\usepackage{mathtext}
\usepackage{enumitem}
\usepackage[T1,T2A]{fontenc}
\usepackage[utf8]{inputenc}
\usepackage[english,russian]{babel}
\usepackage[intlimits]{mathtools}
\usepackage[makeroom]{cancel}
\usepackage{titletoc}
\renewcommand{\bfdefault}{sbc}
\usepackage{ccfonts,eulervm,microtype}
\usepackage{enumitem}
\usepackage[makeroom]{cancel}

\renewcommand\thesubsection{\arabic{subsection}}

\usepackage{tikz}
\usetikzlibrary{arrows}
\usetikzlibrary{calc}

\tikzset{
    pil/.style={
           ->,
           thick,
           shorten <=2pt,
           shorten >=2pt},
    axis/.style={very thick, ->, >=stealth'},
}

\usepackage[portrait,a4paper,margin=2.5cm,headsep=5mm]{geometry}

\author{Ф. Л. Бахарев \thanks{Конспект подготовлен студентом Яскевичем С. В.}}
\title{Функциональный анализ}

\theoremstyle{plain}
\newtheorem{theorem}{Теорема}[subsection]
\newtheorem{corollary}[theorem]{Следствие}
\newtheorem{proposition}[theorem]{Предложение}
\newtheorem{lemma}[theorem]{Лемма}
\newtheorem{exercise}[theorem]{Упражнение}

\theoremstyle{definition}
\newtheorem{definition}[theorem]{Определение}
\newtheorem{remark}[theorem]{Замечание}
\newtheorem{example}[theorem]{Пример}
\newtheorem{examples}[theorem]{Примеры}
\newtheorem{num}[theorem]{}

\newcommand\mb{\mathbb}
\newcommand\real{\mb R}
\newcommand{\complex}{\mb C}
\newcommand\eqdef{\mathrel{\stackrel{\makebox[0pt]{\mbox{\normalfont\tiny def}}}{=}}}
\newcommand\lparagraph[1]{\paragraph{#1}\mbox{}\\}
\newcommand{\pd}[2]{\frac{\partial #1}{\partial #2}}
\newcommand{\uto}{\rightrightarrows}
\newcommand{\underto}[1]{\xrightarrow[#1]{}}
\newcommand{\overto}[1]{\xrightarrow{#1}}
\newcommand{\dif}{\, \mathrm d}
\newcommand{\bigslant}[2]{{\raisebox{.2em}{$#1$}\left/\raisebox{-.2em}{$#2$}\right.}}
\newcommand\ol{\overline}
\DeclareMathOperator{\Ree}{Re}
\DeclareMathOperator{\Img}{Im}
\DeclareMathOperator{\Arg}{Arg}
\DeclareMathOperator{\dist}{dist}
\DeclareMathOperator{\const}{const}
\DeclareMathOperator{\Ln}{Ln}
\DeclareMathOperator{\card}{card}
\DeclareMathOperator{\Ker}{Ker}
\DeclareMathOperator{\sign}{sign}
\DeclareMathOperator{\diam}{diam}
\DeclareMathOperator{\Lin}{Lin}

\begin{document}

\maketitle

\tableofcontents

\pagebreak

\subsection{Линейное нормированное пространство}

\begin{definition}
    Линейное множество $L$ над полем скаляров $\real$ ($\complex$) — множество с
    операциями сложения и умножения на скаляр, удовлетворяющее свойствам:
    \begin{enumerate}
        \item $(x + y) + z = x + (y + z)$ $\forall x,y,z \in L$
        \item $x + y = y + x$ $\forall x,y,z \in L$
        \item Существует элемент $0$ такой, что $x + 0 = x$ $\forall x \in L$
        \item Для любого $x \in L$ существует обратный элемент по сложению $-x$ такой, что
        $-x + x = 0$
        \item $\lambda(\mu x) = (\lambda \mu) x$ $\forall \lambda, \mu
        \in \real(\complex)$, $x \in L$
        \item $\lambda(x + y) = \lambda x + \lambda y$ $\forall \lambda \in 
        \real(\complex)$, $x,y \in L$
        \item $(\lambda + \mu)x = \lambda x + \mu y$ $\forall \lambda,
        \mu \in \real(\complex)$, $x,y \in L$
    \end{enumerate}
\end{definition}

\begin{definition}
    $\varphi: L \to \real$ называется нормой, если:
    \begin{enumerate}
        \item $\varphi(x + y) \leqslant \varphi(x) + \varphi(y)$
        $\forall x, y \in L$
        \item $\varphi(\lambda x) = |\lambda|\varphi(x)$
        $\forall x \in L$, $\lambda \in \real(\complex)$
        \item $\varphi(x) \geqslant 0$ $\forall x \in L$
        \item $\varphi (x) = 0 \iff x = 0$
    \end{enumerate}

    Если выполнены только первых три свойства, то $\varphi$ называется полунормой.
\end{definition}
\begin{remark}
\mbox{}
    \begin{enumerate}
        \item $\rho (x, y) = \varphi(x - y)$ — метрика.
        \item Если на пространстве задана норма $\|\cdot\|$, то $X = (L, \varphi)$ —
        нормированное пространство.
    \end{enumerate}
\end{remark}

\begin{definition}
    $x_n \to x$ в $X$, если $\|x_n - x\| \to 0$ при $n \to \infty$, то есть $\forall
    \varepsilon > 0$ $\exists N$:~$\forall n > N$ $\|x_n - x\| < \varepsilon$
\end{definition}

\begin{definition}
    $\{x_n\} \subset X$ — фундаментальная последовательность (сходящаяся в себе,
    последовательность Коши), если $\|x_n - x_m\| \underto{m,n \to \infty} 0$, то есть
    $\forall \varepsilon > 0$ $\exists N$: $\forall m,n > N$ $\|x_m - x_m\| < \varepsilon$
\end{definition}

\begin{remark}
    $x_n \to x \implies \{x_n\}$ — фундаментальная. Обратное, вообще говоря, неверно.
\end{remark}
\begin{definition}
    Нормированное пространство $X$ называется полным, если из фундаментальности
    последовательности следует существование предела.
\end{definition}

\begin{definition}
    Пусть $x_n \in X$. $\sum\limits_{j = 1}^\infty x_j$ сходится, если
    $S_n = \sum\limits_{j = 1}^n x_j$ имеет предел $\lim S_n = S$. $S$ называется
    суммой ряда.
\end{definition}

\begin{definition}
    Ряд $\sum\limits_{j = 1}^\infty x_j$ называется \emph{сходящимся абсолютно},
    если $\sum\limits_{j = 1}^\infty \|x_j\|$ сходится.
\end{definition}
\begin{remark}
    Из обычной сходимости не следует абсолютная сходимость.
\end{remark}

$S_n$ сходится $\iff |S_n - S_m| \to 0$. Пусть $C_n = \sum\limits_{j = 1}^n \|x\|$.
$C_n$ сходится $\iff |C_n - C_m|~\to~0$.
Если мы хотим, чтобы сходимость $S_n$ была равносильна
$\|S_n - S_m\| \to 0$, то нам нужна полнота пространства.

\begin{definition}
    Полное линейное нормированное пространство называется банаховым пространством (в честь
    польского математика Стефана Банаха).
\end{definition}
\begin{examples}
\mbox{}
    \begin{itemize}
        \item Евклидово пространство: $\real^n$ с нормой
        $\|x\| = |x| = \sqrt[n]{|x_1|^2 + \ldots + |x_n|^2}$ — то же, что
        $\ell_n^2$ с нормой $\|\cdot \|_2$;
        \item $\ell_n^1 = (\real^n, \|\cdot\|_1)$, где
        $\|x\|_1 = |x_1| + \ldots + |x_n|$;
        \item $\ell_n^\infty = (\real^n, \|\cdot\|_\infty)$,
        где $\|x\|_\infty = \max\limits_{1 \leqslant j \leqslant n} |x_j|$;
        \item $\ell_n^p = (\real^n, \|\cdot\|_p$,
        $\|x\|_p = \bigg(\sum\limits_{j=1}^n |x_j|^p\bigg)^{\frac{1}{p}}$,
        $p \geqslant 1$;
        \item $C(\overline \Omega)$ с нормой
        $\|x\| = \max\limits_{t \in \overline \Omega} |x(t)|$, где
        $\Omega$ — область в $\real^m$. $\overline\Omega$ — замыкание
        $\Omega$. Ясно, что $\overline \Omega$ — компакт в
        $\real^m$.
    \end{itemize}
\end{examples}
\begin{exercise}
    Верно ли, что $\|x\|_p \underto{p \to \infty} \|x\|_\infty$?
\end{exercise}

\begin{theorem}
    Пространство $C(\overline \Omega)$ полно.
\end{theorem}
\begin{proof}
    Рассмотрим фундаментальную последовательность $x_n \in C(\overline \Omega)$.
    $$
    \forall \varepsilon > 0\quad \exists N: \forall k, n > N\quad \|x_k - x_n\| =
    \max_{t \in \overline \Omega} |x_n(t) - x_k(t)| < \varepsilon
    $$

    Возьмём $t \in \overline \Omega$. $\{x_n(t)\}$ — числовая последовательность.
    Тогда получаем $|x_n(t) - x_k(t)| < \varepsilon$, отсюда $\{x_n(t)\}$ —
    фундаментальна, значит существует $\lim\limits_{n \to \infty} x_n(t) = x(t)$.

    Проверим, что $\max\limits_{t \in \overline \Omega} |x_n(t) - x(t)|
    \underto{n \to \infty} 0$, т. е. $x_n \uto x$ на $\overline \Omega$.
    Заметим, что $\forall k, n > N$ $|x_k(t) - x_n(t)| < \varepsilon \implies
    |x(t) - x_n(t)| \leqslant \varepsilon$.

    Почему же $x$ непрерывна? Потому что равномерный предел непрерывных функций
    непрерывен.

\end{proof}

Пусть $[a, b] \subset \real$.
Рассмотрим пространство дифференцируемых функций $C^1[a, b]$.
Какую норму на нём выбрать?
\begin{itemize}
    \item $\varphi_1(x) = \max\limits_{t \in [a, b]} |x(t)|$;
    \item $\varphi_2(x) = \max\limits_{t \in [a, b]} |x'(t)|$;
    \item $\varphi_3(x) = \varphi_1(x) + \varphi_2(x)$;
    \item $\varphi_4(x) = |x(a)| + \max\limits_{t \in [a, b]} |x'(t)|$.
\end{itemize}

Заметим, что $\varphi_2$ нормой вообще не является, а $\varphi_1$ не даёт полноты
пространства.
\pagebreak
\begin{theorem}
    \begin{enumerate}
        \item Пространство $(C^1 [a, b], \varphi_1)$ не полно.
        \item Пространство $(C^1 [a, b], \varphi_3)$ полно.
    \end{enumerate}
\end{theorem}

\begin{proof}
    Докажем первое утверждение.

    \emph{Первый аргумент}. $x$ — производная непрерывная на $[a, b]$, негладкая.
        По аппроксимационной теореме Вейерштрасса для любого $\varepsilon > 0$ существует многочлен
        $P$ такой, что $\max\limits_{[a, b]} |P - x| < \varepsilon$

    \emph{Второй аргумент}. Пусть $[a, b] = [-1, 1]$, $x(t) = |t| \notin C^1[a, b]$,
    $x^\varepsilon(t) = |t|^{1 + \varepsilon} \in C^1[a, b]$. $\max |x(t) - x^\varepsilon(t)| 
    \underto{\varepsilon \to 0} 0$.

    Для доказательства второго утверждения возьмём $x_n \in C^1[a, b]$ —
    последовательность, фундаментальную относительно $\varphi_3$.
    $$
    \varphi_3(x_n -x_k) \underto{n, k \to \infty} 0 \implies
    \begin{cases}
        \varphi_1(x_n - x_k) \to 0\\
        \varphi_2(x_n - x_k) \to 0
    \end{cases}
    \implies \exists x \in C^1[a, b], y \in C^1[a, b]
    $$

    $$
    \begin{cases}
        \varphi_1(x_n - x) \to 0 \iff x_n \uto x \text{ на } [a, b]\\
        \varphi_1(x'_n - y) \to 0 \iff x'_n \uto y \text{ на } [a, b]
    \end{cases}
    \implies x \in C^1[a, b], x' = y
    $$

    Отсюда $\varphi_3(x_n - x) \to 0$
\end{proof}

\subsection{Пространства Лебега}

\lparagraph{Неравенство Гёльдера}

Рассмотрим $(T, \mu)$ — пространство с мерой, $x, y$ — измеримые функции, и числа
$p, q > 0$ — сопряжённые показатели, т. е. $\frac{1}{p} + \frac{1}{q} = 1$. 
Тогда верно неравенство:
$$
\int\limits_T |x(t)y(t)|\dif \mu(t) \leqslant
\bigg(\int\limits_T |x(t)|^p\dif \mu(t)\bigg)^{\frac{1}{p}}
\bigg(\int\limits_T |y(t)|^q\dif \mu(t)\bigg)^\frac{1}{q}
$$

\lparagraph{Неравенство Минковского}

Если $(T, \mu)$ — пространство с мерой, $x, y$ — измеримые функции,
$p \geqslant 1$, то верно неравенство:
$$
\bigg(\int\limits_T |x(t)|^p\dif \mu(t)\bigg)^{\frac{1}{p}} +
\bigg(\int\limits_T |y(t)|^q\dif \mu(t)\bigg)^\frac{1}{q} \geqslant
\int\limits_T |x(t) + y(t)|\mathrm d\mu(t)
$$

Обозначение: $\|x\|_p = (\int\limits_T |x|^p)^\frac{1}{p}$.

\begin{remark}
    Частный случай — $p = q = 2$. Тогда неравенство Гёльдера оказывается
    неравенством Коши-Буняковского-Шварца:
    $$
    \int\limits_T |x(t)|\cdot|y(t)|\dif \mu(t) \leqslant
    \bigg(\int\limits_T |x(t)|^2\dif \mu(t)\bigg)^{\frac{1}{2}}
    \bigg(\int\limits_T |y(t)|^2\dif\mu(t)\bigg)^\frac{1}{2}
    $$
\end{remark}
\begin{remark}
    Пусть $T = \mb N$, и если $M \subset \mb N$,
    то $\#M = \card M$ — количество элементов $M$ — будет мерой.
    Рассмотрим функцию $x: \mb N \to k$, где $k$ — некоторое поле скаляров.
    Мы помним, что функция из натуральных чисел называется последовательностью.
    Как можно вычислять $\int\limits_{\mb N}x(n)\mathrm d\#(n)$? Ясно,
    что такой интеграл — это ряд $\sum\limits_{n \in \mb N} x(n)$, а суммируемые 
    функции в этом случае будут абсолютно сходящимися рядами.
    Неравенство Гёльдера будет выглядеть так:
    $$
    \sum_{n \in \mb N} |x_n||y_n| \leqslant
    \bigg(\sum_{n \in \mb N} |x_n|^p\bigg)^\frac{1}{p}
    \bigg(\sum_{n \in \mb N} |y_n|^p\bigg)^\frac{1}{p}
    $$
    А неравенство Минковского — так:
    $$
    \bigg(\sum_{n \in \mb N} |x_n|^p\bigg)^\frac{1}{p} +
    \bigg(\sum_{n \in \mb N} |y_n|^p\bigg)^\frac{1}{p} \geqslant
    \bigg(\sum_{n \in \mb N} |x_n + y_n|\bigg)^\frac{1}{p}
    $$
\end{remark}

\begin{definition}
    Пространство Лебега $\mathcal{L}^p(T, \mu)$ — это множество
    $\{x\, \big| \int\limits_T |x|^p\dif\mu <
    \infty\}$.
    Оно линейно: $x, y \in \mathcal{L}^p \implies
    x + y \in \mathcal{L}^p$ и $\lambda y \in \mathcal{L}^p$
\end{definition}

Заметим, что $\|x\|_p = \bigg(\int\limits_T |x|^p\mathrm d\mu\bigg)^\frac{1}{p}$ — полунорма
на $\mathcal{L}^p(T, \mu)$. Если $\|x\|_p = 0$, то $x = 0$ почти везде.

Чтобы получить норму, введём следующее отношение эквивалентности:
$$
x_1 \sim x_2 \text{ если } x_1 - x_2 = 0 \text{ почти везде.}
$$
Тогда
$$
\bigslant{\mathcal{L}^p(T, \mu)}{\sim} = L^p(T, \mu)
$$
— это настоящее пространство Лебега. В дальнейшем мы будем считать функции, 
отличающиеся на множестве меры нуль, одинаковыми.

\begin{remark}
    Пусть $T \subset \real^n$, $\mu = \lambda$ — мера Лебега. Тогда будем 
    обозначать $L^p(T, \mu) = L^p(T)$.
\end{remark}

\begin{theorem}
    Пространство $L^p(T, \mu)$ полно при $p \geqslant 1$.
\end{theorem}

\begin{example}
    Рассмотрим $L^2(0, +\infty)$ и $L^1(0, +\infty)$. Какое из этих пространств 
    является вложением в другое? Возьмём функцию $x(t) = \frac{1}{t + 1}$.
    $$
    \int\limits_0^\infty \frac{1}{t + 1}\mathrm dt = \infty
    $$
    $$
    \int\limits_0^\infty \frac{1}{(t+1)^2} \mathrm dt < \infty
    $$
    Отсюда видно, что $L^2(0, +\infty) \not\subset L^1(0, +\infty)$. Легко 
    придумать и пример, доказывающий отсутствие включения в обратную сторону.
\end{example}

\begin{theorem}[О вложенности пространств $L^p$]
Пусть $1 \leqslant p_1 < p_2 \leqslant \infty$. Тогда:
    \begin{enumerate}
        \item $\ell^{p_1} \subset \ell^{p_2}$.
        \item Если $(T, \mu)$ — пространство с мерой, $\mu(T) < \infty$, то
        $L^{p_1}(T, \mu) \supset L^{p_2}(T, \mu)$
    \end{enumerate}
\end{theorem}
\pagebreak
\begin{proof}
\mbox{}
    \begin{enumerate}
        \item Пусть $x = (x_1, x_2, x_3, \ldots)$. Хотим проверить, что
        $x \in \ell^{p_1} \implies x \in \ell^{p_2}$.
        $$
        \sum\limits_{j = 1}^\infty |x_j|^{p_1} < \infty \implies
        \exists N\quad \forall j > N\quad |x_j| < 1 \implies
        |x_j|^{p_1} > |x_j|^{p_2}
        $$
        $$
        \sum\limits_{j = N + 1}^\infty |x_j|^{p_1} >
        \sum\limits_{j = N + 1}^\infty |x_j|^{p_2} \implies
        \sum\limits_{j = 1}^\infty |x_j|^{p_2} < \infty \implies x \in \ell^{p_2}
        $$
        \item Для доказательства второго пункта достаточно применить неравенство 
        Гёльдера.
    \end{enumerate}
\end{proof}

\subsection{Непрерывность. Сжимающее отображение}

\begin{definition}
    Возьмём отображение $F: X \to Y$, где $X$ и $Y$ — линейные нормированные
    пространства. $F$ называется непрерывным в точке $x_0$, если:
    $$
    \forall \varepsilon > 0\quad
    \exists \delta > 0:\quad \forall x: \|x - x_0\| < \delta\quad
    \|F(x) -F(x_0)\| < \varepsilon
    $$
    $F$ называется непрерывным, если оно непрерывно во всех точках $X$.
\end{definition}

\begin{example}
        $X = Y = C[0, 1]$, $\|x\|_{C[0, 1]} =
        \max\limits_{t \in [0, 1]} |x(t)|$. Рассмотрим отображение
        $(F(x))(t) = \int\limits_0^t x(s)\dif s$ и докажем, что оно 
        непрерывно.
        $$
        \|F(x_1) - F(x_2)\| = \max_{t \in [0, 1]}
        \bigg|\int\limits_0^t x_1(s)\dif s - \int\limits_0^t x_2(s)\dif s\bigg|
        \leqslant
        $$
        $$
        \leqslant
        \max_{t \in [0, 1]} \int\limits_0^t |x_1(s) - x_2(s)|\dif s \leqslant
        \max_{t \in [0, 1]} t \cdot \|x_1 - x_2\| = \|x_1 - x_2\|
        $$
        Достаточно взять $\delta = \varepsilon$ и всё доказано.
\end{example}

\begin{definition}
    Отображение $F: X \to Y$ называется липшицевым, если существует такое $C$, что
    для всех $x_1, x_2 \in X$ выполнено $\|F(x_1) - F(X_2)\| \leqslant
    C\cdot\|x_1 - x_2\|$
\end{definition}

Заметим, что из липшицевости отображения следует его непрерывность. Достаточно 
взять $\delta = \frac{\varepsilon}{C}$.

\begin{definition}
    Отображение $F: X \to Y$ называется сжимающим, если существует такое
    $\gamma < 1$, что $\forall x_1, x_2 \in X$ выполнено $\|F(x_1) - F(x_2)\|
    \leqslant
    \gamma \|x_1 - x_2\|$.
\end{definition}

\begin{theorem}[Банаха о неподвижной точке]
    Если пространство $X$ — полное, а отображение $F$ — сжимающее, то существует
    единственный элемент $x_\ast \in X$ такой, что $F(x_\ast) = x_\ast$. Этот
    элемент называется неподвижной точкой.
\end{theorem}
\begin{proof}
    Докажем существование. Возьмём \emph{траекторию} точки $x_1$:
    $$
    x_1, \underbrace{F(x_1)}_{x_2}, \underbrace{F(F(x_1))}_{x_3}, \ldots,
    \text{ т. е. } x_{n+1} = F(x_n)
    $$
    $$
    \|x_{n+1} - x_n\| = \|F(x_n) - F(x_{n-1})\| \leqslant \gamma \|x_n - x_{n-1}\|
    \leqslant \gamma^2\|x_{n-1} - x_{n-2}\| \leqslant \ldots \leqslant
    \gamma^{n-1}\underbrace{\|x_2 - x_1\|}_{\alpha}
    $$
    Таким образом, при $m > n$:
    $$
    \|x_m - x_n\| \leqslant \|x_m - x_{m-1}\| + \|x_{m-1} - x_{m-2}\| + \ldots
    + \|x_{n+1} - x_n\| \leqslant \alpha\gamma^{m-2} + \alpha\gamma^{m-3} +\ldots +
    $$
    $$
    + \alpha\gamma^{n-1} \leqslant \sum\limits_{j = n-1}^\infty \alpha\gamma^j =
    \alpha\gamma^{n-1}\frac{1}{1-\gamma} \underto{n \to \infty} 0
    $$
    Отсюда получаем, что $\{x_n\}$ фундаментальна, а значит существует
    $\lim\limits_{n \to \infty} x_n$. Обозначим его за $x_\ast$. Ясно, что это
    и будет неподвижная точка.
    
    Докажем единственность. Пусть $x_\ast$ и $x^\ast$ — две неподвижные точки.
    Тогда:
    $$
    \underbrace{\|F(x_\ast) - F(x^\ast)\|}_{\leqslant \gamma\|x_\ast - x^\ast\|} =
    \|x_\ast - x^\ast\|
    $$
    Отсюда $\|x_\ast - x^\ast\| = 0$, что и требовалось.
\end{proof}

\begin{theorem}
    Пусть пространство $X$ — полное, $F: X \to X$ и существует $n$ такое, что
    $F^n$ — сжимающее. Тогда существует единственная точка $x_\ast$ такая, что
    $F(x_\ast) = x_\ast$.
\end{theorem}
\begin{proof}
    Если $F^n$ сжимающее, то существует (и единственна) неподвижная точка:
    $F^n(x_\ast) = x_\ast$.
    Условие теоремы подразумевает, что если $F$ переводит точку $x_\ast$ в
    некоторую точку $x_1$, которую, в свою очередь, переводит в $x_2$, то через
    $n$ итераций точка $x_{n-1}$ снова переходит в $x_\ast$. Отсюда следует,
    что точки $x_1,\ldots,x_{n-1}$ — тоже неподвижные точки $F^n$. Но по теореме
    Банаха такая точка у $F^n$ только одна, следовательно,
    $x_\ast = x_1 = x_2 = \ldots = x_{n-1}$.
\end{proof}
\begin{example}[Интегральное уравнение Фредгольма I рода]
    Пусть нам даны функции $K(s,t)$ и $a(t)$. Мы хотим найти функцию $x(t)$,
    удовлетворяющую уравнению:
    $$
    x(t) = a(t) + \int\limits_{s_1}^{s_2} K(s,t) x(s) \dif s
    $$
    Будем рассматривать частный случай, в котором $K \in C([0,1]\times[0,1])$,
    $a \in C[0,1]$. Задача — найти $x \in C[0,1]$ такое, что
    $$
    x(t) = a(t) + \int\limits_0^t K(s,t) x(s) \dif s
    $$
\end{example}
\begin{proposition}
    Это уравнение имеет единственное решение.
\end{proposition}
\begin{proof}
    Рассмотрим отображение $F: C[0, 1] \to C[0, 1]$.
    $$
    (F(x))(t) = a(t) + \int\limits_0^t K(s, t)x(s)\dif s
    $$
    Заметим, что оно, вообще говоря, не является сжимающим.
    Рассмотрим также $(F_0(x))(t) = \int\limits_0^t K(s, t)x(s)\dif s$.
    
    Обратим внимание на несколько важных свойств:
    \begin{itemize}
        \item $F_0(x) - F_0(y) = F_0(x - y)$
        \item $F(x) - F(y) = F_0(x) - F_0(y)$
        \item $F^n(x) - F^n(y) = F(F^{n-1}(x) - F^{n-1}(y)) =
        F_0(F^{n-1}(x)) - F_0(F^{n-1}(y)) = F_0(F^{n-1}(x) -~F^{n-1}(y)) =
        F_0^n(x-y)$
    \end{itemize}
    $$
    (F_0(x - y))(t) = \int\limits_0^t K(s_1, t)(x(s_1) - y(s_1))\dif s_1
    $$
    $$
    (F_0^2(x-y))(t) = \int\limits_0^t K(s_2, t)
    \int\limits_0^{s_2} K(s_1, s_2)(x(s_1) - y(s_1))\dif s_1\dif s_2
    $$
    $$
    \cdots
    $$
    $$
    (F_0^n(x-y))(t) = \int\limits_0^t K(s_n, t)
    \int\limits_0^{s_n} K(s_{n-1}, s_n)\int\limits_0^{s_{n-1}}\ldots
    \int\limits_0^{s_2} K(s_1, s_2)(x(s_1) - y(s_1))\dif s_1\dif s_2\ldots
    \dif s_n
    $$
    Получаем:
    $$
    \|F_0^n(x-y)\| = \max_{t \in [0, 1]} |(F_0^n(x-y))(t)|
    \leqslant M^n\|x-y\|\max_{t \in [0, 1]} \int\limits_0^t\int\limits_0^{s_n}
    \int\limits_0^{s_{n-1}}\ldots\int\limits_0^{s_3}\int\limits_0^{s_2}
    \dif s_1\dif s_2 \ldots \dif s_n \leqslant \frac{M^n}{n!}\|x-y\|
    $$
    Здесь $M = \max |K|$. Коэффициент $\frac{M^n}{n!}$ стремится к нулю, а
    это значит, что $F_0^n$ — сжимающее, следовательно, существует неподвижная
    точка.
\end{proof}

\begin{example}
    Допустим, что мы хотим решить дифференциальное уравнение
    $y'(t) = a(t)y(t) + b(t)$, $y(0) = y_0$, $a,b \in C[0, 1]$ на промежутке
    $[0, 1]$. Это
    уравнение имеет единственное решение $y \in C^1[0, 1]$. Как это доказать?
    Рассмотрим интегральное уравнение:
    $$
    x(t) = \int\limits_0^t a(s)x(s)\dif s + B(t)
    $$
    По предыдущей теореме существует $x \in C[0,1]$, решающее это
    уравнение. Для этого уравнения также верны утверждения:
    \begin{itemize}
        \item $x'(t) = a(t)x(t) + b(t)$, где $b(t) = B'(t)$;
        \item $x(0) = B(0)$.
    \end{itemize}
    Для решения исходной задачи достаточно выбрать $B$ такое, что $B' = b$ и
    $B(0) = y_0$. Откуда взять непрерывную дифференцируемость $y$?
    $$
    b\in C[0,1] \implies B\in C^1[0,1],
    $$
    $$
    \quad x\in C[0,1],\, a\in C[0,1] \implies
    \int\limits_0^t x(s)a(s)\dif s \in C^1[0,1]
    $$
    Таким образом всё доказано.
\end{example}

\subsection{Линейные операторы}
\begin{definition}
    Пусть $X$, $Y$ — линейные нормированные пространства над одним полем скаляров.
    Отображение $U: X \to Y$ называется линейным, если:
    \begin{enumerate}
        \item $U(x_1+x_2) = U(x_1) + U(x_2)$ $\forall x_1, x_2 \in X$
        \item $U(\lambda x) = \lambda U(x)$, где $\lambda$ — скаляр, $x \in X$
    \end{enumerate}
\end{definition}
\begin{remark}
    Ясно, что выполнение обоих этих свойств равносильно
    $U(\lambda_1x_1 + \lambda_2x_2) = \lambda_1U(x_1) + \lambda_2U(x_2)$.
\end{remark}
\begin{remark}
    В дальнейшем будем обозначать $U(x)$ как $Ux$.
\end{remark}
\begin{proposition}[Свойства линейных отображений]
\mbox{}
    \begin{enumerate}
        \item $U(0) = 0$;
        \item $U\bigg(\sum\limits_{j=1}^n \lambda_j x_j\bigg) = \sum\limits_{j=1}^n
        \lambda_j Ux_j$;
        \item\label{image_linearity} Если $M \subset X$ — линейное множество,
        то множество
        $U(M)$ линейно в $Y$. Если $M \subset X$ — выпуклое множество,
        то множество $U(M)$ выпукло в $Y$;
        \item\label{preimage_linearity} Если $N \in Y$ — линейное (выпуклое),
        то $U^{-1}(N)$ — линейное (выпуклое). Частный случай: если $N = \{0\}$, то множество
        $U^{-1}(N) = U^{-1}(\{0\}) = \Ker U$ — линейное в $X$;
        \item $\Ker U = \{0\} \iff U$ инъективно;
        \item\label{inverse_linearity} Если $U$ — линейная биекция, то
        $U^{-1}$ — линейное;
        \item Пусть $U_1, U_2: X \to Y$ — линейные. Тогда $U_1 + U_2$,
        $\lambda U_1$ тоже линейны;
        \item Если $X \overto{U} Y \overto{V} Z$, то композиция $V\circ U$
        линейна.
    \end{enumerate}
\end{proposition}
\begin{definition}
    Множество $M$ называется выпуклым, если для любых $x_1,x_2 \in M$ отрезок
    $[x_1, x_2]$ лежит в $M$.
\end{definition}
\begin{proof}[Доказательство предложения]
    Докажем выпуклость в свойстве \ref{image_linearity}.
    $$
    y_1, y_2 \in U(M) \implies \exists x_1, x_2 \in M:\, Ux_1 = y_1,\,
    Ux_2 = y_2
    $$
    $$
    \lambda y_1 + (1-\lambda)y_2 = \lambda Ux_1 + (1 - \lambda)Ux_2 =
    U(\underbrace{\lambda x_1 + (1-\lambda)x_2}_{\in M}) \in U(M)
    $$
    
    В свойстве \ref{preimage_linearity}:
    $$
    x_1, x_2 \in U^{-1}(N) \implies Ux_1, Ux_2 \in N \implies
    \forall \lambda_1, \lambda_2\quad \lambda_1 Ux_1 + \lambda_2 Ux_2 \in N
    \implies
    $$
    $$
    \implies U(\lambda_1 x_1 + \lambda_2 x_2) \in N \implies
    \lambda_1x_1 + \lambda_2x_2 \in U^{-1}(N)
    $$
    
    В свойстве \ref{inverse_linearity} биективность $U$ означает, что
    $\forall y_1, y_2$ $\exists x_1, x_2$ такие, что $Ux_1 = y_1$,
    $Ux_2 = y_2$. Отсюда $U^{-1}(y_1+y_2) = U^{-1}(Ux_1 + Ux_2) =
    U^{-1}(U(x_1 + x_2)) = x_1 + x_2 = U^{-1}(x_1) + U^{-1}(x_2)$.
    
    Доказательства остальных свойств тривиальны.
\end{proof}

\begin{theorem}[Эквивалентные условия непрерывности линейного отображения]
    Пусть $U:~X \to Y$ — линейный оператор. Тогда следующие утверждения
    эквивалентны:
    \begin{enumerate}
        \item $U$ непрерывен;
        \item $U$ непрерывен в нуле;
        \item Образ любого ограниченного множества ограничен;
        \item Существует $C$ такое, что $\forall x\in X$
        выполняется $\|Ux\|_Y = C\|x\|_X$.
    \end{enumerate}
\end{theorem}
\begin{proof}
\mbox{}
    \begin{itemize}
        \item $1 \Rightarrow 2$. Tривиально.
        \item $4 \Rightarrow 1$. $\|Ux_1 - Ux_2\| \leqslant
        C\|x_1 - x_2\|$. Это влечёт липшицевость и, как следствие, непрерывность.
        \item $2 \Rightarrow 3$. Непрерывность в нуле означает, что
        $\forall \varepsilon > 0$ $\exists \delta > 0$ такое, что
        $\|x\|<\delta \implies \|Ux\|<\varepsilon$. Ограниченность множества $M$ в $X$
        означает, что $\exists R:$ $M \subset B_R(0)=\{\|x\|\leqslant R\}$.
        Таким образом, $x\in M \implies \|x\| \leqslant R$.
        $\|\frac{\delta}{2R}x\| \leqslant \frac{\delta}{2} < \delta \implies
        \|U(\frac{\delta}{2R}x)\| < \varepsilon$. Отсюда
        $\|Ux| \leqslant \frac{\varepsilon\cdot 2R}{\delta} \implies Ux
        \in B_{\frac{\varepsilon\cdot 2R}{\delta}}(0)$. То есть, $U(M)$ ограничено.
        \item $3 \Rightarrow 4$. $B_1(0)$ — ограниченное множество. Тогда
        $U(B_1(0))$ — ограничено, т. е. существует такое $C$, что
        $U(B_1(0)) \subset B_C(0)$. Если $\|x\|\leqslant 1$, то $\|Ux\| \leqslant C$.
        Теперь возьмём произвольное $x$. $x' = \frac{x}{\|x\|} \in B_1(0) \implies
        \|Ux'\| \leqslant C$. Но $\|Ux'\| = \|U\big(\frac{x}{\|x\|}\big)\| =
        \frac{1}{\|x\|} \cdot \|Ux\|$. Отсюда $\|Ux\|\leqslant C\|x\|$.
    \end{itemize} 
\end{proof}

\begin{definition}
    Пусть $U: X \to Y$ — линейный непрерывный оператор. Тогда нормой оператора $U$
    называется величина $\|U\| = \inf\,\{C\,\big|\,\|Ux\| \leqslant C \|x\|\}$.
\end{definition}
\begin{remark}
    В формулировке определения инфимум и минимум совпадают (это можно доказать, перейдя к 
    пределу в неравенстве $\|Ux\| \leqslant C \|x\|$).
\end{remark}
\begin{remark}
    Выполнено неравенство $\|Ux\|_Y \leqslant \|U\|\cdot \|x\|_X$. В частности,
    $\frac{\|Ux\|_Y}{\|x\|_X} \leqslant \|U\|$ $\forall x \in X$, т. е. можно записать
    $\|U\| = \sup\limits_{x \neq 0}\frac{\|Ux\|}{\|x\|}$. 
\end{remark}
\begin{theorem}[Об эквивалентных способах определения нормы оператора]
     Пусть $U: X \to Y$ — линейный непрерывный оператор. Тогда:
     $$
     \|U\| = \underbrace{\sup\limits_{x \neq 0}\frac{\|Ux\|}{\|x\|}}_A =
     \underbrace{\sup\limits_{\|x\|\leqslant 1} \|Ux\|}_B =
     \underbrace{\sup\limits_{\|x\| < 1} \|Ux\|}_C =
     \underbrace{\sup\limits_{\|x\| = 1} \|Ux\|}_D
     $$
\end{theorem}
\begin{remark}
    Так как замкнутость и ограниченность, вообще говоря, неравносильна компактности
    (за исключением конечномерных пространств),
    в $\sup\limits_{\|x\|\leqslant 1} \|Ux\|$ максимум может и не достигаться.
\end{remark}
\begin{proof}[Доказательство теоремы]
    Очевидно, что $B \geqslant C$ и $B \geqslant D$.
    $$
    B =\sup\limits_{\|x\|\leqslant 1,\,x \neq 0} \|Ux\| \leqslant
    \sup\limits_{\|x\|\leqslant 1,\,x \neq 0} \frac{\|Ux\|}{\|x\|} \leqslant
    \sup\limits_{x \neq 0} \frac{\|Ux\|}{\|x\|} = A
    $$
    Докажем, что $D \geqslant A$. Возьмём $x'=\frac{x}{\|x\|}$, тогда $\|x'\|=1$ и
    $\|Ux'\|\leqslant D$. $\|U(\frac{x}{\|x\|})\| = \frac{\|Ux\|}{\|x\|}$.
    Итак, $\frac{\|Ux\|}{\|x\|} \leqslant D$, тогда и
    $\sup\limits_{x \neq 0}\frac{\|Ux\|}{\|x\|}$.
    Осталось проверить, что $C \geqslant A$. Возьмём $x \neq 0$, $\varepsilon > 0$.
    Рассмотрим $x' = \frac{x}{\|x\|(1 + \varepsilon)}$. Тогда $\|x\| < 1$. Отсюда
    следует, что $\|Ux'\| \leqslant C \implies \frac{\|Ux\|}{\|x\|(1+\varepsilon)}
    \leqslant C \implies \frac{\|Ux\|}{\|x\|}\leqslant C(1 + \varepsilon)$. Устремив
    $\varepsilon \to 0$, получим
    $A = \sup\limits_{x \neq 0} \frac{\|Ux\|}{\|x\|} \leqslant C$.
\end{proof}

\subsection{Пространства линейных непрерывных операторов}
\begin{definition}
    Пусть $X$, $Y$ — линейные нормированные пространства над одним полем скаляров. Возьмём
    $B(X,Y) = \{U: X \to Y,\, U \text{ — линейно, непрерывно}\}$.
    Это \emph{линейное пространство}.
\end{definition}
\begin{theorem}[О свойствах операторной нормы]
    $U,V \in B(X, Y)$.
    \begin{enumerate}
        \item $\|U\| \geqslant 0$, $\|U\| = 0 \iff U = 0$;
        \item $\|\lambda U\| = |\lambda|\|U\|$ ($\lambda$ — скаляр);
        \item $\|U + V\| \leqslant \|U\| + \|V\|$;
        \item $W \in B(Y,Z)$. $WU \in B(X, Z)$, $\|WU\| \leqslant \|W\|\|U\|$.
    \end{enumerate}
\end{theorem}
\begin{proof}
\mbox{}
    \begin{enumerate}
        \item Неотрицательность очевидна. Если $\|U\|=0$, то $\|Ux\|\leqslant
        0\cdot \|x\| \implies \|Ux\| = 0$ $\forall x$;
        \item $\|\lambda U\| = \sup\limits_{\|x\|=1} \|(\lambda U)(x)\| =
         \sup\limits_{\|x\|=1} |\lambda|\|Ux\| = |\lambda| \sup\limits_{\|x\|=1}
         \|U_x\| =
         |\lambda|\|U\|$;
         \item $x \in X$. $\|(U+V)(x)\| = \|Ux + Vx\| \leqslant \|Ux\| + \|Vx\|
         \leqslant \|U\|\|x\| + \|V\|\|x\| = (\|U\| + \|V\|)\|x\|$
         \item $x \in X$. $\|(WU)(x)\| = \|W(U(x))\| \leqslant \|W\|\cdot \|Ux\|
         \leqslant \|W\|\|U\|\|x\|$.
    \end{enumerate}
\end{proof}
\begin{theorem}[О полноте пространства операторов]
    Если $Y$ полно, то $B(X, Y)$ полно.
\end{theorem}
\begin{proof}
    Возьмём фундаментальную последовательность линейных непрерывных отображений
    $U_n \in B(X, Y)$, то есть $\|U_n - U_m\| \underto{m,n \to \infty} 0$:
    $\forall \varepsilon>0$ $\exists N:$ $\forall m,n > N$ $\|U_n - U_m\| < \varepsilon$.
    Это означает, что $\|(U_n - U_m)(x)\| \leqslant \varepsilon \|x\|$. Следовательно,
    $\{U_nx\}$ фундаментальна в $Y$. Обозначим $Ux = \lim\limits_{n \to \infty}
    U_nx$. Мы хотим проверить, что $U$ непрерывно, линейно и что есть сходимость по норме.
    \begin{enumerate}
        \item (Линейность $U$). $U(\alpha_1x_1 + \alpha_2x_2) = \lim\limits_{n \to \infty}
        U_n(\alpha_1x_1 + \alpha_2x_2) = 
        \alpha_1 \lim U_nx_1 + \alpha_2\lim U_nx_2 = \alpha_1 Ux_1 +
        \alpha_2 Ux_2$
        \item (Нерерывность $U$). Возьмём любое $\varepsilon > 0$, $N$, $\forall m,n>N$,
        $\forall x\in X$. $\|U_nx - U_mx\| \leqslant \varepsilon \|x\| \implies
        \|Ux - U_mx\| \leqslant \varepsilon\|x\|$.
        $\|Ux\| = \|(Ux - U_mx) + U_mx\| \leqslant \|(Ux - U_mx)\| + \|U_mx\| \leqslant
        \varepsilon \|x\| + \|U_m\|\|x\|$. Отсюда $\|U\| \leqslant \varepsilon +
        \|U_m\|$.
        \item (Сходимость $U_n$ к $U$).
        $\forall \varepsilon > 0$ $\exists N$: $\forall m,n>N$ $\forall x\in X$ $
        \|U_nx - U_mx\| \leqslant \varepsilon \|x\|$. Устремив $n$ к бесконечности, 
        получим: $\forall \varepsilon > 0$ $\exists N$: $\forall m > N$
        $\forall x\in X$ $\|Ux - U_mx\| = \|(U - U_m)(x)\| \leqslant \varepsilon \|x\|$
        $\implies \|U - U_m\| \leqslant \varepsilon$. Итак,
        $\forall \varepsilon > 0$ $\exists N$: $\forall m > N$
        $\|U - U_m\| \leqslant \varepsilon$, т. е. $U_n \to U$ в $B(X, Y)$.
    \end{enumerate}
\end{proof}

Следует отметить важный частный случай.
\begin{definition}
    $B (X, \text{поле скаляров}) = X^\ast$ называется
    \emph{сопряжённым пространством к $X$}. $f \in X^\ast$ называется \emph{линейным
    непрерывным функционалом}.
\end{definition}

Норма функционала определяется как $\|f\| = \inf\,\{C\,\big|\,|f(x)|
\leqslant C\|x\|\} =
\sup\limits_{x \neq 0}\frac{|f(x)|}{\|x\|} = \sup\limits_{\|x\|=1} |f(x)|$.

\subsection{Корректно разрешимые задачи}

Рассмотрим отображение $A: X \to Y$. Мы хотим решить уравнение $Ax = f$. $f$ —
какие-то известные данные.

В общей постановке вопроса корректная разрешимость означает три вещи:
\begin{itemize}
    \item Решение существует для любого $f$.
    \item Решение единственно.
    \item Устойчивость: если $f_n \to f$, то для решений верно, что $x_n \to x$. 
    (Здесь $Ax_n = f_n$, $Ax = f$.)
\end{itemize}

В частном случае, когда $X$ и $Y$ — линейные нормированные пространства и
$A$ — линейное отображение, вышеописанные условия равносильны тому, что
$A^{-1} \in B(Y, X)$.

\begin{remark}
    Самый простой пример корректно разрешимой задачи — случай, когда оператор
    $A$ тождественен.
\end{remark}

\begin{theorem}[Об обратимости оператора, близкого к тождественному]
    Если $B \in B(X, X)$, $X$ — полное и  $\|B\| < 1$, то существует оператор
    $(I \pm B)^{-1} \in B(X, X)$. ($I$ — тождественный оператор.)
\end{theorem}
\begin{proof}
    Приведём два способа доказать эту теорему.
    \begin{enumerate}
        \item Возьмём уравнение $(I-B)x = f$. Надо доказать, что для любого
        $f \in X$ существует единственный $x \in X$, решающий это уравнение.
        Это равносильно $x = f + Bx = g(x)$. Заметим, что $x$ удовлетворяет
        уравнению тогда и только тогда, когда $x$ — неподвижная точка отображения
        $g$. Проверим, что $g$ — сжимающее.
        $\|g(x_1) - g(x_2)\| = \|(f + Bx_1) - (f + Bx_2)\| =
        \|Bx_1 - Bx_2\| \leqslant \underbrace{\|B\|}_{< 1} \cdot \|x_1 - x_2\|$.
        
        Теперь проверим устойчивость. Пусть $f_n \to f$, $(I-B)x_n = f_n$,
        $(I-B)x=f$. Нужно проверить, что $x_n \to x$. $x_n = f_n + Bx_n$,
        $x = f + Bx$.
        $$
        \|x_n - x\| = \|f_n + Bx_n - f - Bx\| \leqslant \|f_n - f\| +
        \|Bx_n - Bx\| \leqslant \|f_n - f\| + \|B\|\cdot \|x_n - x\|
        $$
        Отсюда
        $$
        0 \leqslant \underbrace{(1 - \|B\|)}_{>0}\|x_n - x\|\leqslant
        \underbrace{\|f_n - f\|}_{\to 0} \implies
        \|x_n - x\| \to 0
        $$
        \item Докажем формулу $(I - B)^{-1} = I + B + B^2 + B^2 + \ldots$.
        Необходимо проверить, что этот ряд сходится. Докажем, что он
        сходится абсолютно, то есть $\|I\| + \|B\| + \|B^2\| + \ldots < \infty$.
        Заметим, что $\|B^k\| \leqslant \|B\|^k$. Отсюда
        $\|I\| + \|B\| + \|B^2\| + \ldots \leqslant
        \|I\| + \|B\| + \|B\|^2 + \ldots$. Но это — геометрическая прогрессия,
        она сходится.
        Частичные суммы: $S_n = I + B + \ldots B^{n-1}$,
        $(I - B)S_n = S_n(I-B) = I - B^n \underto{n \to \infty} I$.
        Мы воспользовались полнотой пространства, утверждая, что абсолютная
        сходимость влечёт сходимость ряда.
    \end{enumerate}
\end{proof}

\begin{theorem}[Об обратимости оператора, близкого к обратимому]
    Пусть $U \in B(X, Y)$ — линейное отображение и существует $U^{-1}\in B(Y, X)$.
    Кроме того, $X$ \textbf{или} $Y$ — полное пространство.
    Рассмотрим $V\in B(X, Y)$ такой, что $\|V\| < \|U^{-1}\|^{-1}$. Тогда
    существует $(U + V)^{-1} \in B(Y, X)$.
\end{theorem}
\begin{proof}
    $U + V = U(I_X + U^{-1}V)$ (или $(I_Y + VU^{-1})U$).
    Оператор $U$ обратим, обратный к нему оператор
    непрерывен. Получаем $\|U^{-1}V\|\leqslant\|U^{-1}\|\cdot\|V\| < 1$.
\end{proof}

\subsection{Линейные непрерывные функционалы}

Вспомним, что если $X$ — нормированное пространство, то
$X^\ast = B(X, \text{поле скаляров})$ называется сопряжённым к $X$ пространством.
Норма функционала определяется как $\|f\| = \inf\,\{C\,\big|\,|f(x)|
\leqslant C\|x\|\} =
\sup\limits_{x \neq 0}\frac{|f(x)|}{\|x\|} = \sup\limits_{\|x\|=1} |f(x)|$.

\begin{example}[Функционалы в пространстве Лебега]
        Рассмотрим $L^p(T, \mu)$, причём
        $1 < p < \infty$. Возьмём $q$ — сопряжённый показатель такой, что
        $\frac{1}{q} + \frac{1}{p} = 1$. Возьмём также $y_0 = L^q(T, \mu)$.
        Определим функционал $f$ формулой
        $f(x) = \int\limits_T x(t)y_0(t)\dif \mu(t)$. Нам нужно проверить, что
        это действительно функционал, что он непрерывен (линейность очевидна).
        Чтобы этот функционал был функционалом, необходимо, чтобы
        подынтегральная функция была суммируемой. Для этого воспользуемся
        неравенством Гёльдера:
        $$
        \int\limits_T |x(t)y_0(t)| \dif \mu(t) \leqslant
        \bigg(\int\limits_T |x|^p\bigg)^\frac{1}{p}
        \bigg(\int\limits_T |y_0|^q\bigg)^\frac{1}{q} = \|y_0\|_q\cdot \|x\|_p <
        \infty
        $$
        $$
        |f(x)| \leqslant \underbrace{\|y_0\|_q}_{=C} \cdot \|x\|
        \implies \|f\| \leqslant \|y_0\|_q
        $$
        
        Проверим, что $\|f\| \geqslant \|y_0\|_q$.
        $$
        x_0(t) = \frac{|y_0|^q}{y_0} = |y_0|^{q-1} \frac{|y_0|}{y_0} =
        |y_0|^{q-1}\sign y_0
        \implies x_0y_0 = |y_0|^q
        $$
        $$
        |f(x_0)| = \bigg|\int\limits_T x_0y_0\bigg| = \int\limits_T|y_0|^q
        $$
        Но так как $\frac{1}{p} + \frac{1}{q} = 1$, то $(q-1)p = q$.
        $$
        \|x_0\|_p = \bigg(\int\limits_T |x_0|^p\bigg)^\frac{1}{p} =
        \bigg(\int\limits_T |y_0|^{(q-1)p}\bigg)^\frac{1}{p} =
        \bigg(\int\limits_T |y_0|^q\bigg)^\frac{1}{p}
        $$
        $$
        \|f\|\geqslant \frac{|f(x_0)|}{\|x_0\|_p} =
        \frac{\int\limits_T |y_0|^q}{\bigg(\int\limits_T|y_0|^q\bigg)^\frac{1}{p}} =
        \bigg(\int\limits_T |y_0|^q)\bigg)^\frac{1}{q} = \|y_0\|_q
        $$
        Таким образом, $L^q(T, \mu) \hookrightarrow L^p(T, \mu)^\ast$,
        $y_0 \mapsto f$ и $\|y_0\|_q = \|f\|$. Имеет место
        \emph{изометрическое вложение}, и даже более того, биекция.
\end{example}
\begin{example}
    Рассмотрим пространство $C[-1, 1]$. Пусть $f(x) = \int\limits_{-1}^1
    tx(t)\dif t$. Снова хотим доказать, что это функционал, что он непрерывен и 
    линеен.
    Для непрерывности достаточно установить, что $|f(x)| \equiv C\|x\|$.
    $$
    |f(x)| \leqslant \int\limits_{-1}^1 |t||x(t)|\dif t \leqslant
    \max |x| \int\limits_{-1}^1 |t|\dif t = \|x\| \implies \|f\| \leqslant 1
    $$
    Непрерывность доказана.
    Теперь возьмём функцию $x_\varepsilon(t) =
    \begin{cases}
        1,\quad t \geqslant \varepsilon \\
        \frac{t}{\varepsilon},\quad |t|\leqslant \varepsilon \\
        -1,\quad t \leqslant -\varepsilon\\
    \end{cases}$.
    $$
    f(x_\varepsilon) = \int\limits_{-1}^1 tx_\varepsilon(t)\dif t =
    \bigg(\int\limits_{-1}^{-\varepsilon} + \int\limits_\varepsilon^1\bigg)
    |t|\dif t + \int\limits_{-\varepsilon}^\varepsilon \frac{t^2}{\varepsilon}
    \dif t = 1 + O(\varepsilon)
    $$
    Получаем, что $\|f\|\geqslant \frac{f(x_\varepsilon)}{\|x_\varepsilon\|}
    \underto{\varepsilon \to 0} 1$.
    Теперь возьмём
    $y_0 \in L^1(-1, 1)$, $f(x) = \int\limits_{-1}^1 y_0(t)x(t)\dif t$.
    $$
    |f(x)| \leqslant \int\limits_{-1}^1 |y_0||x| \leqslant \|x\|\int\limits_{-1}^1
    |y_0| \leqslant \|y_0\|_1\cdot \|x\|_C
    $$
    Значит, $f$ — линейный непрерывный функционал. $\|f\|=\|y_0\|_1$,
    $x_0(t) = \sign y_0 \notin C$.
\end{example}
\begin{exercise}
    Пусть $\delta(x) = x(0)$. Доказать, что $\delta \notin L^1(-1, 1)$, то есть
    не существует $y_0 \in L^1(-1, 1)$ такого, что $\forall x\in C[-1, 1]$
    $\int\limits_{-1}^1 y_0(t)x(t)\dif t = x(0)$
\end{exercise}

Напомним, что $\ell^\infty = \{x = (x_1, x_2, \ldots),\, \|x\|_\infty = 
\sup\limits_{j \geqslant 1} |x_j| < \infty\}$ и
$c_0 = \{x = (x_1, x_2, \ldots),\, \lim\limits_{j \to \infty} x_j =~0\}$,
$c_0 \subset \ell^\infty$. При этом $\|x\|_{c_0} = \|x\|_\infty$. $c_0$ — полное
нормированное пространство.

\begin{theorem}
    $(c_0)^\ast = \ell^1$
\end{theorem}

Рассмотрим $L_\text{fin} \subset \ell^\infty$ такое, что $x \in L_\text{fin}$,
если у $x$ лишь конечное число ненулевых координат. Отметим, что $L_\text{fin}$
является линейной оболочкой векторов $e_1, e_2, \ldots$, где
$e_k = (0, 0, \ldots, 0,\underbrace{1}_{k}, 0, \ldots)$. Также
$\overline{L_\text{fin}} = c_0$
\begin{itemize}
    \item $x \in c_0 \implies \exists x^{(n)} \in L_\text{fin}$: $x^{(n)} \to 
    \infty$, где $x^{(n)} = (x_1, x_2, \ldots, x_n, 0, 0, \ldots)$.
    $\|x - x^{(n)}\| = \|(0, 0, \ldots, 0, x_{n+1}, x_{n+2}, \ldots)\|_\infty =
    \sup\limits_{j \geqslant n+1}|x_j|$.
    \item $c_0$ замкнуто.
\end{itemize}

\begin{proof}
\mbox{}
    \begin{enumerate}
        \item Возьмём $y^{(0)} \in \ell^1$, где
        $y^{(0)} = (y_1^{(0)}, y_2^{(0)}, \ldots)$
        и $\|y^{(0)}\|_1 = \sum\limits_{j=1}^\infty |y_j^{(0)}| < \infty$.
        Построим по нему функционал на $c_0$. Пусть $x \in c_0$. Рассмотрим
        $f(x) = \sum\limits_{j=1}^\infty x_jy_j^{(0)}$.
        $$
        |f(x)| \leqslant\sum\limits_{j=1}^\infty \underbrace{|x_j|}_{\leqslant \|x\|_\infty}
        |y_j^{(0)}| \leqslant \|x\|_\infty \sum\limits_{j=1}^\infty |y_j^{(0)}| =
        \|y^{(0)}\|_1 \|x\|_\infty \implies
        \|f\| \leqslant \|y^{(0)}\|_1
        $$
        
        Мы построили вложение $\ell^1 \hookrightarrow (c_0)^\ast$, $y^{(0)} 
        \mapsto f$.
        \item Пусть нам дан функционал $f \in (c_0)^\ast$. Мы хотим построить по
        нему $y \in \ell^1$. Положим $f(e_j) = y_j$ ($y = (y_1, y_2, \ldots)$).
        Нам нужно проверить, что $y \in \ell^1$ и что
        $\forall x$ $f(x) = \sum x_jy_j$.
        Возьмём $z^{(n)} =
        (\sign y_1, \sign y_2, \ldots, \sign y_n, 0, 0,\ldots)$.
        $|f(z^{(n)}| \leqslant \|f\|\cdot \|z^{(n)}\|_\infty \leqslant \|f\|$.
        Но левая часть неравенства равна $\sum\limits_{j=1}^\infty |y_j|$. Из
        неравенства следует, что ряд сходится, отсюда $y \in \ell^1$.
        
        Покажем теперь, что $\forall x$ $f(x) = \sum x_jy_j$.
        пусть $x = (x_1, x_2, \ldots) = \sum\limits_{j=1}^\infty x_je_j$.
        $$
        f\bigg(\sum\limits_{j=1}^n x_je_j\bigg) = \sum\limits_{j=1}^n x_jf(e_j) =
        \sum\limits_{j=1}^n x_jy_j \underto{n \to \infty} \sum\limits_{j=1}^\infty
        x_jy_j
        $$
        Левая часть стремится к $f(x)$, так как $\sum\limits_{j=1}^n = x_je_j
        \underto{n \to \infty} x$.
    \end{enumerate}
\end{proof}

\subsection{Интегральные операторы}

Что такое интегральный оператор? Допустим, у нас есть функция двух переменных
$K(s, t)$, называемая \emph{ядром интегрального оператора} (не путать с ядром
оператора). Оператор действует следующим образом: он берёт функцию $x(s)$ и
преобразует её в функцию $(Ux)(t)$ по формуле
$(Ux)(t) = \int K(s, t) x(s)$ (множество интегрирования и мера определяются 
отдельно). Какими свойствами должна обладать функция $K$,
чтобы этот оператор был <<хорошим>>?

\lparagraph{Интегральные операторы в пространствах Лебега}

Будем рассматривать переменные $s$ на множестве $S$ с мерой $\nu$ и $t$
на множестве $T$ с мерой $\mu$, а также функцию
$K: S \times T \to \text{поле скаляров}$, притом измеримую. Пусть $x$ — также
измеримая функция на $S$, $(Ux)(t) = \int\limits_S K(s, t)x(s)\dif \nu(s)$.
Какие условия нужно наложить на функцию $K$, чтобы оператор $U$ действовал из
$L^p(s, \nu)$ в $L^r(T, \mu)$ и был непрерывен?

$$
\int\limits_T |(Ux)(t)|^r \leqslant \int\limits_T
\bigg(\int\limits_S|K(s, t)||x(s)|\dif s\bigg)^r \dif t \leqslant
\int\limits_T\bigg(\bigg(\int\limits_S |K(s, t)|^q\dif s\bigg)^\frac{1}{q}\|x\|_p\bigg)^r\dif t =
$$
$$
= \int\limits_T\bigg(\int\limits_S|K(s, t)|^q\dif s\bigg)^\frac{r}{q}\dif t \cdot \|x\|_p^r
$$
$$
\|Ux\|_r \leqslant \bigg(\int\limits_T\bigg(\int\limits_S|K(s, t)|^q\dif s\bigg)^\frac{r}{q}\dif t
\bigg)^\frac{1}{r} \cdot \|x\|_p
$$
Здесь мы воспользовались неравенством Гёльдера и $\frac{1}{q} + \frac{1}{p} = 1$.

Таким образом, мы доказали следующую теорему.

\begin{theorem}[О гёльдеровских условиях непрерывности]
    Если $\int\limits_T\bigg(\int\limits_S |K(s, t)|^q \dif s\bigg)^\frac{r}{q} \dif t
    < \infty$, то $U$ действует непрерывно из $L^p(s, \nu)$ в $L^r(T, \mu)$.
\end{theorem}

Пусть $p = 2$, $r = 2$, то есть $q = 2$. Тогда:
$$
\int\limits_T \int\limits_S |K(s, t)|^2 \dif s \dif t < \infty \iff
K \in L^2(S\times T, \nu \times \mu)
$$
и $\|U\|\leqslant \|K\|_{L^2(S\times T, \nu\times \mu)}$. Операторы, 
удовлетворяющие таким условиям, называются операторами Гильберта-Шмидта, а
$K$ — ядром Гильберта-Шмидта.
\begin{remark}
    Существуют линейные непрерывные интегральные операторы, не являющиеся
    операторами Гильберта-Шмидта.
\end{remark}

\lparagraph{Тест Шура}

\begin{theorem}[Тест Шура]
    Пусть $(Ux)(t) = \int\limits_S K(s, t)x(s)\dif \nu(s)$.
    Предположим, что существуют строго положительные функции $\varphi: S \to 
    \real$, $\psi: T \to \real$ и числа $A, B \in \real$ такие, что:
    \begin{enumerate}
        \item $\int\limits_S |K(s, t)|\varphi(s)\dif \nu(s) \leqslant A\psi(t)$ 
        для почти всех $t \in T$.
        \item $\int\limits_T |K(s, t)|\psi(t)\dif \mu(t) \leqslant B\varphi(s)$ 
        для почти всех $s \in S$.
    \end{enumerate}
    Тогда $U$ — линейный непрерывный оператор из $L^2(S, \nu)$ в $L^2(T, \mu)$.
\end{theorem}
\begin{proof}
    $$
    |(Ux)(t)| \leqslant \int\limits_S \sqrt{|K(s, t)|\varphi(s)}
    \sqrt{\frac{|K(s,t)||x(s)|^2}{\varphi(s)}} \dif \nu(s) \leqslant
    \underbrace{\bigg(\int\limits_S |K(s,t)|\varphi(s) \dif s\bigg)^\frac{1}{2}
    }_{\leqslant A\psi(t)}\bigg(\int\limits_S \frac{|K(s,t)||x(s)|^2}{\varphi(s)}
    \dif s\bigg)^\frac{1}{2}
    $$
    $$
    \int\limits_T |(Ux)(t)|^2 \dif t \leqslant \int\limits_T A\psi(t)
    \int\limits_S \frac{|K(s,t)||x(s)|^2}{\varphi(s)} \dif s \dif t =
    \int\limits_S A \frac{|x(s)|^2}{\varphi(s)}
    \underbrace{\int\limits_T |K(s, t)|\psi(t)\dif t}_{\leqslant B\varphi(s)} \dif s <
    AB \int\limits_S |x(s)|^2\dif s
    $$
    То есть, $\|Ux\|_2 \leqslant \sqrt{AB}\|x\|_2$.
\end{proof}

\begin{corollary}
    $\|U\| \leqslant \sqrt{AB}$
\end{corollary}

\begin{exercise}
\mbox{}
    \begin{enumerate}
        \item $S = T = (0, 1)$ с мерой Лебега, $K(s,t) = \frac{1}{\sqrt{|s-t|}}$.
        Заметим, что получается оператор, не являющийся
        оператором Гильберта-Шмидта, так как
        $\int\limits_0^1\int\limits_0^1 \frac{\dif s\dif t}{|s-t|} =~+\infty$.
        Придумать тест Шура для этого случая.
        \item $S = T = \real$, $K(s,t) = e^{-(s + t)^2}$. Является $U$ оператором
        Гильберта-Шмидта, и, если нет, является ли он непрерывным?
        \item $S = T = (0, +\infty)$, $K(s,t) = e^{-st}$. Установить непрерывность
        $U$ с помощью теста Шура.
        \item $S = T = \mb N$, $\nu = \mu = \#$,
        $K: \mb N \times \mb N \to \real$. Тогда оператор $U$ равен
        $\sum\limits_{j=1}^\infty K_{ij}x_j$.
    \end{enumerate}
\end{exercise}
\begin{theorem}[Тест Шура в дискретном случае]
    Пусть существуют $\varphi_j > 0$, $\psi_i > 0$, $A, B$ такие, что
    \begin{enumerate}
        \item $\sum |K_{ij}|\varphi_j \leqslant A\psi_i$ $\forall i \in \mb N$
        \item $\sum |K_{ij}|\psi_j \leqslant B\varphi_j$ $\forall j \in \mb N$
    \end{enumerate}
    Тогда $U: \ell^2 \to \ell^2$ непрерывен и $\|U\| \leqslant \sqrt{AB}$.
\end{theorem}

\begin{example}[Оператор Харди]
    Оператор Харди $H$ действует в пространстве $L^2(0, +\infty)$:
    $$
    (Hx)(t) = \frac{1}{t}\int\limits_0^t x(s) \dif s
    $$
    
    Частный случай: $H: \ell^2 \to \ell^2$ и
    $(Hx)_k = \frac{1}{k}(x_1 + \ldots + x_k)$ (среднее арифметическое).
    
    Применим тест Шура.
    $$
    \frac{1}{t}\int\limits_0^t x(s) \dif s = \int\limits_0^\infty K(s,t)x(s)\dif s
    $$
    где $K(s,t) = \frac{1}{t}\chi_{[0,t]}(s) = \frac{1}{t}\chi_{[s, +\infty)}(t)$.
    Возьмём $\varphi(s) \equiv 1$. Тогда
    $$
    \int\limits_0^\infty|K(s,t)|\varphi(s) \dif s =
    \frac{1}{t}\int\limits_0^t \dif s = 1
    $$
    Взяв $\psi(t) \equiv 1$, получим
    $$
    \int\limits_0^\infty|K(s,t)|\psi(t) \dif t = \int\limits_0^\infty \frac{\dif t}{t}
    = \infty
    $$
    Значит, такое $\psi$ не подходит. Возьмём $\psi(t) = t^{-\alpha}$, где
    $\alpha > 0$. Тогда
    $$
    \int\limits_0^\infty|K(s,t)|\psi(t) \dif t =
    \int\limits_0^s \frac{\dif t}{t^{\alpha + 1}} =
    \frac{s^{-\alpha}}{\alpha}
    $$
    В качестве $\varphi(s)$ возьмём $s^{-\alpha}$.
    $$
    \int\limits_0^\infty|K(s,t)|\varphi(s) \dif s =
    \frac{1}{t} \int\limits_0^t s^{-\alpha} \dif s =
    \frac{1}{t} \frac{t^{1- \alpha}}{1 - \alpha} =
    \frac{t^{-\alpha}}{1 - \alpha}
    $$
    Заметим, что при этом должно быть $\alpha < 1$. Кроме того,
    $$
    \|H\| \leqslant \frac{1}{\sqrt{\alpha(1 - \alpha)}}\quad
    \forall \alpha \in (0, 1) \implies \|H\| \leqslant 2
    $$
\end{example}
\begin{exercise}
    Доказать, что $\|H\| = 2$.
\end{exercise}

\lparagraph{Интегральные операторы с непрерывным ядром}

Будем рассматривать ограниченную область $\Omega \subset \real^m$, пространство
$L^2(\Omega)$ и пространство непрерывных функций $C(\overline\Omega)$. Пусть также
у нас есть функция
$K: \overline\Omega \times \overline\Omega \to \real(\complex)$,
$K \in C(\overline\Omega)$, $\|K\|_{C(\overline\Omega)} = M$.

\begin{theorem}
    Рассмотрим оператор $U$ такой, что
    $(Ux)(t) = \int\limits_\Omega K(s,t)x(s) \dif s$. Верно, что
    $U \in B(L^2(\Omega), C(\overline\Omega))$.
\end{theorem}
\begin{proof}
    Докажем, что если $x \in L^2(\Omega)$, то $Ux \in C(\overline\Omega)$.
    (Здесь непрерывность $x$ не гарантируется.)
    
    $$
    |Ux(t_1) - Ux(t_2)| =
    \bigg|\int\limits_\Omega K(s, t_1) - K(s, t_2)x(s) \dif s\bigg| \leqslant
    \bigg(\int\limits_\Omega |K(s, t_1) - K(s, t_2)|^2 \dif s\bigg)^\frac{1}{2}
    \|x\|_2
    $$
    По теореме Кантора $K$ равномерно непрерывно на
    $\overline\Omega \times \overline\Omega$, то есть:
    $$
    \forall \varepsilon > 0\quad\exists \delta > 0:\quad
    \underbrace{|(s_1, t_1) - (s_2, t_2)|}_{\sqrt{|s_1 - s_2|^2 + |t_1 - t_2|^2}}
    < \delta \implies
    |K(s_1, t_1) - K(s_2, t_2)| < \varepsilon
    $$
    
    Если $|t_1 - t_2| < \delta$, то $|K(s, t_1) - K(s, t_2)| < \varepsilon$,
    отсюда $|Ux(t_1) - Ux(t_2) < \varepsilon |\Omega|^\frac{1}{2}\cdot \|x\|_2$
    
    Теперь докажем, что $\|Ux\|_{C(\overline\Omega)}
    \leqslant C \|x\|_{L^2(\Omega)}$.
    $$
    \|Ux\|_{C(\overline\Omega)} = \max\limits_{t \in \overline \Omega}
    \bigg| \int\limits_\Omega K(s, t)x(s) \dif s\bigg| \leqslant
    \max\limits_{t \in \overline \Omega} \bigg(\int\limits_\Omega |K(s, t)|^2
    \dif s\bigg)^\frac{1}{2} \|x\|_2 \leqslant
    (M^2 \cdot |\Omega|)^\frac{1}{2}\|x\|_{L^2(\Omega)}
    $$
\end{proof}

Рассмотрим оператор вложения $j: C(\overline\Omega) \to L^2(\Omega)$,
$x \mapsto x$. Справедливо следствие:

\begin{corollary}
    \begin{enumerate}
        \item $jU \in B(L^2(\Omega), L^2(\Omega))$
        \item $Uj \in B(C(\overline\Omega), C(\overline\Omega))$
    \end{enumerate}
\end{corollary}
\begin{proof}
    Заметим, что $C(\overline\Omega) \subset L^2(\Omega)$.
    $$
    \bigg(\int\limits_\Omega |x(s)|^2 \dif t\bigg)^\frac{1}{2} \leqslant
    \bigg(\|x\|_{C(\overline\Omega)}^2 \cdot |\Omega|\bigg)^\frac{1}{2} =
    |\Omega|^\frac{1}{2} \cdot \|x\|_{C(\overline\Omega)}
    $$
    Получаем
    $$
    \|x\|_{L^2(\Omega)} \leqslant
    |\Omega|^\frac{1}{2} \cdot \|x\|_{C(\overline\Omega)}
    $$

    $$
    \|jx\|_{L^2(\Omega)} = \|x\|_{L^2(\Omega)} \leqslant
    C \cdot \|x\|_{C(\overline\Omega)}
    $$
    То есть $j$ непрерывен.
    $$
    C(\overline\Omega) \hookrightarrow L^2(\Omega) \overto{U} C(\overline\Omega) \
    \hookrightarrow L^2(\Omega)
    $$
\end{proof}

\lparagraph{Операторы со слабой особенностью}

Рассмотрим оператор $Ux(t) = \int\limits_\Omega K(s, t) x(s) \dif s$, причём $K$ —
ядро со слабой особенностью, а $\Omega \subset \real^m$ — ограниченная область.

\begin{definition}
    $K$ — ядро со слабой особенностью, если оно представляется в виде:
    $$
    K(s, t) = \frac{A(s, t)}{|s - t|^\alpha}
    $$
    Здесь $A \in C(\overline\Omega \times \overline\Omega)$, $\alpha < m$
\end{definition}
\begin{example}
    $\Omega = (0, 1)$, $K(s, t) = \frac{1}{\sqrt{|s - t|}}$
\end{example}
\begin{remark}
    Предположим, что $K(s, t) = \frac{a(s, t)}{|s - t|^\alpha}$, $\alpha < m$,
    $a$ — ограниченная функция, непрерывная вне диагонали множества
    $\overline\Omega \times \overline\Omega$, то есть в точках $(s, t)$ таких,
    что $s \neq t$. Тогда $K$ — ядро со слабой особенностью. Почему? Можно
    записать $K(s, t) = \frac{a(s, t)|s-t|^\delta}{|s-t|^{\alpha + \delta}}$, где
    $\alpha + \delta < m$. $A(s, t) = a(s, t)|s-t|^\delta$ непрерывно на
    $\overline\Omega \times \overline\Omega$
\end{remark}

Почему особенность <<слабая>>? Чтобы ответить на этот вопрос, сформулируем лемму.

\begin{lemma}
    Пусть у нас есть шар $B(0, \rho) \subset \real^m$. Тогда
    $\int\limits_{B(0, \rho)} \frac{\dif x}{|x|^\alpha}$ конечен тогда и только 
    тогда, когда $\alpha < m$.
\end{lemma}
\begin{proof}
    Вычислим этот интеграл.
    $$
    \int\limits_{B(0, \rho)} \frac{\dif x}{|x|^\alpha} =
    \int\limits_0^\rho \int\limits_{S_1(0)}
    r^{m-1}\frac{1}{r^\alpha}\dif \theta \dif r =
    |S_1| \int\limits_0^\rho r^{m - \alpha - 1} \dif r =
    |S_1| \frac{r^{m - \alpha}}{m - \alpha}\bigg|_0^\rho =
    |S_1|\frac{\rho^{m - \alpha}}{m - \alpha}
    $$
\end{proof}
\begin{theorem}
    Пусть $U$ — оператор со слабой особенностью:
    $Ux(t) = \int\limits_\Omega K(s,t) x(s) \dif s$, $\Omega \subset \real^m$.
    Тогда $U\in B(L^2(\Omega), L^2(\Omega))$.
\end{theorem}
\begin{proof}
    Применим тест Шура. Возьмём функцию $\varphi(s) \equiv 1$.
    $$
    \int\limits_\Omega|K(s, t)|\dif s =
    \int\limits_\Omega \frac{|A(s,t)|}{|s-t|^\alpha}\dif s \leqslant
    M\cdot \int\limits_\Omega \frac{1}{|s-t|^\alpha}\dif s \leqslant
    M\cdot \int\limits_{B_d(t)} \frac{\dif s}{|s - t|^\alpha} \leqslant
    M \cdot \int\limits_{B_d(0)} \frac{\dif z}{|z|^\alpha} \leqslant
    $$
    $$
    \leqslant M \cdot |S_1|\cdot \frac{d^{m - \alpha}}{m - \alpha}
    $$
    Здесь $A \in C(\overline\Omega \times \overline\Omega)$,
    $\|A\|_{C(\overline\Omega \times \overline\Omega)} = M$,
    $d = \diam \overline\Omega$
    
    Получаем, что $\psi(t) = 1$.
\end{proof}
\begin{theorem}
    В условиях предыдущей теоремы также верно
    $U \in B(C(\overline\Omega), C(\overline\Omega))$.
\end{theorem}
\begin{proof}
    \begin{enumerate}
        \item $\forall x \in C(\overline\Omega)$
        $Ux \in C(\overline\Omega)$
        \item $\|Ux\|_{C(\overline\Omega)} \leqslant C \|x\|_{C(\overline\Omega)}$
        
        Будем доказывать, что $\forall \varepsilon > 0$ $\exists \delta > 0$
        такое, что $\forall t_1, t_2 \in \overline\Omega$ такого, что
        $|t_1 - t_2| < \delta$, $|Ux(t_1) - Ux(t_2)| < \varepsilon$
        $$
        Ux(t_1) - Ux(t_2) = \int\limits_\Omega(K(s, t_1) - K(s, t_2))x(s)\dif s
        $$
        Возьмём $\rho$ такое, что $|t_1 - t_2| < \rho$.
        Разобьём область $\Omega$ на три части:
        $$
        \Omega = \underbrace{(\Omega \backslash (B_{\frac{\rho}{2}}(t_1)
        \cup B_{\frac{\rho}{2}}(t_2))}_{\Omega_{1,2}}
        \cup \underbrace{(B_{\frac{\rho}{2}}(t_1) \cap \Omega)}_{\Omega_1} \cup
        \underbrace{(B_{\frac{\rho}{2}}(t_2) \cap \Omega)}_{\Omega_2}
        $$
        $$
        \bigg|\int\limits_{\Omega_1}\ldots\bigg| \leqslant
        \int\limits_{\Omega_1} |K(s, t_1) - K(s, t_2)||x(s)|\dif s \leqslant
        \|x\|_{C(\overline\Omega)}\bigg(\int\limits_{\Omega_1}|K(s, t_1)|\dif s +
        \int\limits_{\Omega_1}|K(s, t_2)|\dif s\bigg) \leqslant
        $$
        $$
        \leqslant M \cdot \|x\|\bigg(\int\limits_{\Omega_1}
        \frac{\dif s}{|s - t_1|^\alpha} + \int\limits_{\Omega_1}
        \frac{\dif s}{|s - t_2|^\alpha}\bigg) \leqslant
        M \cdot \|x\|\cdot |S_1|
        \bigg(\frac{(\frac{\rho}{2})^{m - \alpha}}{m - \alpha} +
        \frac{(\frac{3\rho}{2})^{m - \alpha}}{m - \alpha}\bigg) < 
        \frac{\varepsilon}{2}
        $$
        Такая же оценка справедлива и для $\Big|\int\limits_{\Omega_2}\ldots\Big|$.
        $$
        \bigg|\int\limits_{\Omega_{1,2}}\ldots\bigg| \leqslant
        \|x\| \int\limits_{\Omega_{1,2}}\bigg|\frac{A(s, t_1)}{|s - t_1|^\alpha} -
        \frac{A(s, t_2)}{|s - t_2|^\alpha}\bigg|\dif s =
        $$
        $$
        = \|x\| \int\limits_{\Omega_{1,2}} \bigg|
        \frac{A(s, t_1)|s - t_2|^\alpha - A(s, t_2)|s - t_1|^\alpha}
        {|s - t_1|^\alpha|s - t_2|^\alpha}\bigg| \dif s \leqslant
        $$
        $$
        \leqslant
        \frac{\|x\|}{(\frac{\rho}{2})^{2\alpha}}
        \underbrace{\int\limits_\Omega \Big|A(s, t_1)|s - t_2|^\alpha -
        A(s, t_2)|s - t_1|^\alpha\Big| \dif s}_{\underto{|t_1 - t_2| \to 0} 0}
        $$
        Мы воспользовались тем, что $|s - t_1|^\alpha \geqslant (\frac{\rho}{2})^\alpha$ и
        $|s - t_2|^\alpha \geqslant (\frac{\rho}{2})^\alpha$.
        
        Возьмём $g \in C(\ol \Omega \times \ol \Omega \times \ol \Omega)$:
        $g(s, t_1, t_2) = A(s, t_1)|s-t_2|^\alpha$. g равномерно непрерывно:
        $$
        \forall \widetilde\varepsilon > 0\quad \exists \widetilde\delta > 0:
        |(s, t_1, t_2) - (s', t'_1, t'_2)| =
        $$
        $$
        =\sqrt{|s-s'|^2 + |t_1 - t'_1|^2 + |t_2 - t'_2|^2} < \widetilde\delta \implies
        |g(s, t_1, t_2) - g(s', t'_1, t'_2)| < \widetilde\varepsilon
        $$
        
        Возьмём $\widetilde\varepsilon = \|x\|^{-1}\big(\frac{\rho}{2}\big)^{2\alpha}
        |\Omega|^{-1}
        \frac{\varepsilon}{2}$.
        Заметим, что под интегралом стоит
        $|g(s, t_1, t_2) - g(s, t_2, t_1)| < \widetilde\varepsilon$
        (при $|(s, t_1, t_2) - (s, t_2, t_1)| = \sqrt{2}|t_1 - t_2| < \widetilde\delta$).
        Отсюда находим $\widetilde\delta$.
        В результате $\delta = \frac{\widetilde\delta}{\sqrt{2}}$.
    \end{enumerate}
\end{proof}

\subsection{Скалярное произведение}

Пусть $X$ — линейное множество над полем скаляров $\real$($\complex$).

\begin{definition}
    $\varphi: X \times X \to \real(\complex)$ называется скалярным произведением,
    если:
    \begin{enumerate}
        \item $\varphi(x_1 + x_2, y) = \varphi(x_1, y) + \varphi(x_2, y)$
        \item $\varphi(\lambda x, y) = \lambda \varphi(x, y)$
        \item $\varphi(x, y) = \ol{\varphi(y, x)}$
        \item $\varphi(x, x) \geqslant 0$ $\forall x \in X$,
        $\varphi(x, x) = 0 \iff x = 0$
    \end{enumerate}
\end{definition}

\begin{proposition}[Свойства скалярного произведения]
\mbox{}
    \begin{enumerate}
        \item $\varphi\Big(\sum\limits_{j = 1}^n \lambda_jx_j, y\Big) =
        \sum\limits_{j=1}^n \lambda_j\varphi(x_j, y)$
        \item\label{scalar_product_halflinearity}
        $\varphi\Big(x, \sum\limits_{j = 1}^n \lambda_jy_j\Big) =
        \sum\limits_{j=1}^n \ol{\lambda_j}\varphi(x, y_j)$
        \item Неравенство Коши-Буняковского:
        $|\varphi(x, y)|^2 \leqslant \varphi(x, x)\varphi(y, y)$
        \item\label{scalar_product_norm} $p(x) = \sqrt{\varphi(x, x)}$
        является нормой.
    \end{enumerate}
\end{proposition}
\begin{proof}
    Докажем свойство \ref{scalar_product_halflinearity}.
    $$
    \varphi\bigg(x, \sum\limits_{j=1}^n \lambda_j y_j\bigg) =
    \ol{\varphi\bigg(\sum\limits_{j=1}^n \lambda_j y_j, x\bigg)} =
    \ol{\sum\limits_{j=1}^n\lambda_j \varphi(y_j, x)} =
    \sum\limits_{j=1}^n \ol{\lambda_j}\varphi(x, y_j)
    $$
    
    Докажем неравенство Коши-Буняковского. Возьмём какой-нибудь скаляр $\lambda$.
    $$
    0 \leqslant \varphi(x + \lambda y, x + \lambda y) = \varphi(x, x) +
    \varphi(x, \lambda y) + \varphi(\lambda y, x) +
    \varphi(\lambda y, \lambda y) =
    $$
    $$
    = \varphi(x, x) + \ol\lambda\varphi(x, y) +
    \underbrace{\lambda\varphi(y, x)}_{\lambda\ol{\varphi(x, y)}} +
    \underbrace{\lambda\ol\lambda\varphi(y, y)}_{|\lambda|^2\varphi(y, y)}
    $$
    Выберем $\lambda$ следующим образом:
    $\lambda = t\varphi(x, y)$ ($t \in \real)$. Тогда получим:
    $$
    = \varphi(x, x) + 2t|\varphi(x, y)|^2 + t^2|\varphi(x, y)|^2\varphi(y, y)
    $$
    Дискриминант этого трёхчлена $D = 4|\varphi(x, y)|^4 -
    4|\varphi(x, y)|^2 \varphi(x, x)\varphi(y, y) \leqslant 0$. Отсюда следует,
    что $|\varphi(x, y)|^2 \leqslant \varphi(x, x)\varphi(y, y)$.
    
    В свойстве \ref{scalar_product_norm} проверим аксиомы нормы:
    \begin{enumerate}
        \item $p(x) \geqslant 0$, $p(x) = 0 \iff x = 0$
        \item $p(\lambda x) = \sqrt{\varphi(\lambda x, \lambda x)} =
        \sqrt{\lambda\ol\lambda\varphi(x, x)} = |\lambda|\sqrt{\varphi(x, x)} =
        |\lambda|p(x)$
        \item Требуется $p(x + y) \leqslant p(x) + p(y)$.
        $$
        \sqrt{(\varphi(x + y, x + y)} \leqslant \sqrt{\varphi(x, x)} +
        \sqrt{(\varphi(y, y)} \iff
        $$
        $$
        \iff \varphi(x + y, x + y) \leqslant \varphi(x, x) +
        2\sqrt{\varphi(x, x)\varphi(y, y)} + \varphi(y, y) \iff
        $$
        $$
        \iff \underbrace{\varphi(x, y) + \varphi(y, x)}_{2\Ree\varphi(x, y)}
        \leqslant 2\sqrt{\varphi(x, x)\varphi(y, y)}
        $$
        $$
        2\Ree \varphi(x, y) \leqslant 2|\varphi(x, y)| \leqslant
        2\sqrt{\varphi(x, x) \varphi(y, y)}
        $$
    \end{enumerate}
\end{proof}
\begin{remark}
    В пространстве со скалярным произведением можно естественным образом завести 
    норму.
\end{remark}

\begin{definition}
    Пространство со скалярным произведением $(X, \varphi)$ называется унитарным.
    Полное унитарное пространство называется гильбертовым (обозначается $H$).
\end{definition}

\begin{proposition}[Непрерывность скалярного произведения]
    Если  $x_n \to x$, $y_n \to y$, то $(x_n, y_n) \to (x, y)$.
\end{proposition}
\begin{proof}
    $$
    |(x_n, y_n) - (x, y)| = |(x_n, y_n) - (x_n, y) + (x_n, y) - (x, y)| \leqslant
    |(x_n, y_n - y)| + |(x_n - x, y)| \leqslant
    $$
    $$
    \leqslant \|x_n\| \cdot \|y_n - y\| + \|x_n - x\|\cdot \|y\| \to 0
    $$
\end{proof}

\begin{proposition}
    Пусть $\sum\limits_{j=1}^\infty x_j$ — сходящийся ряд в $H$ — гильбертовом
    пространстве. Тогда $\bigg(\sum\limits_{j=1}^\infty x_j, y\bigg) =
    \sum\limits_{j=1}^\infty (x_j, y)$.
\end{proposition}
\begin{proof}
    Пусть $S_n = \sum\limits_{j=1}^n x_j$, $S_n \to S$. Тогда
    $(S_n, y) \to (S, y) = \bigg(\sum\limits_{j=1}^\infty x_j, y\bigg)$.
    $$
    (S_n, y) = \bigg(\sum\limits_{j=1}^n x_j, y\bigg) =
    \sum\limits_{j=1}^n(x_j, y) \to
    \sum\limits_{j=1}^\infty(x_j, y)
    $$
\end{proof}

\begin{proposition}[Тождество параллелограмма]
    $\|x + y\|^2 + \|x - y\|^2 = 2(\|x\|^2 + \|y\|^2)$
\end{proposition}
\begin{proof}
    Запишем равенства:
    $$
    \|x \pm y\| = (x \pm y, x \pm y) = \|x^2\| \pm (x, y) \pm (y, x) + \|y^2\|
    $$
    Сложим их. Получим требуемое.
\end{proof}

\begin{example}
    Рассмотрим пространство $X = C[0, 1]$, функции $x(t) = 1$ и $y(t) = t$.
    $\|x + y\| = 2$, $\|x - y\| = 1$, $\|x\| = 1$, $\|y\| = 1$. Отсюда
    $2^2 + 1^2 = 2\cdot (1^2 + 1^2)$ — неверно. Из этого следует, что в 
    пространстве $C[0, 1]$ нельзя ввести скалярное произведение, согласованное
    с естественной нормой.
\end{example}

\begin{proposition}[Формула восстановления]
    Если пространство унитарное, то в нём можно восстановить скалярное 
    произведение по норме. Для вещественного случая:
    $$
    (x, y) = \frac{1}{4}(\|x + y\|^2 - \|x - y\|^2)
    $$
    Для комплексного случая:
    $$
    (x, y) = \frac{1}{4}(\|x + y\|^2 - \|x - y\|^2 + i\|x + y\|^2 - i\|x - y\|^2)
    $$
\end{proposition}
\begin{proof}
    Упражнение.
\end{proof}

\begin{example}
    В пространстве $L^2(T, \mu)$:
    $$
    (x, y) = \int\limits_T x(t)\ol{y(t)}\dif \mu(t)
    $$
    $$
    \sqrt{(x, x)} = \bigg(\int\limits_T |x(t)|^2\dif \mu(t)\bigg)^\frac{1}{2} =
    \|x\|_2
    $$
\end{example}

\subsection{Ортогональность}

\begin{definition}
    Пусть $H$ — гильбертово пространство. Векторы $x, y$ ортогональны, если
    $(x, y) = 0$.
\end{definition}

\begin{proposition}
    \begin{enumerate}
        \item $x \perp x \iff x \perp H \iff x = 0$.
        \item $x \perp y_1, y_2 \implies x \perp (\alpha_1y_1 + \alpha_2y_2)$.
        \item $x \perp y_n$ $\forall n \in \mb N$, $y_n \to y$ $\implies
        x \perp y$.
        \item $x \perp A \implies x \perp \ol{\Lin(A)}$.
        \item (Теорема Пифагора) Если для $x_1, \ldots, x_n$ $x_j \perp x_k$
        $\forall j \neq k$, то $\|x_1 + \ldots + x_n\|^2 =
        \|x_1\|^2 + \ldots + \|x_n\|^2$.
    \end{enumerate}
\end{proposition}

\begin{definition}
    Ряд $\sum\limits_{j=1}^\infty x_j$ в гильбертовом пространстве $H$ называется
    ортогональным, если $\forall j\neq k$ $x_j \perp x_k$.
\end{definition}

\begin{theorem}[О сходимости ортогонального ряда]
\label{orthogonal_series_convergence}
    Рассмотрим ортогональный ряд (1) $\sum\limits_{j=1}^\infty x_j$ и ряд (2)
    $\sum\limits_{j=1}^\infty \|x_j\|^2$. Ряд (1) сходится тогда и только тогда,
    когда ряд (2) сходится. В случае сходимости выполняется теорема Пифагора
    $\Big\|\sum\limits_{j=1}^\infty x_j\Big\|^2 = \sum\limits_{j=1}^\infty \|x_j\|^2$.
\end{theorem}
\begin{proof}
    Обозначим $S_n = \sum\limits_{j=1}^nx_j$,
    $C_n = \sum\limits_{j=1}^n\|x_j\|^2$. Возьмём $m > n$ и воспользуемся 
    критерием Коши.
    $$
    \|S_m - S_n\|^2 = \bigg\|\sum\limits_{j=n+1}^m x_j\bigg\|^2 =
    \sum\limits_{j=n+1}^m \|x_j\| = C_m - C_n = |C_m - C_n|
    $$
    Таким образом, фундаментальность последовательности $S_n$ равносильна
    $\|S_m - S_n\| \underto{m,n \to \infty} 0 \iff \|C_m - C_n\|
    \underto{m,n \to \infty} 0$, что равносильно фундаментальности $C_n$.
    Получаем $\|S_n\|^2 = C_n$.
\end{proof}

\begin{example}[Ряд Фурье]
    $b_0 + \sum\limits_{n \in \mb N}a_n\sin nt + b_n \cos nt$ в пространстве
    $L^2(0, 2\pi)$.
    $$
    \|a_n \sin nt\|_2^2 = |a_n|^2\int\limits_0^{2\pi}|\sin nt|^2\dif t =
    |a_n|^2\pi
    $$
    $$
    \|b_n\cos nt\|^2 = |b_n|^2 \pi
    $$
    $$
    \|b_0\|^2 = b_0^2\cdot 2\pi
    $$
    Чтобы ряд сходился в $L^2(0, 2\pi)$, необходимо и достаточно, чтобы 
    выполнялось $\pi\sum\limits_{n=1}^\infty(|a_n|^2 + |b_n|^2) < \infty$.
\end{example}
\begin{exercise}
    Доказать, что:
    \begin{enumerate}
        \item $\int\limits_0^{2\pi} \cos mt\sin nt \dif t = 0$;
        \item $\int\limits_0^{2\pi} \sin mt\sin nt \dif t = 0$ при $m \neq t$.
    \end{enumerate}
\end{exercise}

\begin{example}
    $\sum\limits_{n \in \mb Z} a_n e^{int}$ в $L^2(0, 2\pi)$.
    $$
    (e^{int}, e^{imt}) = \int\limits_0^{2\pi}e^{int}\ol{e^{imt}}\dif t =
     \int\limits_0^{2\pi} e^{i(n - m)t}\dif t =
     \begin{cases}
         2\pi,\quad n = m \\
         \frac{e^{i(n-m)t}}{i(n-m)}\bigg|_0^{2\pi},\quad n \neq m \\
     \end{cases} =
     \begin{cases}
         2\pi,\quad n = m \\
         0,\quad n \neq m \\
     \end{cases}
    $$
\end{example}

\subsection{Теорема о наилучшем приближении}

\begin{example}
    Допустим, мы рассматриваем пространство $C[-1, 1]$. Возьмём в этом
    пространстве функцию $x_0(t) = t^n$ и множество
    $A = \Lin\{1, t, t^2, \ldots, t^{n-1}\}$ многочленов степени не выше $n-1$.
    Мы хотим найти $\dist(x_0, A) = \inf_{y\in A}\|x_0 - y\|_{C[-1, 1]}$. 
    Необходимо ответить на ряд вопросов: существует ли минимум? единственен ли он?
    как его искать?
\end{example}

\begin{definition}
    $y_0$ называется наилучшим приближением к $x_0$ в $A$, если
    $\dist(x_0, A) = \|x_0 - y_0\|$ и $y_0 \in A$.
\end{definition}

\begin{theorem}[О наилучшем приближении]
    Пусть $H$ — гильбертово пространство, $A \subset~H$ — замкнутое и выпуклое 
    множество, $x_0 \in H$. Тогда существует и единственно наилучшее
    приближение к $x_0$ в $A$.
\end{theorem}
\begin{proof}
    Начнём с единственности. Предположим, что есть две точки $y_1, y_2 \in A$.
    Может ли так случиться, что они обе минимизируют расстояние до $x_0$? Заметим,
    что любая точка на интервале $y_1y_2$ будет ближе к $x_0$, чем $y_1$ и $y_2$.
    Возьмём векторы $u = x_0 - y_1$ и $v = x_0 - y_2$. По тождеству 
    параллелограмма $\|u + v\|^2 + \|u - v\|^2 = 2(\|u\|^2 + \|v\|^2)$, откуда:
    $$
    \underbrace{\|2x_0 - (y_1 + y_2)\|^2}_{(\ast)} + \|y_1 - y_2\|^2 =
    2(\|x_0 - y_1\|^2 + \|x_0 - y_2\|^2)
    $$
    Обозначим $\dist(x_0, A) = d$.
    $$
    (\ast) = 4\|x_0 - \frac{y_1 + y_2}{2}\|^2 \geqslant 4d^2
    $$
    Итого, получаем:
    $$
    \|y_1 - y_2\|^2 \leqslant 2(\|x_0 - y_1\|^2 + \|x_0 - y_2\|^2) - 4d^2
    \quad \forall y_1, y_2 \in A
    $$
    Таким образом, если $y_1, y_2$ — наилучшие приближения, то $\|x_0 - y_1\| = d$
    и $\|x_0 - y_2\| = d$, то $\|y_1 - y_2\|^2 \leqslant 0$, отсюда $y_1 = y_2$.
    Единственность доказана.
    
    Докажем существование. Возьмём последовательность $y_n \in A$ такую, что
    $\|x_0 - y_n\| \to d$ ($y_n$ — \emph{минимизирующая последовательность}).
    $$
    \|y_n - y_m\|^2 \leqslant 2(\underbrace{\|x_0 - y_n\|^2}_{\to d^2} +
    \underbrace{\|x_0 - y_m\|^2}_{\to d^2}) - 4d^2 \to 0
    $$
    Из этого следует, что $\|y_n - y_m\| \underto{m,n \to \infty} 0 \implies
    \exists y_0 = \lim y_n \in A$. $\|x_0 - y_0\| = d$. Существование доказано.
\end{proof}

\begin{theorem}[О проекции]
    Пусть $H$ — гильбертово пространство, $L \subset H$ — линейное замкнутое 
    подмножество $H$. Тогда для любого $x \in H$ существует единственная пара
    элементов $y \in L$, $z \perp L$ таких, что $x = y + z$.
\end{theorem}
\begin{definition}
    $y$ называется \emph{проекцией} вектора $x$ на $L$.
\end{definition}
\begin{proof}[Доказательство теоремы]
    Так как $L$ линейно, то оно выпукло. Значит, существует наилучшее приближение
    $y \in L$ для $x$. Определим $z$ как $x - y$. Проверим, что $z \perp L$.
    Возьмём произвольный вектор $l \in L$, скаляр $\lambda$. Будем рассматривать
    вектор $y + \lambda l$. Очевидно, что
    $\|y + \lambda l - x\|^2 \geqslant \|y - x\|^2$, то есть
    $\|\lambda l - z\|^2 \geqslant \|z\|^2$. Раскроем скобки в скалярных 
    квадратах:
    $$
    (\lambda l, \lambda l) - (\lambda l, z) - (z, \lambda l) + (z, z) \geqslant 
    (z, z)
    $$
    $$
    |\lambda|^2(l, l) - \lambda(l, z) - \ol\lambda(z, l) \geqslant 0\quad
    \forall \lambda \in \complex
    $$
    Пусть $\lambda = t(z, l)$ ($t \in \real$). Тогда
    $$
    t^2|(z, l)|^2(l, l) - 2t|(z, l)|^2 \geqslant 0
    $$
    $$
    |(z, l)|^2(t^2\|l\|^2 - 2t)\geqslant 0
    $$
    Заметим, что если $(z, l) \neq 0$, то $t^2\|l\|^2 \geqslant 2t$ $\forall t$, что неверно.
    Значит, $(z, l) = 0$. Существование доказано.
    
    Докажем единственность. Пусть существует два разложения:
    $x = y_1 + z_1 = y_2 + z_2$ и $y_1,y_2\in L$, $z_1,z_2 \perp L$.
    Рассмотрим $w = y_1 - y_2 = z_2 - z_1$. Но $y_1 - y_2 \in L$, а
    $z_2 - z_1 \perp L$, значит, $w \perp w \implies w = 0$.
\end{proof}

\begin{corollary}
    В условиях предыдущей теоремы
    $\|x\| \geqslant \|y\|$ и $\|x\| \geqslant \|z\| = \dist(x, L)$.
\end{corollary}
\begin{proof}
    $$
    \|x\|^2 = \|y + z\|^2 = \|y\|^2 + \|z\|^2
    $$
\end{proof}

\subsection{Ортогональное дополнение и ортогональные проекторы}

В этом параграфе рассматриваем гильбертово пространство $H$.

\begin{definition}
    Если $A \subset H$, то его ортогональным дополнением называется
    $A^\perp = \{x \in H:\,\forall y \in A\quad x\perp y\}$.
\end{definition}
\begin{proposition}[Свойства ортогонального дополнения]
\mbox{}
    \begin{enumerate}
        \item $A^\perp$ — линейное.
        \item $A^\perp$ — замкнутое множество, то есть если $x_n \in A^\perp$,
        $x_n \to x$, то $x \in A^\perp$.
        \item $(\Lin A)^\perp = A^\perp$.
        \item $(\ol A)^\perp = A^\perp$.
        \item $(\ol{\Lin A})^\perp = A^\perp$.
        \item $A\subset B \implies A^\perp \supset B^\perp$.
        \item $\{0\}^\perp = H$, $H^\perp = \{0\}$.
        \item\label{closed_perp_addition} Если $L$ — линейное подмножество $H$,
        то $(L^\perp)^\perp = \ol L$.
    \end{enumerate}
\end{proposition}
\begin{proof}
    Докажем свойство \ref{closed_perp_addition}. Пусть $L = \ol L$. Тогда
    $(L^\perp)^\perp = \{x \in H:\, x\perp L^\perp\} \supset L$.
    Предположим, что
    $(L^\perp)^\perp \neq L$, $x \in (L^\perp)^\perp \backslash L$.
    По теореме о проекции $x$ представляется в виде $x = y + z$, где
    $y \in L \subset (L^\perp)^\perp$,
    $z \in L^\perp$. Тогда
    $\underbrace{x - y}_{\in (L^\perp)^\perp} = z \in L^\perp$. Это означает,
    что $x-y \perp Z$, откуда $z = x - y = 0$, то есть $x = y \in L$ — 
    противоречие. Значит, в случае замкнутого $L$ его второе ортогональное 
    дополнение совпадает с ним самим. Рассмотрим случай, когда $L$ незамнкуто.
    $L^\perp = (\ol L)^\perp$, $(L\perp)^\perp = ((\ol L)^\perp)^\perp = \ol L$.
\end{proof}

\begin{definition}
    Пусть $L \subset H$ — замкнутое линейное множество и для всех $x \in H$
    существует единственные $y, z$ такие, что $y \in L$, $z \in L^\perp$,
    $x = y + z$. Отображение $P_L: H \to H$, $x \mapsto y$
    называется ортогональным проектором.
\end{definition}

\begin{proposition}[Свойства ортогонального проектора]
    \begin{enumerate}
        \item $P_L$ — линейное отображение.
        \item $P_L \in B(H, H)$, т. е. $P_L$ непрерывно.
    \end{enumerate}
\end{proposition}
\begin{proof}
    \begin{enumerate}
        \item 
        \item Проверим, что $\|P_Lx\| \leqslant C\|x\|$. Предствавим $x$ в виде
        $x = y + z$, где $y \in L$, $z \in L^\perp$. Отсюда
        $\|x\|^2 = \|y\|^2 + \|z\|^2 \implies \|y\| \leqslant \|x\|$. То есть,
        $\|P_Lx\| \leqslant \|x\| \implies \|P_L\| \leqslant 1$.
        \item $\Ker P_L = L^\perp$.
        \item $P_L(H) = L$.
        \item $P_L + P_{L^\perp} = I$
    \end{enumerate}
\end{proof}
\begin{remark}
    Если $L \neq \{0\}$, то $\|P_L\| = 1$ (так как $\forall x \in L P_Lx = x$.
\end{remark}

\begin{theorem}[О характеристике ортогональных проекторов]
    Пусть $U \in B(H, H)$. Чтобы $U$ был ортогональным проектором,
    необходимо и достаточно, чтобы выполнялись условия:
    \begin{enumerate}
        \item $U^2 = U$ (идемпотентность).
        \item $\forall x_1, x_2 \in H$ $(Ux_1, x_2) = (x_1, Ux_2)$
        (самосопряжённость).
    \end{enumerate}
\end{theorem}
\begin{proof}
    Предположим, что $U = P_L$. Проверим идемпотентность.
    Если $x = y + z$, где $y \in L$, $z \in L^\perp$, то $P_Lx = y$, $P_Ly = y$.
    Проверим самосопряжённость. Пусть $x_1 = y_1 + z_1$, $x_2 = y_2 + z_2$
    ($y_j \in L$, $z_j \in L^\perp$). $P_Lx_j = y_j$.
    $(y_1, y_2 + z_2) = (y_1, y_2) = (y_1 + z_1, y_2)$.
    
    Обратно. Положим $L = \{x\in H:\, Ux=x\} = \Ker(U-I)$. Ясно, что $L$ — 
    линейное замкнутое множество (ядро непрерывного оператора всегда 
    замкнуто). Проверим, что $\forall x\in H$ $Ux \in L$, то есть $U(Ux) = Ux$.
    Ясно, что это выполнено. Далее, представим $x$ в виде
    $x = \underbrace{Ux}_{\in L} + (x - Ux)$. Нужно доказать, что
    $x - Ux \in L^\perp$. Возьмём произвольное $y \in L$. Посчитаем скалярное 
    произведение $(x - Ux, y) = (x, y) - (Ux, y) = (x, y) - (x, Uy) =
    (x, y) - (x, y) = 0$.
\end{proof}

\subsection{Ряды Фурье}

\begin{definition}\label{orthogonal_system}
    Пусть $H$ — гильбертово пространство. Система векторов
    $\{e_\alpha\}_{\alpha \in A} \subset H$ ($A$ — некоторое множество индексов)
    называется ортогональной, если $e_\alpha \neq 0 \forall \alpha \in A$ и
    $(e_{\alpha_1}, _{\alpha_2}) = 0$, если $\alpha_1 \neq \alpha_2$. Система
    векторов называется ортонормированной, если она ортогональна и норма каждого
    вектора системы равна единице.
\end{definition}

\begin{remark}
    Чаще всего множество $A$ из определения \ref{orthogonal_system} является
    множеством натуральных или целых чисел, но мы будем перенумеровывать индексы 
    так, чтобы рассматривать множество натуральных чисел.
\end{remark}

\begin{proposition}
    Если система $\{e_j\}_{j=1}^\infty$ ортогональна, то она линейно независима.
\end{proposition}
\begin{proof}
    Предположим, что система линейно зависима. Тогда существуют скаляры
    $\lambda_j$ такие, что $\sum\limits_{j=1}^n \lambda_je_j = 0$. Умножим
    это равенство скалярно на $e_m$. Получим
    $\sum\limits_{j=1}^n \lambda_j(e_j, e_m) = 0$. Отсюда
    $\lambda_m(e_m, e_m) = 0$, но $(e_m, e_m) \neq 0$, значит,
    $\lambda_m = 0 \forall m$.
\end{proof}

\begin{proposition}
    Пусть $\{e_j\}$ — ортогональная система и $x\in H$. Предположим, что $x$ 
    представимо в виде $x = \sum\limits_{j=1}^\infty \lambda_je_j$. Тогда такое
    представление единственно.
\end{proposition}
\begin{proof}
    Рассмотрим представление $x = \sum\limits_{j=1}^\infty \lambda_je_j$, умножим 
    его скалярно на $e_m$.
    $$
    (x, e_m) = (\sum\limits_{j=1}^\infty \lambda_je_j, e_m) =
    \sum\limits_{j=1}^\infty(\lambda_je_j, e_m) = \lambda_m(e_m, e_m)
    $$
    Отсюда $\lambda_m = \frac{(x, e_m)}{(e_m, e_m)}$, то есть коэффициент
    $\lambda_m$ однозначно определяется по $x$.
\end{proof}
\begin{remark}
    Даже если $x$ не представляется в виде ряда, можно вычислить величины
    $\frac{(x, e_m)}{(e_m, e_m)} = \frac{(x, e_m)}{\|e_m\|^2}$.
\end{remark}
\begin{definition}
    $c_m(x) = \frac{(x, e_m)}{\|e_m\|^2}$ — \emph{коэффициенты Фурье} вектора
    $x$ по ортогональной системе $\{e_j\}_{j=1}^\infty$.
    $\sum\limits_{j=1}^\infty c_j(x)$ — ряд Фурье вектора $x$ по этой системе.
\end{definition}

Возникают естественные вопросы:
\begin{itemize}
    \item Для всех ли $x \in H$ ряд Фурье сходится?
    \item Если ряд Фурье сходится, то сходится ли он к $x$?
    \item Как определить, к $x$ или не к $x$ он сходится?
\end{itemize}

\begin{example}\label{fourier_series_bad_example}
    Пусть $H = \real^3$, $e_1 = (1, 0, 0)$, $e_2 = (0, 1, 0)$. Возьмём вектор
$x = (x_1, x_2, x_3)$. $c_1(x) = x_1$, $c_2(x) = x_2$. Значит, ряд Фурье $x$ равен
$x_1e_1 + x_2e_2 = (x_1, x_2, 0)$. Он сходится как конечная сумма, но не к $x$, а
к его проекции на подпространство, натянутое на $e_1$, $e_2$.
\end{example}

\begin{theorem}[О частичных суммах ряда Фурье]
    Пусть у нас есть ортогональная система $\{e_j\}$ в гильбертовом пространстве
    $H$, есть вектор $x$ и его ряд Фурье $\sum\limits_{j=1}^\infty c_j(x)e_j$.
    Рассмотрим $S_n(x) = \sum\limits_{j=1}^n c_j(x)e_j$,
    $L_n = \Lin \{e_1,\ldots,e_n\}$. Тогда:
    \begin{enumerate}
        \item $x - S_n(x) \perp L_n$.
        \item $\|S_n(x)\| \leqslant \|x\|$
    \end{enumerate}
\end{theorem}
\begin{proof}
    Докажем первое утверждение. Возьмём $m: 1 \leqslant m \leqslant n$.
    $$
    (x - S_n(x), e_m) = (x, e_m) - (\sum\limits_{j=1}^n c_j(x)e_j, e_m) =
    (x, e_m) - c_m(x)(e_m, e_m) = 0
    $$
    Докажем второе утверждение. $x = \underbrace{S_n(x)}_{\in L_n} +
    \underbrace{(x - S_n(x))}_{\perp L_n}$. Отсюда $S_n(x) = P_{L_n}(x)$ и
    $\|S_n(x)\|\leqslant \|x\|$, $\|x - S_n(x)\| \leqslant \|x\|$.
\end{proof}
\begin{corollary}
\mbox{}
    \begin{enumerate}
        \item $\|S_n(x)\|^2 = \|\sum\limits_{j=1}^n c_j(x)e_j\|^2 =
        \sum\limits_{j=1}^n |c_j(x)|^2 \cdot \|e_j\|^2 \leqslant \|x\|^2$.
        \item (Неравенство Бесселя)
        $\sum\limits_{j=1}^\infty |c_j(x)|\cdot \|e_j\|^2 \leqslant \|x\|^2$
    \end{enumerate}
\end{corollary}

\begin{theorem}[Риса-Фишера]
    Пусть $\{e_j\}$ — ортогональная система в $H$ — гильбертовом пространстве,
    $x \in H$. Тогда:
    \begin{enumerate}
        \item Ряд Фурье для $x$ сходится.
        \item Если $S(x)$ — сумма этого ряда, то $x - S(x) \perp e_j \forall j$.
        \item $x = S(x) \iff \sum\limits_{j=1}^\infty |c_j(x)|^2\cdot\|e_j\|^2 =
        \|x\|^2$
    \end{enumerate}
\end{theorem}
\begin{proof}
    \begin{enumerate}
        \item Строим ортогональный ряд $\sum\limits_{j=1}^\infty c_j(x)e_j$. По 
        теореме \ref{orthogonal_series_convergence} этот ряд сходится тогда и 
        только тогда, когда $\sum\limits_{j=1}^\infty \|c_j(x)e_j\|^2$
        сходится, то 
        есть $\sum\limits_{j=1}^\infty |c_j(x)|^2\|e_j\|^2$ сходится, что верно по 
        неравенству Бесселя.
        \item $(x - S9x), e_j) = (x, e_j) -
        (\sum\limits_{k=1}^\infty c_k(x)e_k, e_j) = (x - e_j) - c_j(x)(e_j, e_j) = 
        0$
        \item $x = \underbrace{z}_{\in L^\perp} + \underbrace{S(x)}_{\in L}$, где
        $L = \ol{\Lin\{e_j\}}$. По теореме Пифагора
        $\|x\|^2 = \|z\|^2 + \|S(x)\|^2$. Отсюда $x = S(x) \iff z = 0 \iff
        \|x\|^2 = \|S(x)\|^2 = \|\sum\limits_{j=1}^\infty c_j(x) e_j\|^2 =
        \sum\limits_{j=1}^\infty |c_j(x)|^2\|e_j\|^2$.
    \end{enumerate}
\end{proof}

\begin{definition}
    Рассмотрим ортогональную систему $\{e_j\} \subset H$. Эта система называется
    \emph{ортогональным базисом} в $H$, если $\forall x \in H$ $S(x) = x$.
\end{definition}

Возвращаясь к примеру \ref{fourier_series_bad_example}, легко видеть, что 
выбранная в нём ортогональная система не является базисом.

\begin{definition}
    Система векторов $A \subset H$ называется \emph{полной}, если из $x \perp A$
    следует $x = 0$, иначе говоря, $A^\perp = \{0\}$.
\end{definition}

\begin{definition}
    Система векторов $A \subset H$ называется \emph{порождающей}, если
    $\ol{\Lin A} = H$. (Здесь $H$ — не обязательно гильбертово пространство.)
\end{definition}

\begin{theorem}[О характеристике ортогонального базиса]
    Пусть $\{e_j\}$ — ортогональная система в $H$ — гильбертово.
    Тогда следующие утверждения 
    равносильны:
    \begin{enumerate}
        \item $\{e_j\}$ — ортогональный базис.
        \item $\forall x \in H \|x\|^2 =
        \sum\limits_{k=1}^\infty |c_j(x)|^2\|e_j\|^2$.
        \item $\{e_j\}$ — порождающая система.
        \item $\{e_j\}$ — полная система.
    \end{enumerate}
\end{theorem}
\begin{proof}
\mbox{}
    \begin{enumerate}
        \item ($1 \Leftrightarrow 2$). Утверждается в теореме Риса-Фишера.
        \item ($1 \Rightarrow 3$). $x = \sum\limits_{j=1}^\infty c_j(x)e_j =
        \lim\limits_{n \to \infty}
        \underbrace{\sum\limits_{j=1}^n c_j(x) e_j)}_{\in \Lin \{e_j\}} \in
        \ol{\Lin\{e_j\}} \implies \ol{\Lin \{e_j\}} = H$.
        \item ($3 \Rightarrow 4$). $\{e_j\}$ — порождающая система, значит,
        $\ol{\Lin\{e_j\}} = H$. $\{e_j\}^\perp = (\ol{\Lin\{e_j\}})^\perp =
        H_\perp = \{0\} \implies \{e_j\}$ — полная.
        \item ($4 \Rightarrow 1$). $x = z + S(x)$, где $z \perp e_j \forall j$.
        Это означает, что $z = 0$, так как система полная, и $x = S(x)$.
    \end{enumerate}
\end{proof}

\begin{examples}
\mbox{}
    \begin{enumerate}
        \item $\ell^2$. $e_j =
        (0, 0, \ldots, \underbrace{1}_j, 0, \ldots)$.
        $\{e_j\}_{j=1}^\infty$ — ортонормированный базис.
        \item $L^2(0, 2\pi)$.
        $\{1, \sin t, \cos t, \sin 2t, \cos 2t, \ldots, \ldots\}$ —
        ортогональный базис. Как доказать, 
        что это действительно базис? Проще всего в данной ситуации проверить, что
        рассматриваемая система является порождающей. Для этого надо любую функцию
        из $L^2(0, 2\pi)$ научиться приближать линейными комбинациями
        (тригонометрическими многочленами). Таким образом, необходимо доказать, 
        что любая непрерывная функция приближается тригонометрическими 
        многочленами и, кроме того, любая функция из $L^2(0, 2\pi)$ приближается
        непрерывной.
        \item $L^2(0, 2\pi)$. $\{e^{int}\}_{n \in \mb Z}$ — ортогональный базис.
        \item $L^2(0, \pi)$. $\{1,\cos t, \cos 2t, \ldots\}$ — ортогональный 
        базис.
    \end{enumerate}
\end{examples}

\begin{definition}
    Пусть $X$ — нормированное пространство. $X$ сепарабельно, если в нём 
    существует счётное всюду плотное множество.
\end{definition}
\begin{remark}
    Множество $M$ является всюду плотным в $X$, если $\ol M = X$.
\end{remark}

\begin{examples}
\mbox{}
    \begin{enumerate}
        \item $C(\ol \Omega)$ — сепарабельное.
        (Здесь $\Omega \subset \real^n$ — ограниченная область.)
        В этом случае примером счётного всюду плотного множества служит множество
        многочленов с рациональными коэффициентами.
        \item $L^p(\Omega)$ — сепарабельное ($1 \leqslant p < \infty$).
        \item $\ell^p$ ($1 \leqslant p < \infty$). Пример счётного всюду
        плотного множества:
        $M = \{(x_1, x_2,\ldots,x_k, 0, 0, \ldots)\,\big|\,
        k \in \mb N;\, x_n \in \mb Q\}$
        \item $\ell^\infty = \{x = (x_1, \ldots, x_n, \ldots\,\big|\,
         \sup\limits_{k \in \mb N} |x_j| < \infty \}$ — несепарабельное. Почему?
         Предположим, что в нём есть счётное всюду плотное множество $M$.
         Рассмотрим $A \subset \ell^\infty$: $A = \{x = (x_1, x_2,\ldots)\,\big|\,
         x_j \in \{0, 1\}\}$. Заметим, что:
         \begin{enumerate}
            \item $A$ несчётно. ($\ol{0,x_1x_2x_3\ldots} \in [0, 1]$)
            \item $\ldots$
         \end{enumerate}
    \end{enumerate}
\end{examples}

\begin{theorem}[О существовании ортогонального базиса]
    Если $H$ — сепарабельное гильбертово пространство, то в нём существует
    ортогональный базис.
\end{theorem}
\begin{proof}
    Пусть $M$ — счётное всюду плотное множество в $H$:
    $M = \{x_1, x_2, \ldots\}$, $\ol M = H$. Проредим последовательность $x_j$
    так, чтобы все её элементы стали линейно независимы. В результате получим
    новую последовательность $M_1 = \{y_1, y_2,\ldots,y_n,\ldots\} \subset M$.
    Ясно, что $\Lin M_1 \supset M$. Заметим, что $\ol{\Lin M_1} = H$, так как оно
    содержит $\ol M$, и $M_1$ линейно независимо. Будем проводить ортогонализацию:
    Возьмём $e_1 = \frac{y_1}{\|y_1\|}$. Пусть $w_2 = y_2 - (y_2, e_1)e_1$, причём
    $(w_2, e_1) = 0$, $w_2 \neq 0$ и $\Lin\{w_2, e_1\} = \Lin \{y_1, y_2\}$.
    Возьмём $e_2 = \frac{w_2}{\|w_2\|}$. Далее, пусть $w_3 = y_3 - (y_3, e_1)e_1 -
    (y_3, e_2)e_2$ ($(w_3, e_1) = (w_3, e_2) = 0$, $w_3 \neq 0$,
    $\Lin\{w_3, e_1, e_2\} = \Lin\{y_1, y_2, y_3\}$) и берём
    $e_3 = \frac{w_3}{\|w_3\|}$. И так далее. Получим
    $M_2 = \{e_1, e_2, \ldots\}$ — ортонормированную систему.
    $\ol{\Lin M_2} = \ol{\Lin M_1} = H$, то есть, $M_2$ — порождающая система,
    то есть базис.
\end{proof}

\subsection{Теорема Риса}

\begin{lemma}
    Пусть $L$ — линейное множество (над $\real$ или $\complex$),
    $f, g: L \to \real(\complex)$ — линейные функционалы, $\Ker f \subset \Ker g$.
    Тогда существует скаляр $\alpha$ такой, что $g(x) = \alpha f(x)$ $\forall x$.
\end{lemma}
\begin{proof}
\mbox{}
    \begin{enumerate}
        \item Если $f\equiv 0$, то $\Ker f = L = \Ker g \implies g \equiv 0$.
        \item Если $f\cancel \equiv 0$, то $\exists x_0:$ $f(x_0) \neq 0$.
        Возьмём $x\in L$, $y = x - \frac{f(x)}{f(x_0)}x_0$.
        $f(y) = f(x) - \frac{f(x)}{f(x_0)}f(x_0) = 0 \implies
        y \in \Ker f\subset \Ker g \implies g(y) = 0$. Отсюда
        $0 = g(y) = g(x) - \frac{f(x)}{f(x_0)}g(x_0)$ и $\forall x$
        $g(x) = \frac{g(x_0)}{f(x_0)}f(x)$.
    \end{enumerate}
\end{proof}

\begin{theorem}[Риса]
    Пусть $H$ — гильбертово пространство.
    \begin{enumerate}
        \item $\forall y_0 \in H$ $\exists f \in H^\ast:$ $f(x) = (x, y_0)$,
        $\|f\| = \|y_0\|_H$.
        \item $\forall f \in H^\ast$ $\exists y_0 \in H$: $\forall x \in H$
        $f(x) = (x, y_0)$.
    \end{enumerate}
\end{theorem}
\begin{proof}
\mbox{}
    \begin{enumerate}
        \item $f$ линеен, так как скалярное произведение линейно по первому 
        аргументу. Непрерывность $f$ очевидна из неравенства Коши-Буняковского:
        $$
        |f(x)| = |(x, y_0)| \leqslant \|y_0\|\|x\| \implies
        \|f\| \leqslant \|y_0\|
        $$
        $$
        \|f\| = \sup \frac{|f(x)|}{\|x\|} \geqslant \frac{|f(y_0)|}{\|y_0\|} =
        \frac{(y_0, y_0)}{\|y_0\|} = \|y_0\|
        $$
        \item В случае, когда $f \equiv 0$ ясно, что $y_0 = 0$.
        Иначе: $\Ker f \neq H$. Тогда существует $z \neq 0$ такой, что
        $z \perp \Ker f$. Почему это так? Поскольку $f$ непрерывен, то $\Ker f$ —
        замкнутое множество, значит, на него можно спроецировать вектор. Взяв
        $z' \notin \Ker f$, разложим его на две составляющие, одна из которых
        ортогональна $\Ker f$. Примем её за $z$.
        
        Теперь определим $g$ как $g(x) = (x, z)$. Если $f(x) = 0$, то
        $x \in \Ker f$, то есть $x \perp z$, откуда $g(x) = 0$ и $x \in \Ker g$.
        Значит, $\Ker f \subset \Ker g$.
        
        Отсюда по лемме существует $\alpha$: $g(x) = \alpha f(x)$
        $\ldots$
    \end{enumerate}
\end{proof}

\subsection{Сопряжённый оператор}


\begin{theorem}
    Пусть $U \in B(H, H)$, где $H$ — гильбертово пространство. Тогда существует
    единственный оператор $V \in B(H, H)$ такой, что $\forall x,y \in H$
    $(Ux, y) = (x, Vy)$. При этом $\|V\| \leqslant \|U\|$.
\end{theorem}
\begin{definition}
    $V$ называется \emph{сопряжённым оператором} к $U$. Обозначение: $V = U^\ast$.
\end{definition}
\begin{proposition}
    Пусть $x, y \in H$. Если $(x, z) = (y, z) \forall z \in H$, то $x = y$.
\end{proposition}
\begin{proof}
    Из условия следует, что $(x - y, z) = 0 \forall z \in H$, откуда
    $x - y \perp x - y$.
\end{proof}
\begin{proof}[Доказательство теоремы]
    Возьмём вектор $y \in H$ и построим по нему функционал $f \in H^\ast$:
    $f(x) = (Ux, y)$. Очевидно, что он линеен. Проверим непрерывность:
    $$
    |f(x)| = |(Ux, y)| \leqslant \|Ux\|\|y\| \leqslant \big(\|U\|\|y\|\big)\|z\|
    $$
    Мы имеем линейный непрерывный функционал. По теореме Риса существует вектор
    $z$, который его задаёт: $f(x) = (x, z) \forall x$. Таким образом:
    $$
    \forall x\quad (Ux, y) = (x, z)
    $$
    Определим $V(y) = z$, то есть, $\forall x,y$ $(Ux, y) = (x, V(y))$.
    Проверим, что $V$ — линейный непрерывный функционал.
    $$
    (x, V(\alpha_1y_1 + \alpha_2y_2)) = (Ux, \alpha_1x_1 + \alpha_2x_2) =
    \ol{\alpha_1}(Ux, y_1) + \ol{\alpha_2}(Ux, y_2) =
    $$
    $$
    \ol{\alpha_1}(x, V(y_1)( + \ol{\alpha_2}(x, V(y_2) =
    (x, \alpha_1V(y_1) + \alpha_2V(y_2))
    $$
    Так как это выполнено для любого $x$, то $V(\alpha_1y_1 + \alpha_2y_2) =
    \alpha_1V(y_1) + \alpha_2V(y_2)$
    
    Непрерывность $V$:
    $$
    \|Vx\|^2 = (Vx, Vx) = (UVx, x) \leqslant \|UVx\|\cdot\|x\| \leqslant
    \|U\|\|Vx\|\|x\|
    $$
    $$
    \|Vx\| \leqslant \|U\|\|x\|
    $$
    Отсюда $\|V\| \leqslant \|U\|$ и непрерывность доказана.
    
    Докажем единственность $V$. Пусть существуют $V_1$, $V_2$ такие, что
    $\forall x,y$ $(x, V_1y) = (Ux, y) = (x, V_2y)$. Ясно, что $\forall y
    V_1y = V_2y$.
\end{proof}

\begin{proposition}[Свойства сопряжённого оператора]
\mbox{}
    \begin{enumerate}
        \setcounter{enumi}{-1}
        \item $I^\ast = I$, $0^\ast = 0$, $P_L^\ast = P_L$.
        \item $(U^\ast)^\ast = U$.
        \item $\|U^\ast\| = \|U\|$.
        \item $(\alpha_1U_1 + \alpha_2U_2)^\ast =
        \ol{\alpha_1}U_1^\ast + \ol{\alpha_2}U_2^\ast$.
        \item $U, V \in B(H, H)$. Тогда $(VU)^\ast = U^\ast V^\ast$.
        \item $U \in B(H, H)$, $U^{-1} \in B(H, H)$. Тогда
        $\exists (U^\ast)^{-1} \in B(H, H)$, причём
        $(U^\ast)^{-1} = (U^{-1})^\ast$.
        \item (Формула двойственности) $(U(H))^\perp = \Ker U^\ast$,
        $(U^\ast(H))^\perp = \Ker U$.
        \item $(\Ker U^\ast)^\perp = \ol{U(H)}$, $(\Ker U)^\perp = \ol{U^\ast(H)}$
    \end{enumerate}
\end{proposition}
\begin{proof}
\mbox{}
    \begin{enumerate}
        \setcounter{enumi}{-1}
        \item Очевидно.
        \item $\forall x,y$ $(Ux, y) = (x, U^\ast y)$, откуда
        $\underbrace{(U^\ast y, x)}_{(y, U^{\ast\ast}x)} = (y, Ux)$ $\forall x,y$.
        \item $\|U\| \geqslant \|U^\ast\| \geqslant \|U^{\ast\ast}\| = \|U\|$.
        \item $(x, (\alpha_1U_1 + \alpha_2U_2)^\ast y) =
        ((\alpha_1U_1 + \alpha_2U_2)x, y) =
        \alpha_1((U_1x, y) + \alpha_2(U_2x, y) =
        \alpha_1(x, U_1^\ast y) + \alpha_2(x, U_2^\ast y) =
        (x, \ol{\alpha_1}U_1^\ast y + \ol{\alpha_2}U_2^\ast y)$.
        \item $(x, (VU)^\ast y) = (VUx, y) = (Ux, V^\ast y) =
        (x, U^\ast V^\ast y)$.
        \item $UU^{-1} = U^{-1}U = I$. $(U^{-1})^\ast U^\ast =
        U^\ast(U^{-1})^\ast = I^\ast = I$, откуда
        $(U^\ast)^{-1} = (U^{-1})^\ast$.
        \item $x \in \Ker U^\ast \iff U^\ast x = 0 \iff \forall y\in H
        (y, U^\ast x) = 0 \iff \forall y (Uy, x) = 0 \iff x \perp U(H) \iff
        x \in (U(H))^\perp$.
        \item Следует из предыдущего свойства с учётом $(L^\perp)^\perp = \ol{L}$.
    \end{enumerate}
\end{proof}
\begin{definition}
    Оператор $U$ называется замкнутым, если $\ol{U(H)} = U(H)$. В этом случае
    $(\Ker U^\ast)^\perp = U(H)$.
\end{definition}
\begin{remark}
    Рассмотрим задачу $Ux = f$. Для каких $f$ существует решение?
    $f \in U(H)$, то есть $f \perp \Ker U^\ast$ — это условие разрешимости.
\end{remark}

\begin{example}
    Рассмотрим оператор: $Ux(t) = \int\limits_{\Omega} K(s, t)x(s) \dif s$,
    где $\Omega \in \real^m$, $K: \Omega \times \Omega \to \real(\complex)$.
    $U: L^2(\Omega) \to L^2(\Omega)$
    $$
    (Ux, y) = \int\limits_\Omega \bigg(\int\limits_\Omega
    K(s, t) x(s) \dif s\bigg) \ol{y(t)} \dif t =
    \int\limits_\Omega x(s) \underbrace{\int\limits_\Omega
    K(s, t) \ol{y(t) \dif t}}_{\ol{U^\ast y(s)}} \dif s = (x, U^\ast y)
    $$
\end{example}

\subsection{Собственные числа и собственные векторы}

\begin{definition}
    Пусть $X$ — линейное нормированное пространство, $U \in B(X, X)$ — оператор.
    $\lambda$ называется собственным числом оператора $U$, если существует
    $x \in X$, $x \neq 0$ такой, что $Ux = \lambda x$. $x$ называется собственным
    вектором.
    $X_\lambda = \{x \in X \Big | Ux = \lambda x\} = \Ker(U - \lambda I)$~—~
    собственное подпространство. $\dim X_\lambda$ называется кратностью 
    собственного числа $\lambda$.
\end{definition}

\begin{theorem}
    Пусть $H$ — гильбертово пространство, оператор $U$ самосопряжён. Тогда:
    \begin{enumerate}
        \item Все собственные числа оператора $U$ вещественны.
        \item Если $\lambda, \mu$ — собственные числа, $\lambda \neq \mu$ и
        $x, y$ — соответствующие им собственные векторы, то $x \perp y$
    \end{enumerate}
\end{theorem}
\begin{proof}
\mbox{}
    \begin{enumerate}
        \item Пусть $\lambda$ — собственное число, $x$ — собственный вектор.
        $$
        \lambda\|x\|^2 = (\lambda x, x) = (Ux, x) = (x, Ux) = (x, \lambda x) =
        \ol{\lambda}\|x\|^2
        $$
        Отсюда $\lambda = \ol{\lambda}$.
        \item
        $$
        \lambda (x, y) = (\lambda x, y) = (Ux, y) = (x, Uy) = (x, \mu y) =
        \mu(x, y)
        $$
        Отсюда $(\lambda - \mu)(x, y) = 0$.
    \end{enumerate}
\end{proof}

\begin{proposition}
    Если $\lambda$ — собственное число оператора $U$, то
    $\|U\| \geqslant |\lambda|$.
\end{proposition}
\begin{proof}
    $$
    \|U\| = \sup \frac{\|Ux\|}{\|x\|} \geqslant \frac{\|Uy\|}{\|y\|} =
    \frac{|\lambda|\|y\|}{\|y\|} = |\lambda|
    $$
\end{proof}

\subsection{Компактность}

\begin{definition}
    Пусть $X$ — нормированное пространство, $A\subset X$. Множество $A$ называется 
    компактным, если в любой последовательности $\{x_n\} \subset A$ существует сходящаяся
    подпоследовательность $x_{n_k} \to x \in A$.
\end{definition}
\begin{remark}
    Если $A$ — компактно, то оно замкнуто и ограничено. Обратное, вообще говоря, неверно.
\end{remark}
\begin{definition}
    Множество $A$ называется \emph{предкомпактным}, или \emph{относительно компактным}, если
    $\ol A$ компактно. Или, что равносильно, для любой последовательности $\{x_n\} \subset A$
    существует последовательность номеров $n_k$ такая, что $x_{n_k} \to x \in X$.
\end{definition}

\begin{example}
    В $\real^n$ предкомпактность равносильно ограниченности.
\end{example}

\begin{proposition}[Критерий Хаусдорфа]
    Пусть $X$ — нормированное пространство, $A\subset X$. $A$ компактно тогда и только
    тогда, когда $A$ замкнуто и для любого $\varepsilon > 0$ существует конечная
    $\varepsilon$-сеть.
\end{proposition}
\begin{definition}
    Множество $M$ называется $\varepsilon$-сетью для множества $A$, если
    $A \subset \bigcup\limits_{x \in M}B_\varepsilon(x)$ или, что то же самое,
    $\forall a \in A$ $\exists x\in M$: $|x-a|<\varepsilon$.
\end{definition}
Любое множество является $\varepsilon$-сетью для самого себя.
\begin{remark}
    Конечная $\varepsilon$-сеть — это $\varepsilon$-сеть из конечного числа точек.
\end{remark}
\begin{remark}
    В определении $\varepsilon$-сети можно требовать, чтобы сама $\varepsilon$-сеть $M$ была
    подмножеством $A$, а можно не требовать. Критерий Хаусдорфа при этом остаётся в силе.
    Если есть $M \subset X$ — $\varepsilon$-сеть $A$ из конечного числа элементов
    $m_1, m_2, \ldots, m_k$, то существует $2\varepsilon$-сеть из $k$ элементов
    $\{a_1, a_2,\ldots,a_k\} \subset A$. Достаточно выбрать $a_j \in A$ такие, что
    $\|a_j - m_j\| < \varepsilon$.
\end{remark}

\begin{corollary}
    Множество $A$ предкомпактно тогда и только тогда, когда для любого $\varepsilon > 0$ 
    у $A$ существует конечная $\varepsilon$-сеть.
\end{corollary}

\begin{proposition}
    Множество $A \subset X$ предкомпактно тогда и только тогда, когда для любого
    $\varepsilon > 0$ у $A$ существует предкомпактная
    $\varepsilon$-сеть.
\end{proposition}
\begin{proof}
\mbox{}
    \begin{enumerate}
        \item ($\Rightarrow$) Очевидно, так как конечное множество всегда компактно.
        \item ($\Leftarrow$) Возьмём предкомпактную $\varepsilon$-сеть $M_\varepsilon$ 
        множества $A$. Тогда существует $N_\varepsilon$ — предкомпактная $\varepsilon$-сеть
        для $M_\varepsilon$, откуда $N_\varepsilon$ — $2\varepsilon$-сеть для $A$.
    \end{enumerate}
\end{proof}

\begin{theorem}[Арцела-Асколли]
    Пусть $\Omega \subset \real^n$ — ограниченная область. Рассмотрим пространство
    $X = C(\ol \Omega)$ и множество $A \subset C(\ol \Omega)$.
    $A$ предкомпактно тогда и тогда, когда выполняются оба условия:
    \begin{enumerate}
        \item $A$ ограничено (возможно, равномерно ограничено), то есть существует $C$
        такое, что $\forall x\in A$ $\max\limits_{t \in \ol \Omega}|x(t)| \leqslant C$
        \item $A$ равностепенно непрерывно.
    \end{enumerate}
\end{theorem}

Вспомним, что равномерная непрерывность функции $x$ означает:
$$
\forall \varepsilon > 0\quad \exists\delta > 0:\quad \forall s,t \in \ol \Omega\quad
|s - t| < \delta \implies |x(s) - x(t)| < \varepsilon
$$
Равностепенная непрерывность:
$$
\forall \varepsilon > 0\quad \exists\delta > 0:\quad \forall x\in A\,
\forall s,t \in \ol \Omega\quad |s-t|< \delta \implies |x(s) - x(t)| < \varepsilon
$$

Таким образом, по $\varepsilon$ строится $\delta$, общее для всех $x \in A$.

\begin{proof}[Доказательство теоремы]
    Мы докажем только необходимость. Возьмём $\frac{\varepsilon}{3}$-сеть $M\subset A$:
    $x_1, x_2, \ldots, x_k$. Мы можем построить $\delta_1, \delta_2,\ldots,\delta_k$ и взять
    $\delta = \min\{\delta_1, \delta_2,\ldots,\delta_k\}$.
    $\forall x\in A$ $\exists j$: $\|x_j - x\| < \frac{\varepsilon}{3}$
    $\forall s,t\in\ol \Omega:$ $|s-t|<\delta$. Получаем:
    $$
    |x(s) - x(t)| \leqslant |x(s) - x_j(s)| + |x_j(s) - x_j(t)| + |x_j(t) - x(t)| <
    \frac{\varepsilon}{3} + \frac{\varepsilon}{3} + \frac{\varepsilon}{3} = \varepsilon
    $$
\end{proof}

\subsection{Компактные операторы}

\begin{definition}
    Оператор $U \in B(X, Y)$ называется компактным, если образ любого ограниченного
    $M \subset X$ является предкомпактным в $Y$.
\end{definition}

\begin{proposition}[Свойства компактных операторов]
\mbox{}
    \begin{enumerate}
        \item $U$ компактен $\iff$ $U(B_X)$ предкомпактен.
        ($B_X = \{x \in X: \|x\| \leqslant 1\}$)
        \item $U$ компактен $\iff$ образ любой ограниченной последовательность
        $\{x_k\}\subset X$ имеет $n_k$ такие, что $U(x_{n_k})$ сходится в $Y$.
        \item Если $U_1, U_2 \in B(X, Y)$ компактны, то $U_1 + U_2$ и $\lambda U$ компактны.
        \item \label{compact_product}
        Рассмотрим $U \in B(X, Y)$, $B(Y, Z)$. Если $U$ компактен, что $VU$ компактен.
        Если $V$ компактен, то $VU$ компактен.
        \item Тождественный оператор $I \in B(X, X)$ компактен тогда и только тогда, 
        когда $\dim X < \infty$.
        \item Если $U \in (X, Y)$ конечного ранга, то есть $\dim U(X) < \infty$, то $U$
        компактен.
        \item Если $U_n, U \in B(X, Y)$, $U_n \to U$, $U_n$ — компактны, то $U$ тоже 
        компактен.
        \item Если $U \in B(H, H)$ — компактный оператор в гильбертовом пространстве $H$, то
        $U^\ast$ — компактен.
        (Более общий случай этого утверждения называется теоремой Шаудера).
    \end{enumerate}
\end{proposition}
\begin{proof}
\mbox{}
    \begin{enumerate}
        \item ($\Rightarrow$) Очевидно.
        
        ($\Leftarrow$) Так как $M$ ограничено, то существует $R$ такое, что
        $M \subset B_R = RB_X$. $U(M) \subset~RU(B_X)$.
        Но $U(B_X)$ — $\frac{\varepsilon}{R}$-сеть.
        
        \item ($\Rightarrow$) $M = \{x_k\}$ — ограниченное множество, отсюда
        $U(M) = \{Ux_k\}$ предкомпактно в $Y$, значит, из последовательности
        $y_k = Ux_k \in U(M)$ можно выделить сходящуюся подпоследовательность.
        
        ($\Leftarrow$) Возьмём $y_k \in U(M)$. Существует $x_k \in M$ такое, что
        $y_k = Ux_k$. Значит, $\{x_k\}$ — ограниченная последовательность в $X$, и
        существует $n_k$ такое, что $Ux_{n_k}$ сходится. Тогда $y_{n_k} = Ux_{n_k}$.
        
        \item Упражнение.
        \item Докажем для компактного $U$. $VU \in B(X, Z)$. Возьмём ограниченную 
        последовательность $\{x_k\} \subset X$.
        Подействуем на неё оператором $U$: $\{Ux_k\}\subset Y$. $\exists n_k$: $Ux_{n_k}$
        сходится в $Y$. Значит, $VUx_{n_k}$ тоже сходится.
        Теперь рассмотрим случай компактного $V$. Аналогично возьмём ограниченную 
        последовательность $\{x_k\} \subset X$. $\{Ux_k\}$ ограничена в $Y$, значит,
        $\exists n_k$ такие, что $V(Ux_{n_k})$ сходится в $Z$.
        
        \item Докажем для случая гильбертова пространства. Достаточность очевидна из анализа.
        Пусть $H$ — бесконечномерное гильбертово пространство. Существуют $e_1, e_2, \ldots$
        такие, что $(e_j, e_k) = \delta_{j,k}$. Последовательность $\{e_j\}$ ограничена, но
        извлечь из неё сходящуюся подпоследовательность нельзя, так как всегда
        $\|e_j - e_k\| = \sqrt 2$, из чего следует, что единичный шар $B_H$ не предкомпактен 
        и из первого свойства оператор $U$ некомпактен.
        
        \item Очевидно.
        
        \item Известно, что множества $U_n(Bx)$ предкомпактны. Мы хотим доказать, что
        $U(B_x)$ предкомпактно, то есть, для любого $\varepsilon > 0$ существует
        предкомпактная $\varepsilon$-сеть для $U(B_x)$, то есть существует $n$ такое, что
        $\|U_n - U\| < \varepsilon$ или, что то же самое,
        $\forall x \in X$ $\|Ux - U_nx\| < \varepsilon\|x\| \leqslant \varepsilon$.
        Тогда $U_n(B_X)$ — $\varepsilon$-сеть для $U(B_X)$.
        
        \item Возьмём $\{x_n\}$ — ограниченную последовательность. Мы хотим выделить такую
        последовательность номеров $n_k$, что $U_{n_k}^\ast$ сходится.
        $$
        \|U^\ast x\|^2 = (U^\ast x, U^\ast x) =
        (UU^\ast x, x) \leqslant \|UU^\ast x\|\cdot \|x\|        
        $$
        Возьмём произвольные $m, n$. Имеет место оценка:
        $$
        \|U^\ast x_n - U^\ast x_m\|^2 \leqslant
        \|UU^\ast x_n - UU^\ast x_m\| \cdot \|x_n - x_m\|
        $$
        По свойству \ref{compact_product} $UU^\ast$ — компактный оператор, то есть,
        существует $n_k$ такое, что $UU^\ast x_n$ сходится, что завершает доказательство.
    \end{enumerate}
\end{proof}

\subsection{Примеры компактных операторов}

\begin{enumerate}
    \item Рассмотрим пространства $X = C^1[0, 1]$ и $Y = C[0, 1]$ с оператором вложения
    $j: C^1[0, 1] \to C[0, 1]$, $jx = x$. Этот оператор будет компактным.
    Вспомним нормы в рассматриваемых пространствах:
    $$
    \|x\|_{C^1} = \max |x'| + max |x|;\quad \|x\|_C = \max |x|
    $$
    Предкомпактно ли $j(B_X)$? Воспользуемся теоремой Арцела-Асколли, показав равностепенную 
    непрерывность:
    $$
    \forall \varepsilon > 0\quad \exists \delta > 0 : \forall x \in j(B_X)\quad
    \forall s,t \in [0,1] : |s - t| < \delta\quad |x(s) - x(t)| < \varepsilon
    $$
    Здесь $x \in j(B_X)$ влечёт $\max |x'| \leqslant 1$. Взяв $\delta = \varepsilon$, получим
    $|s - t| < \varepsilon$, откуда $|x(s) - x(t)| = |x'(\xi)||s-t| \leqslant |s-t|$.
    
    \item Аналогично, пусть $X = C^a[0,1]$, $Y = L^2(0, 1)$ и $j' : C^1[0,1] \to L^2(0,1)$,
    $j'x = x$. Оператор $j$ также компактен.
    
    \item Интегральный оператор с непрерывным ядром.
    Пусть $\Omega \subset \real^m$, $\ol\Omega$ — компакт. Рассматриваем простанства
    $C(\ol\Omega)$, $L^2(\Omega)$. Пусть $K \in C(\ol \Omega \times \ol \Omega)$ — ядро. Рассмотрим
    интегральный оператор
    $(Ux)(t) = \int\limits_\Omega K(s, t) x(s) \dif s$. Проверим компактность оператора $U$
    при действии его из $L^2(\Omega)$ в $C(\ol \Omega)$. Компактность при действии
    $C(\ol \Omega) \to C(\ol \Omega)$ и $L^2(\Omega) \to L^2(\Omega)$ будет следствием.
    Воспользуемся теоремой Арцела-Асколли. Пусть $B_X$ — шар в $L^2(\Omega)$.
    $U_(B_X)$ предкомпактен в $C(\ol \Omega)$.
    $$
    |Ux(t_1) - Ux(t_2)| \leqslant \int\limits_\Omega |K(s, t_1) - K(s, t_2)||x(s)|\dif s \leqslant
    \underbrace{\|x\|_2}_{\leqslant 1} \cdot
    \bigg(\int\limits_\Omega |K(s, t_1) - K(s, t_2)|^2 \dif s\bigg)^{\frac{1}{2}}
    $$
    $$
    \forall \varepsilon > 0 \quad \exists \delta: |t_1 - t_2| < \delta \implies
    \bigg(\int\limits_\Omega |K(s, t_1) - K(s, t_2)|^2 \dif s\bigg)^{\frac{1}{2}} < \varepsilon
    $$
    
    Другой способ доказательства заключается в приближении $U$ конечномерными операторами.
    Существует многочлен $P$ такой, что $\|P - K\|_{C(\ol \Omega \times \ol \Omega)} < \varepsilon$
    Рассмотрим конечномерный оператор $U_px(t) = \int\limits_\Omega P(s, t) x(s) \dif s$ с 
    вырожденным ядром $P(s, t) = =\sum\limits_{k=1}^N a_k(s)b_k(t)$.
    Тогда будет иметь место $\|U - U_p\| \leqslant C_\varepsilon$.
    
    \item Интегральный оператор с ядром со слабой особенностью.
    Пусть ядро $K(s, t) = \frac{A(s, t)}{|s - t|^\alpha}$, где
    $A \in C(\ol \Omega \times \ol \Omega)$ и $\alpha < m$ — размерности пространства $\real^m$.
    Докажем, что операторы $U\in B(L^2(\Omega), L^2(\Omega)$ и
    $U \in B(C(\ol \Omega), C(\ol \Omega))$ компактны. Рассмотрим $K_n$:
    $$
    K_n(s,t) =
    \begin{cases}
        K(s, t),\text{ если } |s - t| \geqslant \frac{1}{n}\\
        \frac{A(s, t)}{(\frac{1}{n})\alpha},\text{ если } |s-t| \leqslant \frac{1}{n}\\
    \end{cases}
    $$
    $K_n$ — непрерывные ядра, и соответствующие им операторы $U_n$ будут компактными.
    $$
    |Ux(t) - U_nx(t)| = \bigg|\int\limits_\Omega (K(s, t) - K_n(s, t))x(s) \dif s\bigg| =
    \bigg|\int\limits_{\Omega \cap B_{\frac{1}{n}}(t)}\bigg| = \ast
    $$
    
    Для случая $C(\ol \Omega) \to C(\ol \Omega)$:
    $$
    \ast \leqslant \|x\|\cdot \int\limits_{\Omega \cap B_{\frac{1}{n}}(t)} 2|K(s, t)|\dif s
    \leqslant 2\|x\| \cdot \|A\|_{C(\ol \Omega \times \ol \Omega)} \cdot
    \int\limits_{B_{\frac{1}{n}}(t)} \frac{1}{|s-t|^\alpha}\dif s \leqslant
    $$
    $$
    \leqslant
    2\|x\|\cdot \|A\| \cdot \frac{(\frac{1}{n})^{m-a}}{m-a}\cdot |\Omega|
    $$
    
    Для случая $L^2(\Omega) \to L^2(\Omega)$:
    $$
    \ast \leqslant \int\limits_{B_{\frac{1}{n}}(t)}|K(s, t) - K_n(s, t)||x(s)|\dif s \leqslant
    2\int\limits_{B_{\frac{1}{n}}(t)} (K(s, t)||x(s)| \dif s \leqslant
    $$
    $$
    \leqslant
    2\int\limits_{B_{\frac{1}{n}}(t)}|K(s,t)|^\frac{1}{2}|K(s,t)|^\frac{1}{2} |x(s)| \dif s 
    \leqslant
    2\bigg(\underbrace{\int\limits_{B_{\frac{1}{n}}(t)} |K(s, t)|\dif s}_{\leqslant
    C\big(\frac{1}{n}\big)^{m-\alpha}}\bigg)^\frac{1}{2}
    \bigg(\int\limits_{B_{\frac{1}{n}}(t)}|K(s, t)| \cdot |x(s)|^2 \dif s\bigg)^\frac{1}{2}
    $$
    $$
    \|U_nx - Ux\|^2 \leqslant C\Big(\frac{1}{n}\Big)^{m - \alpha}
    \int\limits_\Omega \int\limits_{B_{\frac{1}{n}}(t)}
    |K(s, t)||x(s)|^2 \dif s \dif t =
    $$
    $$
    =
    C\Big(\frac{1}{n}\Big)^{m - \alpha}
    \int\limits_\Omega\int\limits_\Omega \chi_{B_{\frac{1}{n}}(t)}(s)K(s, t) |x(s)|^2 \dif s \dif t
    = \ldots
    $$
\end{enumerate}

%Далее бумажный конспект%

\subsection{Собственные числа и собственные векторы компактных самосопряжённых операторов}

\begin{definition}
    Пусть $U: X \to X$ — оператор, $U \in B(X, X)$. $\lambda$ называется собственным
    числом оператора $U$, если существует такое $x \neq 0$ (называемое
    собственным вектором), что $Ux = \lambda x$.
\end{definition}
\begin{proposition}
\mbox{}
    \begin{enumerate}
        \item $|\lambda| < \|U\|$;
        \item Если $U = U^\ast$, то $\lambda \in \real$ (для гильбертова пространства).
        \item Если $U = U^\ast$, $Ux = \lambda x$, $Uy = \mu y$ и $\lambda \neq \mu$,
        то $x \perp y$.
    \end{enumerate}
\end{proposition}

\begin{lemma}
    Если $U \in B(H, H)$, где $H$ — гильбертово пространство, и $U = U^\ast$, то
    $\|U\| = \sup\limits_{\|x\| = 1}|(Ux, x)|$
\end{lemma}
\begin{proof}
Пусть $\sup\limits_{\|x\| = 1}|(Ux, x)| = A$.
    \begin{enumerate}
        \item $|(Ux, x)| \leqslant \|Ux\|\cdot\|x\| \leqslant \|U\| \cdot \|x\|^2 \implies
        A \leqslant \|U\|$.
        \item $U = U^\ast \implies (Ux, x) \in \real\quad \forall x$, так как
        $(Ux, x) = (x, Ux) = \ol{(Ux, x)}$
        \item $\forall x \in H$ $|(Ux, x)| \leqslant
        A\|x\|^2 \iff |(U(\frac{x}{\|x\|}, \frac{x}{\|x\|})| \leqslant A$
        \item Возьмём $x, y \in H$.
        $$
        (U(x+y), x+y) = (Ux, x) + \underbrace{(Ux, y) + (Uy, x)}_{2\Ree (Ux, y)} + (Uy, y)
        $$
        $$
        (U(x-y), x-y) = (Ux, x) - (Ux, y) - (Uy, x) + (Uy, y)
        $$
        $$
        |4\Ree (Ux, y)| = |(U(x+y), x+y) - (U(x-y), x-y)| \leqslant
        A(\|x + y\|^2 + \|x-y\|^2) = 2A(\|x\|^2 + \|y\|^2)
        $$
        Положим теперь $y = t\cdot Ux$.
        $$
        4t\|Ux\|^2 \leqslant 2A(\|x\|^2 + t^2\|Ux\|^2)
        $$
        $$
        (4t - 2At^2)\|Ux\|^2 \leqslant 2A\|x\|^2
        $$
        Взяв $t = \frac{1}{A}$, получим:
        $$
        \bigg(\frac{4}{A} - \frac{2}{A}\bigg)\|Ux\|^2 \leqslant 2A\|x\|^2 \implies
        \|Ux\|^2 \leqslant A^2 \|x\|^2 \implies \|U\| \leqslant A
        $$
    \end{enumerate}
\end{proof}

\begin{theorem}
    Если $U \in B(H, H)$ — компактный самосопряжённый оператор на гильбертовом пространстве $H$, то
    у него существует собственное число. Более того, это собственное число равно $\|U\|$ или
    $-\|U\|$.
\end{theorem}
\begin{proof}
    Воспользуемся самосопряжённостью. $\|U\| = \sup\limits_{\|x\|=1}|(Ux, x)|$. Это значит,
    что существует последовательность на единичной сфере $\{x_k\}$, $\|x_k\|=1$ такая, что
    $(Ux_k, x_k) \to \mu$, $\mu = \pm \|U\|$. Докажем, что она сходится.
    $$
    0 \leqslant \|Ux_k - \mu x_k\|^2 =
    \underbrace{(Ux_k, Ux_k)}_{\|Ux_k\|^2 \leqslant \|U\|^2 \cdot \|x_k\|^2 = \mu^2} -
    (\underbrace{(Ux_k, \mu x_k) + (\mu x_k, Ux_k)}_{2\mu(Ux_k, x_k) \to 2 \mu^2})
    + |\mu|^2\underbrace{\|x_k\|^2}_{1} \to 0
    $$
    Мы получили, что $Ux_k - \mu x_k = \eta_k \to 0$. Самое время воспользоваться компактностью.
    Выделим подпоследовательность $n_k$ такую, что $Ux_{n_k}$ сходится. Получим, что
    $\mu x_{n_k}$ тоже сходится. Перейдём к пределу в $Ux_{n_k} - \mu x_{n_k} = \eta_{n_k}$:
    $Ux_0 - \mu x_0 = 0$. Заметим, что $x_0 \neq 0$, потому что $\|x_0\| = 1$. Получили
    $\mu = \pm \|U\|$, что и требовалось.
\end{proof}

\begin{remark}\label{eigennumber_sequence}
    Пусть $U$ — компактный самосопряжённый оператор на гильбертовом пространстве и $\lambda_1$ — 
    его собственное число: $|\lambda_1| = \|U\|$, $e_1$ — соответствующий ему собственный вектор:
    $\|e_1\| = 1$. $Ue_1 = \lambda_1 e_1 \in \Lin\{e_1\}$. Возьмём $x \perp \Lin\{e_1\}$. Тогда
    $Ux \perp \Lin\{e_1\}$, так как:
    $$
    (Ux, e_1) = (x, Ue_1) = \lambda_1(x, e_1) = 0
    $$
    Пусть $H_1 = \{e_1\}^\perp$. $U(H_1) \subset U_1$ и $U_1 = U|_{H_1}$ — компактный 
    самосопряжённый оператор на $H_1$, у которого существует собственное число
    $\lambda_2$:
    $$
    |\lambda_2| = \|U_1\| = \sup\limits_{x \in H_1, \|x\|=1} |(Ux, x)| =
    \sup\limits_{x \perp e_1, \|x\|=1} |(Ux, x)|
    $$
    Пусть $e_2$ — собственный вектор, соотвествующий $\lambda_2$, $H_2 = \{e_1, e_2\}^\perp$,
    $U(H_2) \subset H_2$, $U|_{H_1} = U_2$ — компактный самосопряжённый оператор на $U_2$, у 
    которого существует собственное число $\lambda_3$: $|\lambda_3| = \|U_2\| =
    \sup\limits_{x\perp\{e_1,e_2\}, \|x\|=1} |(Ux, x)|$, и пусть $e_3$ — собственный вектор, ему
    соответствующий. Мы получили последовательность собственных чисел $\lambda_i$ и собственных
    векторов $e_i$
\end{remark}

\begin{theorem}
    Пусть $U$ — компактный самосопряжённый оператор на гильбертовом пространстве. Тогда
    $\forall \varepsilon > 0$ множества вида $[-\|U\|, -\varepsilon] \cup [\varepsilon, \|U\|]$
    содержат лишь конечное количество собственных чисел.
\end{theorem}
\begin{proof}
    Если собственных чисел в таком множестве бесконечно много, то существует последовательность
    $\lambda_k \to \lambda \in [-\|U\|, - \varepsilon] \cup [\varepsilon, \|U\|$ ($\lambda_k$ —
    различные собственные числа). Пусть $e_k$ — соответствующие нормированные собственные векторы, 
    попарно ортогональные. Выделим подпоследовательность $n_k$ такую, что $Ue_{n_k}$ сходится.
    $\frac{1}{\lambda_{n_k}}Ue_{n_k} = e_{n_k}$, отсюда $e_{n_k}$ сходится, что невозможно для
    ортонормированных векторов.
\end{proof}

\begin{theorem}
    Пусть $U$ — компактный самосопряжённый оператор на гильбертовом пространстве. Тогда
    все ненулевые собственные числа имеют конечную кратность.
\end{theorem}

Вспомним, что кратность собственного числа есть размерность собственного подпространства
$H_\lambda = \Ker (U - \lambda I) = \{x | Ux = \lambda x\}$.

\begin{proof}[Доказательство теоремы]
    Пусть $\lambda \neq 0$ — собственное число $U$.
    $\Ker(U-\lambda I) = \Ker \bigg(I -~\frac{1}{\lambda}U\bigg)$ — имеет конечную размерность
    по теореме Фредгольма.
\end{proof}

\begin{remark}
    Процедура из замечания \ref{eigennumber_sequence} собирает все ненулевые собственные числа.
    Возможны два варианта:
    \begin{enumerate}
        \item $\lambda_j \neq 0$, $\lambda_j \to 0$.
        \item Начиная с некоторого $j$ все собственные числа будут нулевыми.
    \end{enumerate}
\end{remark}

\begin{theorem}[Гильберта-Шмидта]
    Пусть $H$ — гильбертово пространство, $U$ — компактный оператор на нём. Рассмотрим 
    упорядоченную последовательность собственных чисел $\{\lambda_j\}$:
    $|\lambda_1| \geqslant |\lambda_2| \geqslant \ldots \geqslant |\lambda_k| \to 0$;
    собственные векторы
    $e_1, e_2, \ldots$, $(e_j, e_k) = \delta_{j,k}$. Любой $x\in H$ можно
    разложить в ряд Фурье:
    $$
    x = z + \sum\limits_{j=1}^\infty (x, e_j)e_j
    $$
    При этом $z \in \Ker U$
\end{theorem}
\begin{proof}
    Будем пользоваться обозначениями из замечания \ref{eigennumber_sequence}.
    Рассмотрим частичную сумму $x_n = \sum\limits_{j=1}^n (x, e_j)e_j$. Заметим, что
    $x - x_n \in H_n$, то есть $x - x_n \perp e_1, \ldots, e_n$. Это означает, что
    $\|U(x - x_n)\| \leqslant \|U_n\| \cdot \|x - x_n\| = |\lambda_{n+1}| \cdot \|x - x_n\|
    \leqslant |\lambda_{n+1}|\cdot \|x\| \to 0$.
    $$
    x = z + \lim\limits_{n \to \infty} x_n
    $$
    $$
    Ux = Uz + \lim\limits_{n \to \infty} Ux_n
    $$
    $$
    Uz = \lim\limits_{n \to \infty} (Ux - Ux_n) = 0
    $$
    Здесь мы воспользовались непрерывностью $U$.
\end{proof}

\end{document}
