\documentclass[11pt,openany,a4paper]{scrartcl}

\usepackage{indentfirst}
\usepackage{amsmath,amsthm,amssymb,amsfonts,amsopn}
\usepackage{mathtext}
\usepackage{enumitem}
\usepackage[T1,T2A]{fontenc}
\usepackage[utf8]{inputenc}
\usepackage[english,russian]{babel}
\usepackage[intlimits]{mathtools}
\usepackage[makeroom]{cancel}
\usepackage{titletoc}
\renewcommand{\bfdefault}{sbc}
\usepackage{ccfonts,eulervm,microtype}
\usepackage{enumitem}
\usepackage{tikz}
\usetikzlibrary{arrows}
\usetikzlibrary{calc}

\tikzset{
    pil/.style={
           ->,
           thick,
           shorten <=2pt,
           shorten >=2pt},
    axis/.style={very thick, ->, >=stealth'},
}

\usepackage[portrait,a4paper,margin=2.5cm,headsep=5mm]{geometry}

\author{Ф. Л. Бахарев \thanks{Конспект подготовлен студентом Яскевичем С. В.}}
\title{Функциональный анализ}

\theoremstyle{plain}
\newtheorem{theorem}{Теорема}[section]
\newtheorem{corollary}[theorem]{Следствие}
\newtheorem{proposition}[theorem]{Предложение}
\newtheorem{lemma}[theorem]{Лемма}
\newtheorem{exercise}[theorem]{Упражнение}

\theoremstyle{definition}
\newtheorem{definition}[theorem]{Определение}
\newtheorem{remark}[theorem]{Замечание}
\newtheorem{example}[theorem]{Пример}
\newtheorem{examples}[theorem]{Примеры}
\newtheorem{num}[theorem]{}

\newcommand\mb{\mathbb}
\newcommand\real{\mb R}
\newcommand{\complex}{\mb C}
\newcommand\eqdef{\mathrel{\stackrel{\makebox[0pt]{\mbox{\normalfont\tiny def}}}{=}}}
\newcommand\lparagraph[1]{\paragraph{#1}\mbox{}\\}
\newcommand{\pd}[2]{\frac{\partial #1}{\partial #2}}
\newcommand{\uto}{\rightrightarrows}
\newcommand{\underto}[1]{\xrightarrow[#1]{}}
\newcommand{\overto}[1]{\xrightarrow{#1}}
\newcommand{\dif}{\, \mathrm d}
\newcommand{\bigslant}[2]{{\raisebox{.2em}{$#1$}\left/\raisebox{-.2em}{$#2$}\right.}}
\DeclareMathOperator{\Ree}{Re}
\DeclareMathOperator{\Img}{Im}
\DeclareMathOperator{\Arg}{Arg}
\DeclareMathOperator{\dist}{dist}
\DeclareMathOperator{\const}{const}
\DeclareMathOperator{\Ln}{Ln}
\DeclareMathOperator{\card}{card}
\DeclareMathOperator{\Ker}{Ker}
\DeclareMathOperator{\sign}{sign}
\DeclareMathOperator{\diam}{diam}

\begin{document}

\maketitle

\tableofcontents

\pagebreak

\section{Линейное нормированное пространство}

\begin{definition}
    Линейное множество $L$ над полем скаляров $\real$ ($\complex$) — множество с
    операциями сложения и умножения на скаляр, удовлетворяющее свойствам:
    \begin{enumerate}
        \item $(x + y) + z = x + (y + z)$ $\forall x,y,z \in L$
        \item $x + y = y + x$ $\forall x,y,z \in L$
        \item Существует элемент $0$ такой, что $x + 0 = x$ $\forall x \in L$
        \item Для любого $x \in L$ существует обратный элемент по сложению $-x$ такой, что
        $-x + x = 0$
        \item $\lambda(\mu x) = (\lambda \mu) x$ $\forall \lambda, \mu
        \in \real(\complex)$, $x \in L$
        \item $\lambda(x + y) = \lambda x + \lambda y$ $\forall \lambda \in 
        \real(\complex)$, $x,y \in L$
        \item $(\lambda + \mu)x = \lambda x + \mu y$ $\forall \lambda,
        \mu \in \real(\complex)$, $x,y \in L$
    \end{enumerate}
\end{definition}

\begin{definition}
    $\varphi: L \to \real$ называется нормой, если:
    \begin{enumerate}
        \item $\varphi(x + y) \leqslant \varphi(x) + \varphi(y)$
        $\forall x, y \in L$
        \item $\varphi(\lambda x) = |\lambda|\varphi(x)$
        $\forall x \in L$, $\lambda \in \real(\complex)$
        \item $\varphi(x) \geqslant 0$ $\forall x \in L$
        \item $\varphi (x) = 0 \iff x = 0$
    \end{enumerate}

    Если выполнены только первых три свойства, то $\varphi$ называется полунормой.
\end{definition}
\begin{remark}
\mbox{}
    \begin{enumerate}
        \item $\rho (x, y) = \varphi(x - y)$ — метрика.
        \item Если на пространстве задана норма $\|\cdot\|$, то $X = (L, \varphi)$ —
        нормированное пространство.
    \end{enumerate}
\end{remark}

\begin{definition}
    $x_n \to x$ в $X$, если $\|x_n - x\| \to 0$ при $n \to \infty$, то есть $\forall
    \varepsilon > 0$ $\exists N$:~$\forall n > N$ $\|x_n - x\| < \varepsilon$
\end{definition}

\begin{definition}
    $\{x_n\} \subset X$ — фундаментальная последовательность (сходящаяся в себе,
    последовательность Коши), если $\|x_n - x_m\| \underto{m,n \to \infty} 0$, то есть
    $\forall \varepsilon > 0$ $\exists N$: $\forall m,n > N$ $\|x_m - x_m\| < \varepsilon$
\end{definition}

\begin{remark}
    $x_n \to x \implies \{x_n\}$ — фундаментальная. Обратное, вообще говоря, неверно.
\end{remark}
\begin{definition}
    Нормированное пространство $X$ называется полным, если из фундаментальности
    последовательности следует существование предела.
\end{definition}

\begin{definition}
    Пусть $x_n \in X$. $\sum\limits_{j = 1}^\infty x_j$ сходится, если
    $S_n = \sum\limits_{j = 1}^n x_j$ имеет предел $\lim S_n = S$. $S$ называется
    суммой ряда.
\end{definition}

\begin{definition}
    Ряд $\sum\limits_{j = 1}^\infty x_j$ называется \emph{сходящимся абсолютно},
    если $\sum\limits_{j = 1}^\infty \|x_j\|$ сходится.
\end{definition}
\begin{remark}
    Из абсолютной сходимости не следует обычная сходимость.
\end{remark}

$S_n$ сходится $\iff |S_n - S_m| \to 0$. Пусть $C_n = \sum\limits_{j = 1}^n \|x\|$.
$C_n$ сходится $\iff |C_n - C_m|~\to~0$.
Если мы хотим, чтобы сходимость $S_n$ была равносильна
$\|S_n - S_m\| \to 0$, то нам нужна полнота пространства.

\begin{definition}
    Полное линейное нормированное пространство называется банаховым пространством (в честь
    польского математика Стефана Банаха).
\end{definition}
\begin{examples}
\mbox{}
    \begin{itemize}
        \item Евклидово пространство: $\real^n$ с нормой
        $\|x\| = |x| = \sqrt[n]{|x_1|^2 + \ldots + |x_n|^2}$ — то же, что
        $\ell_n^2$ с нормой $\|\cdot \|_2$;
        \item $\ell_n^1 = (\real^n, \|\cdot\|_1)$, где
        $\|x\|_1 = |x_1| + \ldots + |x_n|$;
        \item $\ell_n^\infty = (\real^n, \|\cdot\|_\infty)$,
        где $\|x\|_\infty = \max\limits_{1 \leqslant j \leqslant n} |x_j|$;
        \item $\ell_n^p = (\real^n, \|\cdot\|_p$,
        $\|x\|_p = \bigg(\sum\limits_{j=1}^n |x_j|^p\bigg)^{\frac{1}{p}}$,
        $p \geqslant 1$;
        \item $C(\overline \Omega)$ с нормой
        $\|x\| = \max\limits_{t \in \overline \Omega} |x(t)|$, где
        $\Omega$ — область в $\real^m$. $\overline\Omega$ — замыкание
        $\Omega$. Ясно, что $\overline \Omega$ — компакт в
        $\real^m$.
    \end{itemize}
\end{examples}
\begin{exercise}
    Верно ли, что $\|x\|_p \underto{p \to \infty} \|x\|_\infty$?
\end{exercise}

\begin{theorem}
    Пространство $C(\overline \Omega)$ полно.
\end{theorem}
\begin{proof}
    Рассмотрим фундаментальную последовательность $x_n \in C(\overline \Omega)$.
    $$
    \forall \varepsilon > 0 \exists N: \forall k, n > N \|x_k - x_n\| =
    \max_{t \in \overline \Omega} |x_n(t) - x_k(t)| < \varepsilon
    $$

    Возьмём $t \in \overline \Omega$. $\{x_n(t)\}$ — числовая последовательность.
    Тогда получаем $|x_n(t) - x_k(t)| < \varepsilon$, отсюда $\{x_n(t)\}$ —
    фундаментальна, значит существует $\lim\limits_{n \to \infty} x_n(t) = x(t)$.

    Проверим, что $\max\limits_{t \in \overline \Omega} |x_n(t) - x(t)|
    \underto{n \to \infty} 0$, т. е. $x_n \uto x$ на $\overline \Omega$.
    Заметим, что $\forall k, n > N$ $|x_k(t) - x_n(t)| < \varepsilon \implies
    |x(t) - x_n(t)| \leqslant \varepsilon$.

    Почему же $x$ непрерывна? Потому что равномерный предел непрерывных функций
    непрерывен.

\end{proof}

Пусть $[a, b] \subset \real$.
Рассмотрим пространство дифференцируемых функций $C^1[a, b]$.
Какую норму на нём выбрать?
\begin{itemize}
    \item $\varphi_1(x) = \max\limits_{t \in [a, b]} |x(t)|$;
    \item $\varphi_2(x) = \max\limits_{t \in [a, b]} |x'(t)|$;
    \item $\varphi_3(x) = \varphi_1(x) + \varphi_2(x)$;
    \item $\varphi_4(x) = |x(a)| + \max\limits_{t \in [a, b]} |x'(t)|$.
\end{itemize}

Заметим, что $\varphi_2$ нормой вообще не является, а $\varphi_1$ не даёт полноты
пространства.
\pagebreak
\begin{theorem}
    \begin{enumerate}
        \item Пространство $(C^1 [a, b], \varphi_1)$ не полно.
        \item Пространство $(C^1 [a, b], \varphi_3)$ полно.
    \end{enumerate}
\end{theorem}

\begin{proof}
    Докажем первое утверждение.

    \emph{Первый аргумент}. $x$ — производная непрерывная на $[a, b]$, негладкая.
        По теореме Вейерштрасса для любого $\varepsilon > 0$ существует многочлен
        $P$ такой, что $\max\limits_{[a, b]} |P - x| < \varepsilon$

    \emph{Второй аргумент}. Пусть $[a, b] = [-1, 1]$, $x(t) = |t| \notin C^1[a, b]$,
    $x^\varepsilon(t) = |t|^{1 + \varepsilon} \in C^1[a, b]$. $\max |x(t) - x^\varepsilon(t)| 
    \underto{\varepsilon \to 0} 0$.

    Для доказательства второго утверждения возьмём $x_n \in C^1[a, b]$ —
    последовательность, фундаментальную относительно $\varphi_3$.
    $$
    \varphi_3(x_n -x_k) \underto{n, k \to \infty} 0 \implies
    \begin{cases}
        \varphi_1(x_n - x_k) \to 0\\
        \varphi_2(x_n - x_k) \to 0
    \end{cases}
    \implies \exists x \in C[a, b], y \in C[a, b]
    $$

    $$
    \begin{cases}
        \varphi_1(x_n - x) \to 0 \iff x_n \uto x \text{ на } [a, b]\\
        \varphi_1(x'_n - y) \to 0 \iff x'_n \uto y \text{ на } [a, b]
    \end{cases}
    \implies x \in C^1[a, b], x' = y
    $$

    Отсюда $\varphi_3(x_n - x) \to 0$
\end{proof}

\section{Пространства Лебега}

\lparagraph{Неравенство Гёльдера}

Рассмотрим $(T, \mu)$ — пространство с мерой, $x, y$ — измеримые функции, и числа
$p, q > 0$ — сопряжённые показатели, т. е. $\frac{1}{p} + \frac{1}{q} = 1$. 
Тогда верно неравенство:
$$
\int\limits_T |x(t)y(t)|\dif \mu(t) \leqslant
\bigg(\int\limits_T |x(t)|^p\dif \mu(t)\bigg)^{\frac{1}{p}}
\bigg(\int\limits_T |y(t)|^q\dif \mu(t)\bigg)^\frac{1}{q}
$$

\lparagraph{Неравенство Минковского}

Если $(T, \mu)$ — пространство с мерой, $x, y$ — измеримые функции,
$p \geqslant 1$, то верно неравенство:
$$
\bigg(\int\limits_T |x(t)|^p\dif \mu(t)\bigg)^{\frac{1}{p}} +
\bigg(\int\limits_T |y(t)|^q\dif \mu(t)\bigg)^\frac{1}{q} \geqslant
\int\limits_T |x(t)y(t)|\mathrm d\mu(t)
$$

Обозначение: $\|x\|_p = (\int\limits_T |x|^p)^\frac{1}{p}$.

\begin{remark}
    Частный случай — $p = q = 2$. Тогда неравенство Гёльдера оказывается
    неравенством Коши-Буняковского-Шварца:
    $$
    \int\limits_T |x(t)|\cdot|y(t)|\dif \mu(t) \leqslant
    \bigg(\int\limits_T |x(t)|^2\dif \mu(t)\bigg)^{\frac{1}{2}}
    \bigg(\int\limits_T |y(t)|^2\dif\mu(t)\bigg)^\frac{1}{2}
    $$
\end{remark}
\begin{remark}
    Пусть $T = \mb N$, и если $M \subset \mb N$,
    то $\#M = \card M$ — количество элементов $M$ — будет мерой.
    Рассмотрим функцию $x: \mb N \to k$, где $k$ — некоторое поле скаляров.
    Мы помним, что функция из натуральных чисел называется последовательностью.
    Как можно вычислять $\int\limits_{\mb N}x(n)\mathrm d\#(n)$? Ясно,
    что такой интеграл — это ряд $\sum\limits_{n \in \mb N} x(n)$, а суммируемые 
    функции в этом случае будут абсолютно сходящимися рядами.
    Неравенство Гёльдера будет выглядеть так:
    $$
    \sum_{n \in \mb N} |x_n||y_n| \leqslant
    \bigg(\sum_{n \in \mb N} |x_n|^p\bigg)^\frac{1}{p}
    \bigg(\sum_{n \in \mb N} |y_n|^p\bigg)^\frac{1}{p}
    $$
    А неравенство Минковского — так:
    $$
    \bigg(\sum_{n \in \mb N} |x_n|^p\bigg)^\frac{1}{p} +
    \bigg(\sum_{n \in \mb N} |y_n|^p\bigg)^\frac{1}{p} \geqslant
    \bigg(\sum_{n \in \mb N} |x_n||y_n|\bigg)^\frac{1}{p}
    $$
\end{remark}

\begin{definition}
    Пространство Лебега $\mathcal{L}^p(T, \mu)$ — это множество
    $\{x\, \big| \int\limits_T |x|^p\dif\mu <
    \infty\}$.
    Оно линейно: $x, y \in \mathcal{L}^p \implies
    x + y \in \mathcal{L}^p$ и $\lambda y \in \mathcal{L}^p$
\end{definition}

Заметим, что $\|x\|_p = \bigg(\int\limits_T |x|^p\mathrm d\mu\bigg)^\frac{1}{p}$ — полунорма
на $\mathcal{L}^p(T, \mu)$. Если $\|x\|_p = 0$, то $x = 0$ почти везде.

Чтобы получить норму, введём следующее отношение эквивалентности:
$$
x_1 \sim x_2 \text{ если } x_1 - x_2 = 0 \text{ почти везде.}
$$
Тогда
$$
\bigslant{\mathcal{L}^p(T, \mu)}{\sim} = L^p(T, \mu)
$$
— это настоящее пространство Лебега. В дальнейшем мы будем считать функции, 
отличающиеся на множестве меры нуль, одинаковыми.

\begin{remark}
    Пусть $T \subset \real^n$, $\mu = \lambda$ — мера Лебега. Тогда будем 
    обозначать $L^p(T, \mu) = L^p(T)$.
\end{remark}

\begin{theorem}
    Пространство $L^p(T, \mu)$ полно при $p \geqslant 1$.
\end{theorem}

\begin{example}
    Рассмотрим $L^2(0, +\infty)$ и $L^1(0, +\infty)$. Какое из этих пространств 
    является вложением в другое? Возьмём функцию $x(t) = \frac{1}{t + 1}$.
    $$
    \int\limits_0^\infty \frac{1}{t + 1}\mathrm dt = \infty
    $$
    $$
    \int\limits_0^\infty \frac{1}{(t+1)^2} \mathrm dt < \infty
    $$
    Отсюда видно, что $L^2(0, +\infty) \not\subset L^1(0, +\infty)$. Легко 
    придумать и пример, доказывающий отсутствие включения в обратную сторону.
\end{example}

\begin{theorem}[О вложенности пространств $L^p$]
Пусть $1 \leqslant p_1 < p_2 \leqslant \infty$. Тогда:
    \begin{enumerate}
        \item $\ell^{p_1} \subset \ell^{p_2}$.
        \item Если $(T, \mu)$ — пространство с мерой, $\mu(T) < \infty$, то
        $L^{p_1}(T, \mu) \supset L^{p_2}(T, \mu)$
    \end{enumerate}
\end{theorem}
\pagebreak
\begin{proof}
\mbox{}
    \begin{enumerate}
        \item Пусть $x = (x_1, x_2, x_3, \ldots)$. Хотим проверить, что
        $x \in \ell^{p_1} \implies x \in \ell^{p_2}$.
        $$
        \sum\limits_{j = 1}^\infty |x_j|^{p_1} < \infty \implies
        \exists N\quad \forall j > N\quad |x_j| < 1 \implies
        |x_j|^{p_1} < |x_j|^{p_2}
        $$
        $$
        \sum\limits_{j = N + 1}^\infty |x_j|^{p_1} >
        \sum\limits_{j = N + 1}^\infty |x_j|^{p_2} \implies
        \sum\limits_{j = 1}^\infty |x_j|^{p_2} < \infty \implies x \in \ell^{p_2}
        $$
        \item Для доказательства второго пункта достаточно применить неравенство 
        Гёльдера.
    \end{enumerate}
\end{proof}

\section{Непрерывность. Сжимающее отображение}

\begin{definition}
    Возьмём отображение $F: X \to Y$, где $X$ и $Y$ — линейные нормированные
    пространства. $F$ называется непрерывным в точке $x_0$, если:
    $$
    \forall \varepsilon > 0\quad
    \exists \delta > 0:\quad \forall x: \|x - x_0\| < \delta\quad
    \|F(x) -F(x_0)\| < \varepsilon
    $$
    $F$ называется непрерывным, если оно непрерывно во всех точках $X$.
\end{definition}

\begin{example}
        $X = Y = C[0, 1]$, $\|x\|_{C[0, 1]} =
        \max\limits_{t \in [0, 1]} |x(t)|$. Рассмотрим отображение
        $(F(x))(t) = \int\limits_0^t x(s)\dif s$ и докажем, что оно 
        непрерывно.
        $$
        \|F(x_1) - F(x_2)\| = \max_{t \in [0, 1]}
        \bigg|\int\limits_0^t x_1(s)\dif s - \int\limits_0^t x_2(s)\dif s\bigg|
        \leqslant
        $$
        $$
        \leqslant
        \max_{t \in [0, 1]} \int\limits_0^t |x_1(s) - x_2(s)|\dif s \leqslant
        \max_{t \in [0, 1]} t \cdot \|x_1 - x_2\| = \|x_1 - x_2\|
        $$
        Достаточно взять $\delta = \varepsilon$ и всё доказано.
\end{example}

\begin{definition}
    Отображение $F: X \to Y$ называется липшицевым, если существует такое $C$, что
    для всех $x_1, x_2 \in X$ выполнено $\|F(x_1) - F(X_2)\| \leqslant
    C\cdot\|x_1 - x_2\|$
\end{definition}

Заметим, что из липшицевости отображения следует его непрерывность. Достаточно 
взять $\delta = \frac{\varepsilon}{C}$.

\begin{definition}
    Отображение $F: X \to Y$ называется сжимающим, если существует такое
    $\gamma < 1$, что $\forall x_1, x_2 \in X$ выполнено $\|F(x_1) - F(x_2)\|
    \leqslant
    \gamma \|x_1 - x_2\|$.
\end{definition}

\begin{theorem}[Банаха о неподвижной точке]
    Если пространство $X$ — полное, а отображение $F$ — сжимающее, то существует
    единственный элемент $x_\ast \in X$ такой, что $F(x_\ast) = x_\ast$. Этот
    элемент называется неподвижной точкой.
\end{theorem}
\begin{proof}
    Докажем существование. Возьмём \emph{траекторию} точки $x_1$:
    $$
    x_1, \underbrace{F(x_1)}_{x_2}, \underbrace{F(F(x_1))}_{x_3}, \ldots,
    \text{ т. е. } x_{n+1} = F(x_n)
    $$
    $$
    \|x_{n+1} - x_n\| = \|F(x_n) - F(x_{n-1})\| \leqslant \gamma \|x_n - x_{n-1}\|
    \leqslant \gamma^2\|x_{n-1} - x_{n-2}\| \leqslant \ldots \leqslant
    \gamma^{n+1}\underbrace{\|x_2 - x_1\|}_{\alpha}
    $$
    Таким образом, при $m > n$:
    $$
    \|x_m - x_n\| \leqslant \|x_m - x_{m-1}\| + \|x_{m-1} - x_{m-2}\| + \ldots
    + \|x_{n+1} - x_n\| \leqslant \alpha\gamma^{m-2} + \alpha\gamma^{m-3} +\ldots +
    $$
    $$
    + \alpha\gamma^{n-1} \leqslant \sum\limits_{j = n-1}^\infty \alpha\gamma^j =
    \alpha\gamma^{n-1}\frac{1}{1-\gamma} \underto{n \to \infty} 0
    $$
    Отсюда получаем, что $\{x_n\}$ фундаментальна, а значит существует
    $\lim\limits_{n \to \infty} x_n$. Обозначим его за $x_\ast$. Ясно, что это
    и будет неподвижная точка.
    
    Докажем единственность. Пусть $x_\ast$ и $x^\ast$ — две неподвижные точки.
    Тогда:
    $$
    \underbrace{\|F(x_\ast) - F(x^\ast)\|}_{\leqslant \gamma\|x_\ast - x^\ast\|} =
    \|x_\ast - x^\ast\|
    $$
    Отсюда $\|x_\ast - x^\ast\| = 0$, что и требовалось.
\end{proof}

\begin{theorem}
    Пусть пространство $X$ — полное, $F: X \to X$ и существует $n$ такое, что
    $F^n$ — сжимающее. Тогда существует единственная точка $x_\ast$ такая, что
    $F(x_\ast) = x_\ast$.
\end{theorem}
\begin{proof}
    Если $F^n$ сжимающее, то существует (и единственна) неподвижная точка:
    $F^n(x_\ast) = x_\ast$.
    Условие теоремы подразумевает, что если $F$ переводит точку $x_\ast$ в
    некоторую точку $x_1$, которую, в свою очередь, переводит в $x_2$, то через
    $n$ итераций точка $x_{n-1}$ снова переходит в $x_\ast$. Отсюда следует,
    что точки $x_1,\ldots,x_{n-1}$ — тоже неподвижные точки $F^n$. Но по теореме
    Банаха такая точка у $F^n$ только одна, следовательно,
    $x_\ast = x_1 = x_2 = \ldots = x_{n-1}$.
\end{proof}
\begin{example}[Интегральное уравнение Фредгольма I рода]
    Пусть нам даны функции $K(s,t)$ и $a(t)$. Мы хотим найти функцию $x(t)$,
    удовлетворяющую уравнению:
    $$
    x(t) = a(t) + \int\limits_{s_1}^{s_2} K(s,t) x(s) \dif s
    $$
    Будем рассматривать частный случай, в котором $K \in C([0,1]\times[0,1])$,
    $a \in C[0,1]$. Задача — найти $x \in C[0,1]$ такое, что
    $$
    x(t) = a(t) + \int\limits_0^t K(s,t) x(s) \dif s
    $$
\end{example}
\begin{proposition}
    Это уравнение имеет единственное решение.
\end{proposition}
\begin{proof}
    Рассмотрим отображение $F: C[0, 1] \to C[0, 1]$.
    $$
    (F(x))(t) = a(t) + \int\limits_0^t K(s, t)x(s)\dif s
    $$
    Заметим, что оно, вообще говоря, не является сжимающим.
    Рассмотрим также $(F_0(x))(t) = \int\limits_0^t K(s, t)x(s)\dif s$.
    
    Обратим внимание на несколько важных свойств:
    \begin{itemize}
        \item $F_0(x) - F_0(y) = F_0(x - y)$
        \item $F(x) - F(y) = F_0(x) - F_0(y)$
        \item $F^n(x) - F^n(y) = F(F^{n-1}(x) - F^{n-1}(y)) =
        F_0(F^{n-1}(x)) - F_0(F^{n-1}(y)) = F_0(F^{n-1}(x) - F^{n-1}(y)) =
        F_0^n(x-y)$
    \end{itemize}
    $$
    (F_0(x - y))(t) = \int\limits_0^t K(s_1, t)(x(s_1) - y(s_1))\dif s_1
    $$
    $$
    (F_0^2(x-y))(t) = \int\limits_0^t K(s_2, t)
    \int\limits_0^{s_2} K(s_1, s_2)(x(s_1) - y(s_1))\dif s_1\dif s_2
    $$
    $$
    \cdots
    $$
    $$
    (F_0^n(x-y))(t) = \int\limits_0^t K(s_n, t)
    \int\limits_0^{s_n} K(s_{n-1}, s_n)\int\limits_0^{s_{n-1}}\ldots
    \int\limits_0^{s_2} K(s_1, s_2)(x(s_1) - y(s_1))\dif s_1\dif s_2\ldots
    \dif s_n
    $$
    Получаем:
    $$
    \|F_0^n(x-y)\| = \max_{t \in [0, 1]} |(F_0^n(x-y))(t)|
    \leqslant M^n\|x-y\|\max_{t \in [0, 1]} \int\limits_0^t\int\limits_0^{s_n}
    \int\limits_0^{s_{n-1}}\ldots\int\limits_0^{s_3}\int\limits_0^{s_2}
    \dif s_1\dif s_2 \ldots \dif s_n \leqslant \frac{M^n}{n!}\|x-y\|
    $$
    Здесь $M = \max |K|$. Коэффициент $\frac{M^n}{n!}$ стремится к нулю, а
    это значит, что $F_0^n$ — сжимающее, следовательно, существует неподвижная
    точка.
\end{proof}

\begin{example}
    Допустим, что мы хотим решить дифференциальное уравнение
    $y'(t) = a(t)y(t) + b(t)$, $y(0) = y_0$, $a,b \in C[0, 1]$ на промежутке
    $[0, 1]$. Это
    уравнение имеет единственное решение $y \in C^1[0, 1]$. Как это доказать?
    Рассмотрим интегральное уравнение:
    $$
    x(t) = \int\limits_0^t a(s)x(s)\dif s + B(t)
    $$
    По предыдущей теореме существует $x \in C[0,1]$, решающее это
    уравнение. Для этого уравнения также верны утверждения:
    \begin{itemize}
        \item $x'(t) = a(t)x(t) + b(t)$, где $b(t) = B'(t)$;
        \item $x(0) = B(0)$.
    \end{itemize}
    Для решения исходной задачи достаточно выбрать $B$ такое, что $B' = b$ и
    $B(0) = y_0$. Откуда взять непрерывную дифференцируемость $y$?
    $$
    b\in C[0,1] \implies B\in C^1[0,1],
    $$
    $$
    \quad x\in C[0,1],\, a\in C[0,1] \implies
    \int\limits_0^t xa \in C^1[0,1]
    $$
    Таким образом всё доказано.
\end{example}

\section{Линейные операторы}
\begin{definition}
    Пусть $X$, $Y$ — линейные нормированные пространства над одним полем скаляров.
    Отображение $U: X \to Y$ называется линейным, если:
    \begin{enumerate}
        \item $U(x_1+x_2) = U(x_1) + U(x_2)$ $\forall x_1, x_2 \in X$
        \item $U(\lambda x) = \lambda U(x)$, где $\lambda$ — скаляр, $x \in X$
    \end{enumerate}
\end{definition}
\begin{remark}
    Ясно, что выполнение обоих этих свойств равносильно
    $U(\lambda_1x_1 + \lambda_2x_2) = \lambda_1U(x_1) + \lambda_2U(x_2)$.
\end{remark}
\begin{remark}
    В дальнейшем будем обозначать $U(x)$ как $Ux$.
\end{remark}
\begin{proposition}[Свойства линейных отображений]
\mbox{}
    \begin{enumerate}
        \item $U(0) = 0$;
        \item $U(\sum\limits_{j=1}^n \lambda_j x_j) = \sum\limits_{j=1}^n
        \lambda_j Ux_j$;
        \item\label{image_linearity} Если $M \subset X$ — линейное множество,
        то множество
        $U(M)$ линейно в $Y$. Если $M \subset X$ — выпуклое множество,
        то множество $U(M)$ выпукло в $Y$;
        \item\label{preimage_linearity} Если $N \in Y$ — линейное (выпуклое),
        то $U^{-1}(N)$ — линейное (выпуклое). Частный случай: если $N = \{0\}$, то множество
        $U^{-1}(N) = U^{-1}(\{0\}) = \Ker U$ — линейное в $X$;
        \item $\Ker U = \{0\} \iff U$ инъективно;
        \item\label{inverse_linearity} Если $U$ — линейная биекция, то
        $U^{-1}$ — линейное;
        \item Пусть $U_1, U_2: X \to Y$ — линейные. Тогда $U_1 + U_2$,
        $\lambda U_1$ тоже линейны;
        \item Если $X \overto{U} Y \overto{V} Z$, то композиция $V\circ U$
        линейна.
    \end{enumerate}
\end{proposition}
\begin{definition}
    Множество $M$ называется выпуклым, если для любых $x_1,x_2 \in M$ отрезок
    $[x_1, x_2]$ лежит в $M$.
\end{definition}
\begin{proof}[Доказательство предложения]
    Докажем выпуклость в свойстве \ref{image_linearity}.
    $$
    y_1, y_2 \in U(M) \implies \exists x_1, x_2 \in M:\, Ux_1 = y_1,\,
    Ux_2 = y_2
    $$
    $$
    \lambda y_1 + (1-\lambda)y_2 = \lambda Ux_1 + (1 - \lambda)Ux_2 =
    U(\underbrace{\lambda x_1 + (1-\lambda)x_2}_{\in M}) \in U(M)
    $$
    
    В свойстве \ref{preimage_linearity}:
    $$
    x_1, x_2 \in U^{-1}(N) \implies Ux_1, Ux_2 \in N \implies
    \forall \lambda_1, \lambda_2\quad \lambda_1 Ux_1 + \lambda_2 Ux_2 \in N
    \implies
    $$
    $$
    \implies U(\lambda_1 x_1 + \lambda_2 x_2) \in N \implies
    \lambda_1x_1 + \lambda_2x_2 \in U^{-1}(N)
    $$
    
    В свойстве \ref{inverse_linearity} биективность $U$ означает, что
    $\forall y_1, y_2$ $\exists x_1, x_2$ такие, что $Ux_1 = y_1$,
    $Ux_2 = y_2$. Отсюда $U^{-1}(y_1+y_2) = U^{-1}(Ux_1 + Ux_2) =
    U^{-1}(U(x_1 + x_2)) = x_1 + x_2 = U^{-1}(x_1) + U^{-1}(x_2)$.
    
    Доказательства остальных свойств тривиальны.
\end{proof}

\begin{theorem}[Эквивалентные условия непрерывности линейного отображения]
    Пусть $U:~X \to Y$ — линейный оператор. Тогда следующие утверждения
    эквивалентны:
    \begin{enumerate}
        \item $U$ непрерывен;
        \item $U$ непрерывен в нуле;
        \item Образ любого ограниченного множества ограничен;
        \item Существует $C$ такое, что $\forall x\in X$
        выполняется $\|U_x\|_Y = C\|x\|_X$.
    \end{enumerate}
\end{theorem}
\begin{proof}
\mbox{}
    \begin{itemize}
        \item $1 \Rightarrow 2$. Tривиально.
        \item $4 \Rightarrow 1$. $\|Ux_1 - Ux_2\| \leqslant
        C\|x_1 - x_2\|$. Это влечёт липшицевость и, как следствие, непрерывность.
        \item $2 \Rightarrow 3$. Непрерывность в нуле означает, что
        $\forall \varepsilon > 0$ $\exists \delta > 0$ такое, что
        $\|x\|<\delta \implies \|Ux\|<\varepsilon$. Ограниченность множества $M$ в $X$
        означает, что $\exists R:$ $N \subset B_R(0)=\{\|x\|\leqslant R\}$.
        Таким образом, $x\in M \implies \|x\| \leqslant R$.
        $\|\frac{\delta}{2R}x\| \leqslant \frac{\delta}{2} < \delta \implies
        \|U(\frac{\delta}{2R}x)\| < \varepsilon$. Отсюда
        $\|Ux| \leqslant \frac{\varepsilon\cdot 2R}{\delta} \implies Ux
        \in B_{\frac{\varepsilon\cdot 2R}{\delta}}(0)$. То есть, $U(M)$ ограничено.
        \item $3 \Rightarrow 4$. $B_1(0)$ — ограниченное множество. Тогда
        $U(B_1(0))$ — ограничено, т. е. существует такое $C$, что
        $U(B_1(0)) \subset B_C(0)$. Если $\|x\|\leqslant 1$, то $\|Ux\| \leqslant C$.
        Теперь возьмём произвольное $x$. $x' = \frac{x}{\|x\|} \in B_1(0) \implies
        \|Ux'\| \leqslant C$. Но $\|Ux\| = \|U\big(\frac{x}{\|x\|}\big)\| =
        \frac{1}{\|x\|} \cdot \|Ux\|$. Отсюда $\|Ux\|\leqslant C\|x\|$.
    \end{itemize} 
\end{proof}

\begin{definition}
    Пусть $U: X \to Y$ — линейный непрерывный оператор. Тогда нормой оператора $U$
    называется величина $\|U\| = \inf\,\{C\,\big|\,\|Ux\| \leqslant C \|x\|\}$.
\end{definition}
\begin{remark}
    В формулировке определения инфимум и минимум совпадают (это можно доказать, перейдя к 
    пределу в неравенстве $\|Ux\| \leqslant C \|x\|$).
\end{remark}
\begin{remark}
    Выполнено неравенство $\|Ux\|_Y \leqslant \|U\|\cdot \|x\|_X$. В частности,
    $\frac{\|Ux\|_Y}{\|x\|_X} \leqslant \|U\|$ $\forall x \in X$, т. е. можно записать
    $\|U\| = \sup\limits_{x \neq 0}\frac{\|Ux\|}{\|x\|}$. 
\end{remark}
\begin{theorem}[Об эквивалентных способах определения нормы оператора]
     Пусть $U: X \to Y$ — линейный непрерывный оператор. Тогда:
     $$
     \|U\| = \underbrace{\sup\limits_{x \neq 0}\frac{\|Ux\|}{\|x\|}}_A =
     \underbrace{\sup\limits_{\|x\|\leqslant 1} \|U_x\|}_B =
     \underbrace{\sup\limits_{\|x\| < 1} \|U_x\|}_C =
     \underbrace{\sup\limits_{\|x\| = 1} \|U_x\|}_D
     $$
\end{theorem}
\begin{remark}
    Так как замкнутость и ограниченность, вообще говоря, неравносильна компактности
    (за исключением конечномерных пространств),
    в $\sup\limits_{\|x\|\leqslant 1} \|Ux\|$ максимум может и не достигаться.
\end{remark}
\begin{proof}[Доказательство теоремы]
    Очевидно, что $B \geqslant C$ и $B \geqslant D$.
    $$
    B =\sup\limits_{\|x\|\leqslant 1,\,x \neq 0} \|Ux\| \leqslant
    \sup\limits_{\|x\|\leqslant 1,\,x \neq 0} \frac{\|Ux\|}{\|x\|} \leqslant
    \sup\limits_{x \neq 0} \frac{\|Ux\|}{\|x\|} = A
    $$
    Докажем, что $D \geqslant A$. Возьмём $x'=\frac{x}{\|x\|}$, тогда $\|x'\|=1$ и
    $\|Ux\|\leqslant D$. $\|U(\frac{x}{\|x\|})\| = \frac{\|Ux\|}{\|x\|}$.
    Итак, $\frac{\|Ux\|}{\|x\|} \leqslant D$, тогда и
    $\sup\limits_{x \neq 0}\frac{\|Ux\|}{\|x\|}$.
    Осталось проверить, что $C \geqslant A$. Возьмём $x \neq 0$, $\varepsilon > 0$.
    Рассмотрим $x' = \frac{x}{\|x\|(1 + \varepsilon)}$. Тогда $\|x\| < 1$. Отсюда
    следует, что $\|Ux'\| \leqslant C \implies \frac{\|Ux\|}{\|x\|(1+\varepsilon)}
    \leqslant C \implies \frac{\|Ux\|}{\|x\|}\leqslant C(1 + \varepsilon)$. Устремив
    $\varepsilon \to 0$, получим
    $A = \sup\limits_{x \neq 0} \frac{\|Ux\|}{\|x\|} \leqslant C$.
\end{proof}

\section{Пространства линейных непрерывных операторов}
\begin{definition}
    Пусть $X$, $Y$ — линейные нормированные пространства над одним полем скаляров. Возьмём
    $B(X,Y) = \{U: X \to Y,\, U \text{ — линейно, непрерывно}\}$.
    Это \emph{линейное пространство}.
\end{definition}
\begin{theorem}[О свойствах операторной нормы]
    $U,V \in B(X, Y)$.
    \begin{enumerate}
        \item $\|U\| \geqslant 0$, $\|U\| = 0 \iff U = 0$;
        \item $\|\lambda U\| = |\lambda|\|U\|$ ($\lambda$ — скаляр);
        \item $\|U + V\| \leqslant \|U\| + \|V\|$;
        \item $W \in B(Y,Z)$. $WU \in B(X, Z)$, $\|WU\| \leqslant \|W\|\|U\|$.
    \end{enumerate}
\end{theorem}
\begin{proof}
\mbox{}
    \begin{enumerate}
        \item Неотрицательность очевидна. Если $\|U\|=0$, то $\|Ux\|\leqslant
        0\cdot \|x\| \implies \|Ux\| = 0$ $\forall x$;
        \item $\|\lambda U\| = \sup\limits_{\|x\|=1} \|(\lambda U)(x)\| =
         \sup\limits_{\|x\|=1} |\lambda|\|Ux\| = |\lambda| \sup\limits_{\|x\|=1}
         \|U_x\| =
         |\lambda|\|U\|$;
         \item $x \in X$. $\|(U+V)(x) = \|Ux + Vx\| \leqslant \|Ux\| + \|Vx\|
         \leqslant \|U\|\|x\| + \|V\|\|x\| = (\|U\| + \|V\|)\|x\|$
         \item $x \in X$. $\|(WU)(x)\| = \|W(U(x))\| \leqslant \|W\|\cdot \|Ux\|
         \leqslant \|W\|\|U\|\|x\|$.
    \end{enumerate}
\end{proof}
\begin{theorem}[О полноте пространства операторов]
    Если $Y$ полно, то $B(X, Y)$ полно.
\end{theorem}
\begin{proof}
    Возьмём фундаментальную последовательность линейных непрерывных отображений
    $U_n \in B(X, Y)$, то есть $\|U_n - U_m\| \underto{m,n \to \infty} 0$:
    $\forall \varepsilon>0$ $\exists N:$ $\forall m,n > N$ $\|U_n - U_m\| < \varepsilon$.
    Это означает, что $\|(U_n - U_m)(x)\| \leqslant \varepsilon \|x\|$. Следовательно,
    $\{U_nx\}$ фундаментальна в $Y$. Обозначим $Ux = \lim\limits_{n \to \infty}
    U_nx$. Мы хотим проверить, что $U$ непрерывно, линейно и что есть сходимость.
    \begin{enumerate}
        \item (Линейность $U$). $U(\alpha_1x_1 + \alpha_2x_2) = \lim\limits_{n \to \infty}
        U_n(\alpha_1x_1 + \alpha_2x_2) = 
        \alpha_1 \lim U_nx_1 + \alpha_2\lim U_nx_2 = \alpha_1 Ux_1 +
        \alpha_2 Ux_2$
        \item (Нерерывность $U$). Возьмём любое $\varepsilon > 0$, $N$, $\forall m,n>N$,
        $\forall x\in X$. $\|U_nx - U_mx\| \leqslant \varepsilon \|x\| \implies
        \|Ux - U_mx\| \leqslant \varepsilon\|x\|$.
        $\|Ux\| = \|(Ux - U_mx) + U_mx\| \leqslant \|(Ux - U_mx)\| + \|U_mx\| \leqslant
        \varepsilon \|x\| + \|U_m\|\|x\|$. Отсюда $\|U\| \leqslant \varepsilon +
        \|U_m\|$.
        \item (Сходимость $U_n$ к $U$).
        $\forall \varepsilon > 0$ $\exists N$: $\forall m,n>N$ $\forall x\in X$ $
        \|U_nx - U_mx\| \leqslant \varepsilon \|x\|$. Устремив $n$ к бесконечности, 
        получим: $\forall \varepsilon > 0$ $\exists N$: $\forall m > N$
        $\forall x\in X$ $\|Ux - U_mx\| = \|(U - U_m)(x)\| \leqslant \|x\|$
        $\implies \|U - U_m\| \leqslant \varepsilon$. Итак,
        $\forall \varepsilon > 0$ $\exists N$: $\forall m > N$
        $\|U - U_m\| \leqslant \varepsilon$, т. е. $U_n \to U$ в $B(X, Y)$.
    \end{enumerate}
\end{proof}

Следует отметить важный частный случай.
\begin{definition}
    $B (X, \text{поле скаляров }) = X^\ast$ называется
    \emph{сопряжённым пространством к $X$}. $f \in X^\ast$ называется \emph{линейным
    непрерывным функционалом}.
\end{definition}

Норма функционала определяется как $\|f\| = \inf\,\{C\,\big|\,|f(x)|
\leqslant C\|x\|\} =
\sup\limits_{x \neq 0}\frac{|f(x)|}{\|x\|} = \sup\limits_{\|x\|=1} |f(x)|$.

\section{Корректно разрешимые задачи}

Рассмотрим отображение $A: X \to Y$. Мы хотим решить уравнение $Ax = f$. $f$ —
какие-то известные данные.

В общей постановке вопроса корректная разрешимость означает три вещи:
\begin{itemize}
    \item Решение существует для любого $f$.
    \item Решение единственно.
    \item Устойчивость: если $f_n \to f$, то для решений верно, что $x_n \to x$. 
    (Здесь $Ax_n = f_n$, $Ax = f$.)
\end{itemize}

В частном случае, когда $X$ и $Y$ — линейные нормированные пространства и
$A$ — линейное отображение, вышеописанные условия равносильны тому, что
$A^{-1} \in B(Y, X)$.

\begin{remark}
    Самый простой пример корректно разрешимой задачи — случай, когда оператор
    $A$ тождественен.
\end{remark}

\begin{theorem}[Об обратимости оператора, близкого к тождественному]
    Если $B \in B(X, X)$, $X$ — полное и  $\|B\| < 1$, то существует оператор
    $(I \pm B)^{-1} \in B(X, X)$. ($I$ — тождественный оператор.)
\end{theorem}
\begin{proof}
    Приведём два способа доказать эту теорему.
    \begin{enumerate}
        \item Возьмём уравнение $(I-B)x = f$. Надо доказать, что для любого
        $f \in X$ существует единственный $x \in X$, решающий это уравнение.
        Это равносильно $x = f + Bx = g(x)$. Заметим, что $x$ удовлетворяет
        уравнению тогда и только тогда, когда $x$ — неподвижная точка отображения
        $g$. Проверим, что $g$ — сжимающее.
        $\|g(x_1) - g(x_2)\| = \|(f + Bx_1) - (f + Bx_2)\| =
        \|Bx_1 - Bx_2\| \leqslant \underbrace{\|B\|}_{< 1} \cdot \|x_1 - x_2\|$.
        
        Теперь проверим устойчивость. Пусть $f_n \to f$, $(I-B)x_n = f_n$,
        $(I-B)x=f$. Нужно проверить, что $x_n \to x$. $x_n = f_n + Bx_n$,
        $x = f + Bx$.
        $$
        \|x_n - x\| = \|f_n + Bx_n - f - Bx\| \leqslant \|f_n - f\| +
        \|Bx_n - Bx\| \leqslant \|f_n - f\| + \|B\|\cdot \|x_n - x\|
        $$
        Отсюда
        $$
        0 \leqslant \underbrace{(1 - \|B\|)}_{>0}\|x_n - x\|\leqslant
        \underbrace{\|f_n - f\|}_{\to 0} \implies
        \|x_n - x\| \to 0
        $$
        \item Докажем формулу $(I - B)^{-1} = I + B + B^2 + B^2 + \ldots$.
        Необходимо проверить, что этот ряд сходится. Докажем, что он
        сходится абсолютно, то есть $\|I\| + \|B\| + \|B^2\| + \ldots < \infty$.
        Заметим, что $\|B^k\| \leqslant \|B\|^k$. Отсюда
        $\|I\| + \|B\| + \|B^2\| + \ldots \leqslant
        \|I\| + \|B\| + \|B\|^2 + \ldots$. Но это — геометрическая прогрессия,
        она сходится.
        Частичные суммы: $S_n = I + B + \ldots B^{n-1}$,
        $(I - B)S_n = S_n(I-B) = I - B^n \underto{n \to \infty} I$.
        Мы воспользовались полнотой пространства, утверждая, что абсолютная
        сходимость влечёт сходимость ряда.
    \end{enumerate}
\end{proof}

\begin{theorem}[Об обратимости оператора, близкого к обратимому]
    Пусть $U \in B(X, Y)$ — линейное отображение и существует $B^{-1}\in B(Y, X)$.
    Кроме того, $X$ \textbf{или} $Y$ — полное пространство.
    Рассмотрим $V\in B(X, Y)$ такой, что $\|V\| < \|U^{-1}\|^{-1}$. Тогда
    существует $(U + V)^{-1} \in B(Y, X)$.
\end{theorem}
\begin{proof}
    $U + V = U(I_X + U^{-1}V)$ (или $(I_Y + VU^{-1})U$).
    Оператор $U$ обратим, обратный к нему оператор
    непрерывен. Получаем $\|U^{-1}V\|\leqslant\|U^{-1}\|\cdot\|V\| < 1$.
\end{proof}

\section{Линейные непрерывные функционалы}

Вспомним, что если $X$ — нормированное пространство, то
$X^\ast = B(X, \text{поле скаляров})$ называется сопряжённым к $X$ пространством.
Норма функционала определяется как $\|f\| = \inf\,\{C\,\big|\,|f(x)|
\leqslant C\|x\|\} =
\sup\limits_{x \neq 0}\frac{|f(x)|}{\|x\|} = \sup\limits_{\|x\|=1} |f(x)|$.

\begin{example}[Функционалы в пространстве Лебега]
        Рассмотрим $L^p(T, \mu)$, причём
        $1 < p < \infty$. Возьмём $q$ — сопряжённый показатель такой, что
        $\frac{1}{q} + \frac{1}{p} = 1$. Возьмём также $y_0 = L^q(T, \mu)$.
        Определим функционал $f$ формулой
        $f(x) = \int\limits_T x(t)y_0(t)\dif \mu(t)$. Нам нужно проверить, что
        это действительно функционал, что он непрерывен (линейность очевидна).
        Чтобы этот функционал был функционалом, необходимо, чтобы
        подынтегральная функция была суммируемой. Для этого воспользуемся
        неравенством Гёльдера:
        $$
        \int\limits_T |x(t)y_0(t)| \dif \mu(t) \leqslant
        \bigg(\int\limits_T |x|^p\bigg)^\frac{1}{p}
        \bigg(\int\limits_T |y_0|^q\bigg)^\frac{1}{q} = \|y_0\|_q\cdot \|x\|_p <
        \infty
        $$
        $$
        |f(x)| \leqslant \underbrace{\|y_0\|_q}_{=C} \cdot \|x\|
        \implies \|f\| \leqslant \|y_0\|_q
        $$
        
        Проверим, что $\|f\| \geqslant \|y_0\|_q$.
        $$
        x_0(t) = \frac{|y_0|^q}{y_0} = |y_0|^{q-1} \frac{|y_0|}{y_0} =
        |y_0|^{q-1}\sign y_0
        \implies x_0y_0 = |y_0|^q
        $$
        $$
        |f(x_0)| = \bigg|\int\limits_T x_0y_0\bigg| = \int\limits_T|y_0|^q
        $$
        Но так как $\frac{1}{p} + \frac{1}{q} = 1$, то $(q-1)p = q$.
        $$
        \|x_0\|_p = \bigg(\int\limits_T |x_0|^p\bigg)^\frac{1}{p} =
        \bigg(\int\limits_T |y_0|^{(q-1)p}\bigg)^\frac{1}{p} =
        \bigg(\int\limits_T |y_0|^q\bigg)^\frac{1}{p}
        $$
        $$
        \|f\|\geqslant \frac{|f(x_0)|}{\|x_0\|_p} =
        \frac{\int\limits_T |y_0|^q}{\bigg(\int\limits_T} \cdots
        $$
        Таким образом, $L^q(T, \mu) \hookrightarrow L^q(T, \mu)^\ast$,
        $y_0 \mapsto f$ и $\|y_0\|_q = \|f\|$. Имеет место
        \emph{изометрическое вложение}, и даже более того, биекция.
\end{example}
\begin{example}
    Рассмотрим пространство $C[-1, 1]$. Пусть $f(x) = \int\limits_{-1}^1
    tx(t)\dif t$. Снова хотим доказать, что это функционал, что он непрерывен и 
    линеен.
    Для непрерывности достаточно установить, что $|f(x)| \equiv C\|x\|$.
    $$
    |f(x)| \leqslant \int\limits_{-1}^1 |t||x(t)|\dif t \leqslant
    \max |x| \int\limits_{-1}^1 |t|\dif t = \|x\| \implies \|f\| \leqslant 1
    $$
    Непрерывность доказана.
    Теперь возьмём функцию $x_\varepsilon(t) =
    \begin{cases}
        1,\quad t \geqslant \varepsilon \\
        \frac{t}{\varepsilon},\quad |t|\leqslant \varepsilon \\
        -1,\quad t \leqslant -\varepsilon\\
    \end{cases}$.
    $$
    f(x_\varepsilon) = \int\limits_{-1}^1 tx_\varepsilon(t)\dif t =
    \bigg(\int\limits_{-1}^{-\varepsilon} + \int\limits_\varepsilon^1\bigg)
    |t|\dif t + \int\limits_{-\varepsilon}^\varepsilon \frac{t^2}{\varepsilon}
    \dif t = 1 + O(\varepsilon)
    $$
    Получаем, что $\|f\|\geqslant \frac{f(x_\varepsilon}{\|x_\varepsilon\|}
    \underto{\varepsilon \to 0} 1$.
    Теперь возьмём
    $y_0 \in L^1(-1, 1)$, $f(x) = \int\limits_{-1}^1 y_0(t)x(t)\dif t$.
    $$
    |f(x)| \leqslant \int\limits_{-1}^1 |y_0||x| \leqslant \|x\|\int\limits_{-1}^1
    |y_0| \leqslant \|y_0\|_1\cdot \|x\|_C
    $$
    Значит, $f$ — линейный непрерывный функционал. $\|f\|=\|y_0\|_1$,
    $x_0(t) = \sign y_0 \notin C$.
\end{example}
\begin{exercise}
    Пусть $\delta(x) = x(0)$. Доказать, что $\delta \notin L^1(-1, 1)$, то есть
    не существует $y_0 \in L^1(-1, 1)$ такого, что $\forall x\in C[-1, 1]$
    $\int\limits_{-1}^1 y_0(t)x(t)\dif t = x(0)$
\end{exercise}

\begin{theorem}
    $(c_0)^\ast = \ell^1$
\end{theorem}

Напомним, что $\ell^\infty = \{x = (x_1, x_2, \ldots),\, \|x\|_\infty = 
\sup\limits_{j \geqslant 1} |x_j| < \infty\}$ и
$c_0 = \{x = (x_1, x_2, \ldots),\, \lim\limits_{j \to \infty} x_j = 0\}$,
$c_0 \subset \ell^\infty$. При этом $\|x\|_{c_0} = \|x\|_\infty$. $c_0$ — полное
нормированное пространство.

Рассмотрим $L_\text{fin} \subset \ell^\infty$ такое, что $x \in L_\text{fin}$,
если у $x$ лишь конечное число ненулевых координат. Отметим, что $L_\text{fin}$
является линейной оболочкой векторов $e_1, e_2, \ldots$, где
$e_k = (0, 0, \ldots, 0,\underbrace{1}_{k}, 0, \ldots)$. Также
$\overline{L_\text{fin}} = c_0$
\begin{itemize}
    \item $x \in c_0 \implies \exists x^{(n)} \in L_\text{fin}$: $x^{(n)} \to 
    \infty$, где $x^{(n)} = (x_1, x_2, \ldots, x_n, 0, 0, \ldots)$.
    $\|x - x^{(n)}\| = \|(0, 0, \ldots, 0, x_{n+1}, x_{n+2}, \ldots)\|_\infty =
    \sup\limits_{j \geqslant n+1}|x_j|$.
    \item $c_0$ замкнуто.
\end{itemize}

\begin{proof}
\mbox{}
    \begin{enumerate}
        \item Возьмём $y^{(0)} \in \ell^1$, где
        $y^{(0)} = (y_1^{(0)}, y_2^{(0)}, \ldots)$
        и $\|y^{(0)}\|_1 = \sum\limits_{j=1}^\infty |y_j^{(0)}| < \infty$.
        Построим по нему функционал на $c_0$.
        
        $\cdots$
        
        Мы построили вложение $\ell^1 \hookrightarrow (c_0)^\ast$, $y^{(0)} 
        \mapsto f$.
        \item Пусть нам дан функционал $f \in (c_0)^\ast$. Мы хотим построить по
        нему $y \in \ell^1$. Положим $f(e_j) = y_j$ ($y = (y_1, y_2, \ldots)$).
        Нам нужно проверить, что $y \in \ell^1$ и что
        $\forall x$ $f(x) = \sum x_jy_j$.
        Возьмём $z^{(n)} =
        (\sign y_1, \sign y_2, \ldots, \sign y_n, 0, 0,\ldots)$.
        $|f(z^{(n)}| \leqslant \|f\|\cdot \|z^{(n)}\|_\infty \leqslant \|f\|$.
        Но левая часть неравенства равна $\sum\limits_{j=1}^\infty |y_j|$. Из
        неравенства следует, что ряд сходится, отсюда $y \in \ell^1$.
        
        Покажем теперь, что $\forall x$ $f(x) = \sum x_jy_j$.
        пусть $x = (x_1, x_2, \ldots) = \sum\limits_{j=1}^\infty x_je_j$.
        $$
        f\bigg(\sum\limits_{j=1}^n x_je_j\bigg) = \sum\limits_{j=1}^n x_jf(e_j) =
        \sum\limits_{j=1}^n x_jy_j \underto{n \to \infty} \sum\limits_{j=1}^\infty
        x_jy_j
        $$
        Левая часть стремится к $f(x)$, так как $\sum\limits_{j=1}^n = x_je_j
        \underto{n \to \infty} x$.
    \end{enumerate}
\end{proof}

\section{Интегральные операторы. Часть I}

Что такое интегральный оператор? Допустим, у нас есть функция двух переменных
$K(s, t)$, называемая \emph{ядром интегрального оператора} (не путать с ядром
оператора). Оператор действует следующим образом: он берёт функцию $x(s)$ и
преобразует её в функцию $(Ux)(t)$ по формуле
$(Ux)(t) = \int K(s, t) x(s)$ (множество интегрирования и мера определяются 
отдельно). Какими свойствами должна обладать функция $K$,
чтобы этот оператор был <<хорошим>>?

\subsection{Интегральные операторы в пространствах Лебега}

Будем рассматривать переменные $s$ на множестве $S$ с мерой $\nu$ и $t$
на множестве $T$ с мерой $\mu$, а также функцию
$K: S \times T \to \text{поле скаляров}$, притом измеримую. Пусть $x$ — также
измеримая функция на $S$, $(Ux)(t) = \int\limits_S K(s, t)x(s)\dif \nu(s)$.
Какие условия нужно наложить на функцию $K$, чтобы оператор $U$ действовал из
$L^p(s, \nu)$ в $L^r(T, \mu)$ и был непрерывен?

$$
\int\limits_T |(Ux)(t)|^r \leqslant \int\limits_T
\bigg(\int\limits_S|K(s, t)||x(s)|\dif s\bigg)^r \dif t \leqslant
\int\limits_T\bigg(\bigg(\int\limits_S |K(s, t)| \cdots
$$

Таким образом, мы доказали следующую теорему.

\begin{theorem}[О гёльдеровских условиях непрерывности]
    Если $\int\limits_T\bigg(\int\limits_S |K|^q \dif s\bigg)^\frac{r}{q} \dif t
    < \infty$, то $U$ действует непрерывно из $L^p(s, \nu)$ в $L^r(T, \mu)$.
\end{theorem}

Пусть $p = 2$, $r = 2$, то есть $q = 2$. Тогда:
$$
\int\limits_T \int\limits_S |K(s, t)|^2 \dif s \dif t < \infty \iff
K \in L^2(S\times T, \nu \times \mu)
$$
и $\|U\|\leqslant \|K\|_{L^2(S\times T, \nu\times \mu)}$. Операторы, 
удовлетворяющие таким условиям, называются операторами Гильберта-Шмидта, а
$K$ — ядром Гильберта-Шмидта.
\begin{remark}
    Существуют линейные непрерывные интегральные операторы, не являющиеся
    операторами Гильберта-Шмидта.
\end{remark}

\subsection{Тест Шура}

\begin{theorem}[Тест Шура]
    Пусть $(Ux)(t) = \int\limits_S K(s, t)x(s)\dif \nu(s)$.
    Предположим, что существуют строго положительные функции $\varphi: S \to 
    \real$, $\psi: T \to \real$ и числа $A, B \in \real$ такие, что:
    \begin{enumerate}
        \item $\int\limits_S |K(s, t)|\varphi(s)\dif \nu(s) \leqslant A\psi(t)$ 
        для почти всех $t \in T$.
        \item $\int\limits_T |K(s, t)|\psi(s)\dif \mu(s) \leqslant B\varphi(s)$ 
        для почти всех $s \in S$.
    \end{enumerate}
    Тогда $U$ — линейный непрерывный оператор из $L^2(S, \nu)$ в $L^2(T, \mu)$.
\end{theorem}
\begin{proof}
    $$
    |(Ux)(t)| \leqslant \int\limits_S \sqrt{|K(s, t)|\varphi(s)}
    \sqrt{\frac{|K(s,t)||x(s)|^2}{\varphi(s)}} \dif \nu(s) \leqslant
    \underbrace{\bigg(\int\limits_S |K(s,t)|\varphi(s) \dif s\bigg)^\frac{1}{2}
    }_{\leqslant A\psi(t)}\bigg(\int\limits_S \frac{|K(s,t)||x(s)|^2}{\varphi(s)}
    \dif s\bigg)^\frac{1}{2}
    $$
    $$
    \int\limits_T |(Ux)(t)|^2 \dif t \leqslant \int\limits_T A\psi(t)
    \int\limits_S \frac{|K(s,t)||x(s)|^2}{\varphi(s)} \dif s \dif t
    $$
\end{proof}

\begin{exercise}
\mbox{}
    \begin{enumerate}
        \item $S = T = (0, 1)$ с мерой Лебега, $K(s,t) = \frac{1}{\sqrt{|s-t|}}$.
        Заметим, что получается оператор, не являющийся
        оператором Гильберта-Шмидта, так как
        $\int\limits_0^1\int\limits_0^1 \frac{1}{|s-t|\dif s\dif t} = +\infty$.
        Придумать тест Шура для этого случая.
        \item $S = T = \real$, $K(s,t) = e^{-(s + t)^2}$. Является $U$ оператором
        Гильберта-Шмидта, и, если нет, является ли он непрерывным?
        \item $S = T = (0, +\infty)$, $K(s,t) = e^{-st}$. Установить непрерывность
        $U$ с помощью теста Шура.
        \item $S = T = \mb N$, $\nu = \mu = \#$,
        $K: \mb N \times \mb N \to \real$. Тогда оператор $U$ равен
        $\sum\limits_{j=1}^\infty K_{ij}x_j$.
    \end{enumerate}
\end{exercise}
\begin{theorem}[Тест Шура в дискретном случае]
    Пусть существуют $\varphi_j > 0$, $\psi_i > 0$, $A, B$ такие, что
    \begin{enumerate}
        \item $\sum |K_{ij}|\varphi_j \leqslant A\psi_i$ $\forall i \in \mb N$
        \item $\sum |K_{ij}|\psi_j \leqslant B\varphi_j$ $\forall j \in \mb N$
    \end{enumerate}
    Тогда $U: \ell^2 \to \ell^2$ непрерывен и $\|U\| \leqslant \sqrt{AB}$.
\end{theorem}

\begin{example}[Оператор Харди]
    Оператор Харди $H$ действует в пространстве $L^2(0, +\infty)$:
    $$
    (Hx)(t) = \frac{1}{t}\int\limits_0^t x(s) \dif s
    $$
    
    Частный случай: $H: \ell^2 \to \ell^2$ и
    $(Hx)_k = \frac{1}{k}(x_1 + \ldots + x_k)$ (среднее арифметическое).
    
    Применим тест Шура.
    $$
    \frac{1}{t}\int\limits_0^t x(s) \dif s = \int\limits_0^\infty K(s,t)x(s)\dif s
    $$
    где $K(s,t) = \frac{1}{t}\chi_{[0,t]}(s) = \frac{1}{t}\chi_{[s, +\infty)(t)}$.
    Возьмём $\varphi(s) \equiv 1$. Тогда
    $$
    \int\limits_0^\infty|K(s,t)|\varphi(s) \dif s =
    \frac{1}{t}\int\limits_0^t \dif s = 1
    $$
    Взяв $\psi(t) \equiv 1$, получим
    $$
    \int\limits_0^\infty|K(s,t)|\psi(t) \dif t = \int\limits_0^\infty \frac{1}{t}
    \dif t = \infty
    $$
    Значит, такое $\psi$ не подходит. Возьмём $\psi(t) = t^{-\alpha}$, где
    $\alpha > 0$. Тогда
    $$
    \int\limits_0^\infty|K(s,t)|\psi(t) \dif t =
    \int\limits_0^\infty \frac{1}{t^{\alpha + 1}} \dif t =
    \frac{s^{-\alpha}}{\alpha}
    $$
    В качестве $\varphi(s)$ возьмём $s^{-\alpha}$.
    $$
    \int\limits_0^\infty|K(s,t)|\varphi(s) \dif s =
    \frac{1}{t} \int\limits_0^t s^{-\alpha} \dif s =
    \frac{1}{t} \frac{t^{1- \alpha}}{1 - \alpha} =
    \frac{t^{-\alpha}}{1 - \alpha}
    $$
    Заметим, что при этом должно быть $\alpha < 1$. Кроме того,
    $$
    \|H\| \leqslant \frac{1}{\sqrt{\alpha(1 - \alpha)}}\quad
    \forall \alpha \in (0, 1) \implies \|H\| \leqslant 2
    $$
\end{example}
\begin{exercise}
    Доказать, что $\|H\| = 2$.
\end{exercise}

\subsection{Интегральные операторы с непрерывным ядром}

Будем рассматривать ограниченную область $\Omega \subset \real^m$, пространство
$L^2(\Omega)$ и пространство непрерывных функций $C(\overline\Omega)$. Пусть также
у нас есть функция
$K: \overline\Omega \times \overline\Omega \to \real(\complex)$,
$K \in C(\overline\Omega)$, $\|K\|_{C(\overline\Omega)} = M$.

\begin{theorem}
    Рассмотрим оператор $U$ такой, что
    $(Ux)(t) = \int\limits_\Omega K(s,t)x(s) \dif s$. Верно, что
    $U \in B(L^2(\Omega), C(\overline\Omega))$.
\end{theorem}
\begin{proof}
    Докажем, что если $x \in L^2(\Omega)$, то $Ux \in C(\overline\Omega)$.
    (Здесь непрерывность $x$ не гарантируется.)
    
    $$
    |Ux(t_1) - Ux(t_2)| =
    \bigg|\int\limits_\Omega K(s, t_1) - K(s, t_2)x(s) \dif s\bigg| \leqslant
    \bigg(\int\limits_\Omega |K(s, t_1) - K(s, t_2)|^2 \dif s\bigg)^\frac{1}{2}
    \|x\|_2
    $$
    По теореме Кантора $K$ равномерно непрерывно на
    $\overline\Omega \times \overline\Omega$, то есть:
    $$
    \forall \varepsilon > 0\quad\exists \delta > 0:\quad
    \underbrace{|(s_1, t_1) - (s_2, t_2)|}_{\sqrt{|s_1 - s_2|^2 + |t_1 - t_2|^2}}
    < \delta \implies
    |K(s_1, t_1) - K(s_2, t_2)| < \varepsilon
    $$
    
    Если $|t_1 - t_2| < \delta$, то $|K(s, t_1) - K(s, t_2)| < \varepsilon$,
    отсюда $|Ux(t_1) - Ux(t_2) < \varepsilon |\Omega|^\frac{1}{2}\cdot \|x\|_2$
    
    Теперь докажем, что $\|Ux\|_{C(\overline\Omega)}
    \leqslant C \|x\|_{L^2(\Omega)}$.
    $$
    \|Ux\|_{C(\overline\Omega)} = \max\limits_{t \in \overline \Omega}
    \bigg| \int\limits_\Omega K(s, t)x(s) \dif s\bigg| \leqslant
    \max\limits_{t \in \overline \Omega} \bigg(\int\limits_\Omega |K(s, t)|^2
    \dif s\bigg)^\frac{1}{2} \|x\|_2 \leqslant
    (M^2 \cdot |\Omega|)^\frac{1}{2}\|x\|_{L^2(\Omega)}
    $$
\end{proof}

Рассмотрим оператор вложения $j: C(\overline\Omega) \to L^2(\Omega)$,
$x \mapsto x$. Справедливо следствие:

\begin{corollary}
    \begin{enumerate}
        \item $jU \in B(L^2(\Omega), L^2(\Omega))$
        \item $Uj \in B(C(\overline\Omega), C(\overline\Omega))$
    \end{enumerate}
\end{corollary}
\begin{proof}
    Заметим, что $C(\overline\Omega) \subset L^2(\Omega)$.
    $$
    \bigg(\int\limits_\Omega |x(s)|^2 \dif t\bigg)^\frac{1}{2} \leqslant
    \bigg(\|x\|_{C(\overline\Omega)}^2 \cdot |\Omega|\bigg)^\frac{1}{2} =
    |\Omega|^\frac{1}{2} \cdot \|x\|_{C(\overline\Omega)}
    $$
    Получаем
    $$
    \|x\|_{L^2(\Omega)} \leqslant
    |\Omega|^\frac{1}{2} \cdot \|x\|_{C(\overline\Omega)}
    $$

    $$
    \|jx\|_{L^2(\Omega)} = \|x\|_{L^2(\Omega)} \leqslant
    C \cdot \|x\|_{C(\overline\Omega)}
    $$
    То есть $j$ непрерывен.
    $$
    C(\overline\Omega) \hookrightarrow L^2(\Omega) \overto{U} C(\overline\Omega) \
    \hookrightarrow L^2(\Omega)
    $$
\end{proof}

\subsection{Операторы со слабой особенностью}

Рассмотрим оператор $Ux(t) = \int\limits_\Omega K(s, t) x(s) \dif s$, причём $K$ —
ядро со слабой особенностью, а $\Omega \subset \real^m$.

\begin{definition}
    $K$ — ядро со слабой особенностью, если оно представляется в виде:
    $$
    K(s, t) = \frac{A(s, t)}{|s - t|^\alpha}
    $$
    Здесь $A \in C(\overline\Omega \times \overline\Omega)$, $\alpha < m$
\end{definition}
\begin{example}
    $\Omega = (0, 1)$, $K(s, t) = \frac{1}{\sqrt{|s - t|}}$
\end{example}
\begin{remark}
    Предположим, что $K(s, t) = \frac{a(s, t)}{|s - t|^\alpha}$, $\alpha < m$,
    $a$ — ограниченная функция, непрерывная вне диагонали множества
    $\overline\Omega \times \overline\Omega$, то есть в точках $(s, t)$ таких,
    что $s \neq t$. Тогда $K$ — ядро со слабой особенностью. Почему? Можно
    записать $K(s, t) = \frac{a(s, t)|s-t|^\delta}{|s-t|^{\alpha + \delta}}$, где
    $\alpha + \delta < m$. $A(s, t) = a(s, t)|s-t|^\delta$ непрерывно на
    $\overline\Omega \times \overline\Omega$
\end{remark}

Почему особенность <<слабая>>? Чтобы ответить на этот вопрос, сформулируем лемму.

\begin{lemma}
    Пусть у нас есть шар $B(0, \rho) \subset \real^m$. Тогда
    $\int\limits_{B(0, \rho)} \frac{\dif x}{|x|^\alpha}$ конечен тогда и только 
    тогда, когда $\alpha < m$.
\end{lemma}
\begin{proof}
    Вычислим этот интеграл.
    $$
    \int\limits_{B(0, \rho)} \frac{\dif x}{|x|^\alpha} =
    \int\limits_0^\rho \int\limits_{S_1(0)}
    r^{m-1}\frac{1}{r^\alpha}\dif \theta \dif r =
    |S_1| \int\limits_0^\rho r^{m - \alpha - 1} \dif r =
    |S_1| \frac{r^{m - \alpha}}{m - \alpha}\bigg|_0^\rho =
    |S_1|\frac{\rho^{m - \alpha}}{m - \alpha}
    $$
\end{proof}
\begin{theorem}
    Пусть $U$ — оператор со слабой особенностью:
    $Ux(t) = \int\limits_\Omega K(s,t) x(s) \dif s$, $\Omega \subset \real^m$.
    Тогда $U\in B(L^2(\Omega), L^2(\Omega))$.
\end{theorem}
\begin{proof}
    Применим тест Шура. Возьмём функцию $\varphi(s) \equiv 1$.
    $$
    \int\limits_\Omega|K(s, t)|\dif s =
    \int\limits_\Omega \frac{|A(s,t)|}{|s-t|^\alpha}\dif s \leqslant
    M\cdot \int\limits_\Omega \frac{1}{|s-t|^\alpha}\dif s \leqslant
    M\cdot \int\limits_{B_d(t)} \frac{\dif s}{|s - t|^\alpha} \leqslant
    M \cdot \int\limits_{B_d(0)} \frac{\dif z}{|z|^\alpha} \leqslant
    $$
    $$
    \leqslant M \cdot \frac{d^{m - \alpha}}{m - \alpha}
    $$
    Здесь $A \in C(\overline\Omega \times \overline\Omega)$,
    $\|A\|_{C(\overline\Omega \times \overline\Omega)} = M$,
    $d = \diam \overline\Omega$
    
    Получаем, что $\psi(t) = 1$.
\end{proof}
\begin{theorem}
    В условиях предыдущей теоремы также верно
    $U \in B(C(\overline\Omega), C(\overline\Omega))$.
\end{theorem}
\begin{proof}
    \begin{enumerate}
        \item $x \in C(\overline\Omega) \implies Ux \in C(\overline\Omega)$
        \item $\|Ux\|_{C(\overline\Omega)} \leqslant C \|x\|_{C(\overline\Omega)}$
        $$
        |(Ux)(t)| = \bigg|\int\limits_\Omega
        \frac{A(s,t)}{|s,t|^\alpha}x(s)\dif s\bigg| \leqslant M\cdot \|x\|
        \int\limits_\Omega \frac{\dif s}{|s - t|^\alpha} \leqslant
        \frac{Md^{m - a}}{m - \alpha}\|x\|.
        $$
    \end{enumerate}
\end{proof}

\end{document}
