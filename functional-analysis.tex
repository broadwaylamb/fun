\documentclass[11pt,openany,a4paper]{scrartcl}

\usepackage{indentfirst}
\usepackage{amsmath,amsthm,amssymb,amsfonts,amsopn}
\usepackage{mathtext}
\usepackage{enumitem}
\usepackage[T1,T2A]{fontenc}
\usepackage[utf8]{inputenc}
\usepackage[english,russian]{babel}
\usepackage[intlimits]{mathtools}
\usepackage[makeroom]{cancel}
\usepackage{titletoc}
\renewcommand{\bfdefault}{sbc}
\usepackage{ccfonts,eulervm,microtype}
\usepackage{enumitem}
\usepackage{tikz}
\usetikzlibrary{arrows}
\usetikzlibrary{calc}

\tikzset{
    pil/.style={
           ->,
           thick,
           shorten <=2pt,
           shorten >=2pt},
    axis/.style={very thick, ->, >=stealth'},
}

\usepackage[portrait,a4paper,margin=2.5cm,headsep=5mm]{geometry}

\author{Ф. Л. Бахарев \thanks{Конспект подготовлен студентом Яскевичем С. В.}}
\title{Функциональный анализ}

\theoremstyle{plain}
\newtheorem{theorem}{Теорема}[section]
\newtheorem{corollary}[theorem]{Следствие}
\newtheorem{proposition}[theorem]{Предложение}
\newtheorem{lemma}[theorem]{Лемма}
\newtheorem{exercise}[theorem]{Упражнение}

\theoremstyle{definition}
\newtheorem{definition}[theorem]{Определение}
\newtheorem{remark}[theorem]{Замечание}
\newtheorem{example}[theorem]{Пример}
\newtheorem{examples}[theorem]{Примеры}
\newtheorem{num}[theorem]{}

\newcommand\mb{\mathbb}
\newcommand\real{\mb R}
\newcommand{\complex}{\mb C}
\newcommand\eqdef{\mathrel{\stackrel{\makebox[0pt]{\mbox{\normalfont\tiny def}}}{=}}}
\newcommand\lparagraph[1]{\paragraph{#1}\mbox{}\\}
\newcommand{\pd}[2]{\frac{\partial #1}{\partial #2}}
\newcommand{\uto}{\rightrightarrows}
\DeclareMathOperator{\Ree}{Re}
\DeclareMathOperator{\Img}{Im}
\DeclareMathOperator{\Arg}{Arg}
\DeclareMathOperator{\dist}{dist}
\DeclareMathOperator{\const}{const}
\DeclareMathOperator{\Ln}{Ln}

\begin{document}

\maketitle

\tableofcontents

\pagebreak

\section{Линейное нормированное пространство}

\begin{definition}
	Линейное множество $L$ над полем скаляров $\real$ ($\complex$) — множество с
	операциями сложения и умножения на скаляр, удовлетворяющее свойствам:
	\begin{enumerate}
		\item $(x + y) + z = x + (y + z)$ $\forall x,y,z \in L$
		\item $x + y = y + x$ $\forall x,y,z \in L$
		\item Существует элемент $0$ такой, что $x + 0 = x$ $\forall x \in L$
		\item Для любого $x \in L$ существует обратный элемент по сложению $-x$ такой, что
		$-x + x = 0$
		\item $\lambda(\mu x) = (\lambda \mu) x$ $\forall \lambda, \mu$ — скаляров,
		$x \in L$
		\item $\lambda(x + y) = \lambda x + \lambda y$
		\item $(\lambda + \mu)x = \lambda x + \lambda y$
	\end{enumerate}
\end{definition}

\begin{definition}
	$\varphi: L \to \real$ называется нормой, если:
	\begin{enumerate}
		\item $\varphi(x + y) \leqslant \varphi(x) \varphi(y)$
		\item $\varphi(\lambda x) = |\lambda|\varphi(x)$
		\item $\varphi(x) \geqslant 0$
		\item $\varphi (x) = 0 \iff x = 0$
	\end{enumerate}
	
	Если выполнены только первых три свойства, то $\varphi$ называется полунормой.
\end{definition}
\begin{remark}
	\begin{enumerate}
		\item $\rho (x, y) = \varphi(x - y)$ — метрика
		\item Если на пространстве задана норма $\|\cdot\|$, то $X = (L, \varphi)$ — 
		нормированное пространство.
	\end{enumerate}
\end{remark}

\begin{definition}
	$x_n \to x$ в $X$, если $\|x_n - x\| \to 0$ при $n \to \infty$, то есть $\forall 
	\varepsilon > 0 \exists N:$~$\forall n > N$ $\|x_n - x\| < \varepsilon$
\end{definition}

\begin{definition}
	$\{x_n\} \subset X$ — фундаментальная последовательность (сходящаяся в себе, 
	последовательность Коши), если $\|x_n - x_m\| \to 0$ при $m,n \to \infty$, то есть
	$\forall \varepsilon > 0 \exists N:$ $\forall m,n > N$ $\|x_m - x_m\| < \varepsilon$
\end{definition}

\begin{remark}
	$x_n \to x \implies \{x_n\}$ — фундаментальная. Обратное, вообще говоря, неверно.
\end{remark}
\begin{definition}
	Нормированное пространство $X$ называется полным, если из фундаментальности 
	последовательности следует существование предела.
\end{definition}

\begin{definition}
	Пусть $x_n \in X$. $\sum\limits_{j = 1}^\infty x_j$ сходится, если
	$S_n = \sum\limits_{j = 1}^n x_j$ имеет предел $\lim S_n = S$. $S$ называется
	суммой ряда.
\end{definition}

\begin{definition}
	Ряд сходится абсолютно, если $\sum\limits_{j = 1}^\infty \|x\|$ сходится.
\end{definition}
\begin{remark}
	Из абсолютной сходимости не следует обычная сходимость.
\end{remark}

$S_n$ сходится $\iff |S_n - S_m| \to 0$. Пусть $C_n = \sum\limits_{j = 1}^n \|x\|$.
$C_n$ сходится $\iff |C_n - C_m|~\to~0$.
Если мы хотим, чтобы сходимость $S_n$ была равносильна
$\|S_n - S_m\| \to 0$, то нам нужна полнота пространства.

\begin{definition}
	Полное линейное нормированное пространство называется банаховым пространством (в честь 
	польского математика Стефана Банаха).
\end{definition}
\begin{examples}
	\begin{itemize}
		\item Евклидово пространство: $\real^n$ с нормой
		$\|x\| = |x| = \sqrt[n]{|x_1|^2 + \ldots + |x_n|^2}$ — то же, что
		$\ell_n^2$ с нормой $\|\cdot \|_2$
		\item $\ell_n^1 = (\real^n, \|\cdot\|_1)$, где $\|x\|_1 = |x_1| + \ldots + |x_n|$
		\item $\ell_n^\infty = (\real^n, \|\cdot\|_\infty)$,
		где $\|x\|_\infty = \max\limits_{1 \leqslant j \leqslant n} |x_j|$
		\item $\ell_n^p = (\real(\complex), \|\cdot\|_p$,
		$\|x\|_p = (\sum_{j=1}^n |x_j|^p)^{\frac{1}{p}}$, $p \geqslant 1$
		\item Пусть $\Omega$ — область в $\real^m$, т. е. ограниченное открытое множество.
		$\overline\Omega$ — замыкание $\Omega$. Ясно, что $\overline \Omega$ — компакт в
		$\real^m$. Рассмотрим пространство $C(\overline \Omega)$ с нормой
		$\|x\| = \max_{t \in \overline \Omega} |x(t)|$
	\end{itemize}
\end{examples}
\begin{exercise}
	Верно ли, что $\|x\|_p \to \|x\|_\infty$ при $p \to \infty$?
\end{exercise}

\begin{theorem}
	Пространство $C(\overline \Omega)$ полно.
\end{theorem}
\begin{proof}
	Рассмотрим фундаментальную последовательность $x_n \in C(\overline \Omega)$.
	$$
	\forall \varepsilon > 0 \exists N: \forall k, n > N \|x_k - x_n\| =
	\max_{t \in \overline \Omega} |x_n(t) - x_k(t)| < \varepsilon
	$$
	
	Возьмём $t \in \overline \Omega$. $\{x_n(t)\}$ — числовая последовательность.
	Тогда получаем $|x_n(t) - x_k(t)| < \varepsilon$, отсюда $\{x_n(t)\}$ — 
	фундаментальна, значит существует $\lim\limits_{n \to \infty} x_n(t) = x(t)$.
	
	Проверим, что $\max\limits_{t \in \overline \Omega} |x_n(t) - x(t)| \to 0$ при
	$n \to \infty$, т. е. $x_n \uto x$ на $\overline \Omega$.
	Заметим, что $\forall k, n > N |x_k(t) - x_n(t)| < \varepsilon \implies
	|x(t) - x_n(t)| \leqslant \varepsilon$.
	
	Почему же $x$ непрерывна? Потому что равномерный предел непрерывных функций 
	непрерывен.

\end{proof}

Пусть $[a, b] \subset \real$ Рассмотрим пространство дифференцируемых функций $C^1[a, b]$. 
Какую норму на нём выбрать?
\begin{itemize}
	\item $\varphi_1(x) = \max\limits_{t \in [a, b]} |x(t)|$
	\item $\varphi_2(x) = \max\limits_{t \in [a, b]} |x'(t)|$
	\item $\varphi_3(x) = \varphi_1(x) + \varphi_2(x)$
	\item $\varphi_4(x) = |x(a)| + \max\limits_{t \in [a, b]} |x'(t)|$
\end{itemize}

Заметим, что $\varphi_2$ нормой вообще не является, а $\varphi_1$ не даёт полноты 
пространства.

\begin{theorem}
	\begin{enumerate}
		\item Пространство $(C^1 [a, b], \varphi_1)$ не полно;
		\item Пространство $(C^1 [a, b], \varphi_3)$ полно;
	\end{enumerate}
\end{theorem}

\begin{proof}
	Докажем первое утверждение.
	
	\emph{Первый аргумент}. $x$ — производная непрерывная на $[a, b]$, негладкая.
		По теореме Вейерштрасса для любого $\varepsilon > 0$ существует многочлен
		$P$ такой, что $\max\limits_{[a, b]} |P - x| < \varepsilon$
	
	\emph{Второй аргумент}. Пусть $[a, b] = [-1, 1]$, $x(t) = |t| \notin C^1[a, b]$,
	$x^\varepsilon(t) = |t|^{1 + \varepsilon} \in C^1[a, b]$. $\max |x(t) - x^\varepsilon(t)| \to 0$ при $\varepsilon \to 0$.
	
	Для доказательства второго утверждения возьмём $x_n \in C^1[a, b]$ — 
	последовательность, фундаментальную относительно $\varphi_3$.
	$$
	\varphi_3(x_n -x_k) \to 0\text{ при } n, k \to \infty \implies
	\begin{cases}
		\varphi_1(x_n - x_k) \to 0\\
		\varphi_2(x_n - x_k) \to 0
	\end{cases}
	\implies \exists x \in C[a, b], y \in C[a, b]
	$$
	
	$$
	\begin{cases}
		\varphi_1(x_n - x) \to 0 \iff x_n \uto x \text{ на } [a, b]\\
		\varphi_1(x'_n - y) \to 0 \iff x'_n \uto y \text{ на } [a, b]
	\end{cases}
	\implies x \in C^1[a, b], x' = y
	$$
	
	Отсюда $\varphi_3(x_n - x) \to 0$
\end{proof}

\end{document}